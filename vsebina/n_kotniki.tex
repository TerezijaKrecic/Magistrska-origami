\section{Konstrukcija pravilnih $n$-kotnikov}.

Iz evklidske geometrije vemo, da lahko pravilni $n$-kotnik narišemo z neoznačenim ravnilom in šestilom natanko tedaj, ko je število $n$ oblike $ n = 2^r(2^{2^s} + 1) $, kjer sta $r$ in $s$ nenegativni celi števili, število $2^{2^s} + 1$ pa je praštevilo\footnote{Fermat je domneval, da je vsako število oblike $2^{2^s} + 1$ praštevilo. To pa ni res. Že Euler je ugotovil, da je število $2^{2^5} + 1$ sestavljeno. Deljivo je s številom 641.}~\cite[str.\ 78]{jerman1998}.

\textcolor{red}{Kdaj so n-kotniki konstruktibilni? izrek, dokaz???}
