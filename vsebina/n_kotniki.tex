\section{Konstrukcija pravilnih $n$-kotnikov}

% konstrukcija enakostraničnega trikotnika, ki ga lahko dobimo na več načinov
% konstrukcije tudi kakih drugih pravilnih $n$-kotnikov

Izhajamo iz izreka~\ref{izr:evkl_konstr}.

Izrek: Pravilni $n$-kotnik lahko narišemo le s šestilom in ravnilom natanko tedaj, ko je število $n$ oblike $ n = 2^r(2^{2^s} + 1) $, kjer sta $r$ in $s$ nenegativni celi števili, število $2^{2^s} + 1$ pa je praštevilo.

Fermat je domneval, da je vsako število oblike $2^{2^s} + 1$ praštevilo. To pa ni res. Že Euler je ugotovil, da je število $2^{2^5} + 1$ sestavljeno. Deljivo je s številom 641.

To vse je iz~\cite[str.\ 78]{jerman1998}.