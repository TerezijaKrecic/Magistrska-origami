\section{Zaključek}

Po zgledu evklidskih konstrukcij smo v nalogi spoznali, da lahko geometrijske in algebraične probleme rešujemo tudi s prepogibanjem papirja, pri čemer je origami celo močnejše orodje od neoznačenega ravnila in šestilo. Poleg kosntrukcij vseh razdalj, ki jih lahko opravijo z evklidskih orodjem, znamo tudi z origamijem zrcaliti in vrteti točke ter prenašati razdalje, korak naprej iz evklidskih konstrukcij pa je zmožnost izvajanja operacije kubičnega kubiranja in trisekcije poljubnega kota ter prepogibanja skupnih tangent na nekatere stožnice.

Zaradi tega nam prepogibanje papirja lahko reši tudi kubične in kvartične enačbe ter konstruira več pravilnih večkotnikov kot nam ga lahko evklidsko orodje. Zaslugo za nadvlado origamija nad njim ima Belochin pregib, ki edina od origami operacij ne more biti konstruirana z neoznačenim ravnilom in šestilom.

V nalogi smo se osredotočili le na origami z enkratnimi prepogibi. Če bi dovolili opravljanje več pregibov zapored in šele nato razgrnitev papirja, se nam s tem odpre še veliko več možnosti za raziskovanje, kaj lahko z origamijem v matematiki počnemo. Med drugim lahko poljuben kot razdelimo na pet skladnih delov in rešujemo enačbe pete stopnje. Znanost se v zadnjih petdesetih letih z origamijem kot pripomočkom za raziskovanje mateamtike veliko ukvarja in tudi aktivno aplicira v naš vsakdan. Primer je efektivno zlaganje objektov samih vase, med drugim satelitov, da je strošek pošiljanja v vesolje čimmanjši.

Origami je odličen primer, kako se lahko v običajnih stvareh okoli nas skriva čista matematika. Ker je papir s prepogibi in njihovimi presečišči odličen model evklidske ravnine, je kot didaktičen pripomoček za popestritev pouka origami zelo priporočljiv. Učenci lahko z njim preko različnih pristopov raziskujejo matematiko in razvijajo abstraktno in kritično mišljenje. Iz opazovanja konstrukcije lahko sami postavljajo hipoteze in jih dokažejo ali ovržejo; lahko sami raziskujejo, kako s prepogibanjem pokazati kakšne geometrijske lastnosti. Učitelj jih mora znati usmerjati, hkrati pa jim dati svobodo raziskovanja, saj se ravno s tem krepi samostojnost, odgovornost in veselje do izzivov.