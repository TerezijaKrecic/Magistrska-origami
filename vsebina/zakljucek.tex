\section{Zaključek}

Prepogibanje z enkratnimi pregibi, brez svinčnika. Znamo zrcalit, rotirati točke, prenašati razdalje (simulacija krožnice).

Origami konstrukcije so močnejše od evklidskih -- medtem ko lahko z neoznačenim ravnilom in šestilom konstruiramo rešitve poljubne kvadratne enačbe z racionalnimi koeficienti (oz.\ poljubna števila oblike $a + b\sqrt{r}; a,b,r \in \Q$), zmoremo s prepogibanjem papirja poleg tega reševati še kubične in kvartične enačbe, tretjiniti poljubne kote, konstruirati tretje korene poljubnih origami-konstruktibilnih števil ter \textcolor{red}{konstruirati $N$-kotnike za vsak $N$ oblike $2^i 3^j (2^k 3^l+1)$, kjer je število v oklepaju praštevilo}. Zaslugo za nadvlado origamija nad evklidskim orodjem ima Belochin pregib, ki edina od origami operacij ne more biti konstruirana z neoznačenim ravnilom in šestilom.

Zelo didaktičen pripomoček za popestritev pouka in pokazat, kako se matematiki skriva v malih stvareh, kako jo opaziti, kako sam izpeljati stvari, raziskovati, kako konstrukcija deluje (preko podvprašanj ipd.\ kot ima hull2013).