\section{Reševanje nerešljivih starogrških problemov}
\label{pogl:starogrskiproblemi}

% - trisekcija kota (Abe, Justin)
% - podvojitev kocke (Messer, Beloch)

% Naštej vse tri. Onega od $\pi$ se ne da rešiti?
% reševanje dveh starogrških problemov, ki ju z evklidskimi orodji -- dokazano -- ne znamo rešiti; to sta \emph{podvojitev kocke} (oz. konstrukcija $\sqrt[3]{2}$) in \emph{trisekcija kota}. Izkaže se, da se da vsakega od njiju rešiti celo na več kot en način! (to je omenjeno v 2.3, tko da trisekcija mora imet več načinov, čene popravi)


% Geometric Constructions str. 29 -- legenda od Apolonu, ki je zahteval 2x večji oltar, da prežene kugo

\subsection{Trisekcija kota}

% SPODNJI DOKAZ JE PREPISAN IZ VIRA IN SAMO CITIRAN V POGLAVJU 2.3

% Algebraični dokaz, da z evklidskim orodjem ne moremo tretjiniti poljubnega kota -- dokažimo za kot $60°$. (iz~\cite[str.\ 77--78]{jerman1998})

% Kot $60°$ znamo narisati. Če bi ga znali razdeliti na tri enake dele, bi potemtake znali narisati tudi kot $20°$, s tem pa (ker znamo risati pravokotnice) tudi $\cos 20°$ \textcolor{red}{slika z enotsko krožnico}. Pokažimo, da to ne gre.

% Izračunajmo minimalni polinom števila $\cos 20°$. Ker je
% $$ \frac{1}{2} = \cos 60° = \cos(3 \cdot 20°) = 4 \cos^3 20° - 3 \cos 20°, $$
% ima polinom $ p(x) = 8 x^3 - 6x - 1 $ ničlo $ \cos 20°$. Minimalni polinom števila $ \cos 20°$ deli polinom $p$. Če polinom $p$ razpade na produkt dveh polinomov s koeficienti v $\Q$, je eden od polinomov zagotovo linearen. To pa bi pomenilo, da ima polinom $p$ vsaj eno racionalno ničlo. Edini kandidatki za racionalne ničle polinoma $p$ so števila iz množice
% $$ \{\pm 1, \pm \frac{1}{2}, \pm \frac{1}{4}, \pm \frac{1}{8} \}. $$4Nobeno od teh števil ni ničla polinoma $p$, zato se $p$ ne da razcepiti na produkt dveh polinomov z racionalnimi koeficienti. Minimalni polinom števila $ \cos 20°$ je torej enaka
% $$ m(x) = \frac{1}{8} p(x) = x^3 - \frac{3}{4} x - \frac{1}{8}. $$
% Tako je razsežnost prostora $\Q(\cos 20°)$  nad obsegom $\Q$ enaka $3$ in števila $ \cos 20° $ se ne da narisati le z ravnilom in šestilom.

% Zato trisekcija kota v splošnem ni mogoča.

Neka konstrukcija je v~\cite[str.\ 155]{geometricconstructions}.

\subsection{Podvojitev kocke}

V prostoru imamo kocko. Ali se da samo z ravnilom in šestilom narisati stranico kocke, ki ima dvakrat večjo prostornino kot dana kocka?

Če je stranica kocke dolga 1, je stranica podvojene kocke dolga $\sqrt[3]{2}$. Ker je obseg $\Q(\sqrt[3]{2})$ vektorski prostor razsežnosti $3$ nad obsegom $\Q$ (enačba $ x^3 - 2 = 0 $ nima racionalne rešitve), podvojitev kocke ni mogoča~\cite[str. 78]{jerman1998}.