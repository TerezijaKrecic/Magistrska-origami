\section{Reševanje enačb}
\label{pogl:enacbe}


\textcolor{red}{popravi da so captioni slik, ki so v več kot eni vrstici, poravbnani na sredini}

% - enačbe 4. stopnje (TEŽJE) --> Alhazen's problem (gl.\ članek od Vavšetiča in tudi~\cite[str.\ 137--139]{geometricconstructions})
% - enačbe 5. stopnje (TEŽJE)

Zapustimo deloma področje geometrije in si poglejmo, kako lahko s prepogibanjem papirja rešujemo enačbe z racionalnimi koeficienti.

Spomnimo se še, da smo origami-konstruktibilna števila definirali kot vsa števila, ki jih lahko s prepogibanjem konstruiramo preko na začetku dane abscine osi, izhodišča $(0,0)$ in točke $(1,0)$ in da lahko kar predpostavimo, da imamo dan celoten koordinatni sistem z abscisno in ordinatno osjo, izhodiščem ter enoto $1$ na obeh oseh (definicija~\ref{def:origami_konstruktibilnost}). V tej ravnini bomo konstruirali rešitve naših enačb, da pa bo pregibov čim manj in s tem preglednost večja, lahko pomožne točke in premice narišemo kar s svinčnikom (ker bi jih tako ali tako znali konstruirati s pregibi).

V tem poglavju si bomo pogledali, kako z origamijem na prefinjen način -- brez konstrukcije preko zgornjih petih operacij -- rešujemo enačbe druge, tretje, četrte in pete stopnje \textcolor{red}{(Kej tudi za peto stopnjo? Čenje mal zmanjšej teve oznake)}. Za vsak $n \in \{2, 3, 4, 5\}$ bomo reševali enačbo
$$ a_n x^n + a_{n-1} x^{n-1} + \ldots + a_2 x^2 + a_1 x + a_0 = 0, $$
kjer so $a_i \in \Q$ za vsak $i \in \{1, 2, \ldots, n\}$ in $a_n \neq 0$. Ker bo $i$ največ $5$, bomo namesto koeficientov $a_i$ uporabljali kar črke $a, b, c, d, e, f$.

Začnimo z najbolj osnovno, t.\ j.\ linearno enačbo. Enačba $ax + b = 0$ ima rešitev $x = -b/a$, ki je racionalno število in po izreku~\ref{izr:origami_konstruktibilnost} origami-konstruktibilno. Če bi želeli rešitev konstruirati brez računanja. bi lahko v ravnini prepognili premico $y = ax + b$ (konstruiramo npr.\ točki (0, b) in (1, a+b) in skoznju prepogneno premico) in njeno presečišče z abscisno osjo je iskana rešitev. Vendar je ta konstrukcija bolj zamudna kot direktna konstrukcija izračunane rešitve.

Za rešitev kvadratne enačbe lahko uporabimo znanje iz podpoglavja~\ref{podpogl:evkl_konstruktibilnost} -- vemo, da lahko z neoznačenim ravnilom in šestilom konstruiramo natanko števila oblike $a + b\sqrt{r}$, kjer so $a, b, r \in \Q$. Take oblike je tudi splošna rešitev kvadratne enačbe z racionalnimi koeficienti in ker lahko ta števila konstruiramo tudi z origamijem, lahko preko kvadratne formule najprej izračunamo realni rešitvi in ju nato s prepogibanjem papirja preko operacij seštevanja, odštevanja, množenja, deljenja in korenjenja konstruiramo (izrek~\ref{izr:origami_konstruktibilnost}). Zanima pa nas, ali jih je mogoče konstruirati tudi brez predhodnega računanja.

Ključno vlogo bosta v nadaljevanju odigrali origami operaciji~\ref{op:O6} in~\ref{op:O7}. Prva nam hkrati s konstrukcijo tangente na parabolo določi tudi točko na paraboli, skozi katero je pregib tangenten na stožnico, to pa je ekvivalentno reševanju kvadratne enačbe. Druga pa s konstrukcijo skupne tangente na dve paraboli omogoča reševanje kubične enačbe, saj imata paraboli v evklidski ravnini največ tri skupne tangente (trditev~\ref{trd:stevilo_skupnih_tangent} v nadaljevanju). Znano je tudi, da lahko reševanje kvartične enačbe prevedemo na reševanje kubične ali celo kvadratne (\textcolor{red}{dej vir, npr.\ uni članek v zavihku z enačbami}), zato nam origami rešuje tudi enačbe četrte stopnje. To je že skoraj 100 let nazaj odkrila italijanska matematičarka Margherita P.\ Beloch, ki je s tem odkrila pravo moč prepogibanja papirja.

Za rešitev enačbe pete stopnje pa bomo morali prekršiti pravilo o enkratnih prepogibih. Namreč \textcolor{red}{(preveri)} ni načina, kako s po enimi pregibom naenkrat konstruirati rešitev, \textcolor{red}{DOKONČAJ -- Alperin, Lang, Lucero so neki delali, 2-fold origami ???}

\subsection{Kvadratna enačba}
\label{podpogl:kvadratna_enacba}

Rešujemo enačbo oblike
$$ a x^2 + b x + c = 0, $$
kjer so $a, b, c \in \Q$ in velja $a \neq 0$.  Njeni splošni rešitvi sta
$$ x_{1,2} = \frac{-b \pm \sqrt{b^2 - 4ac}}{2a},$$ vendar so konstrukcije vseh teh števil časovno potratno, zato iščemo hitrejši način reševanja.

Postopek, ki si ga bomo pogledali v nadaljevanju, predpostavlja $a = 1$. Ker je vodilni koeficient neničeln, lahko z njim enačbo delimo in pri tem še vedno dobimo racionalne koeficiente, zato lahko predpostavko brez škode za splošnost sprejmemo. Nova oblika enačbe je tako
\begin{equation}
    \label{eq:spl_kv_en}
    x^2 + bx + c = 0.
\end{equation}
Predpostavimo, da ima enačba dve različni realni rešitvi oz.\ da je diskriminanta enačbe pozitivna, t.\ j.\ $D = b^2 - 4c > 0$. Če realnih ničel ni, o origami konstrukciji rešitev namreč nima smisla razpravljati. Če je rešitev ena, je podana kot $x = -b/2$, kar je origami-konstruktibilno število in se ga lahko takoj konstruira.

Enačba~\ref{eq:spl_kv_en} nam poda pokončno parabolo $y = x^2 + bx + c$ z vodoravno premico vodnico in dvema ničlama, ki sta rešitvi naše enačbe. Iščemo absciso presečišča parabole z abscisno osjo.

Zopet se bomo poslužili dosedanjega znanja o operaciji~\ref{op:O6}. Ta nam s pregibom skozi dano točko $B$, ki točko $A$ položi na premico $a$, konstruira tangento na parabolo z goriščem v točki $A$ in premico vodnico $a$.

Naša parabola je z enačbo seveda natančno določena. Ideja iskane konstrukcije rešitev enačbe je določiti tako točko $B$ (najlažje kar na osi parabole), da bi nam izvedba operacije~\ref{op:O6} podala tangento na parabolo ravno v njeni ničli. Želeni pregib mora potekati skozi točko $B$ in gorišče $A$ položiti na tisto točko $A'$ na premici vodnici $a$, ki ima enako absciso kot ničla parabole. (gl.\ sliko~\ref{fig:tockaB_in_O6}). Taka točka $B$ je z osjo parabole in katerokoli izmed ničlama (zaradi simetrije) natanko določena.

\begin{figure}[h]
    \centering
    \includegraphics[width=0.5\textwidth]{images/kvadratna_enacba/tockaB_in_O6.png}
    \caption[Iskanje točke $B$]{Operacijo~\ref{op:O6} skozi iskano točko $B$ poda rešitev kvadratne enačbe.}
    \label{fig:tockaB_in_O6}
\end{figure}

Edina nevarnost, da ta konstrukcija ne bo delovala, je možnost, da točka $B$ kdaj ne bo origami-konstruktibilna točka. Zato sedaj izračunajmo njene koordinate in se prepričajmo, da se to nikoli ne bo zgodilo.

Najprej iz dane enačbe parabole določimo njeno gorišče $A$ in premico vodnico $a$. Spomnimo se, da iz enačbe parabole oblike
$$ (x - x_0)^2 = 2p(y - y_0) $$
takoj razberemo koordinati gorišča $(x_0, y_0)$ in enačbo premice vodnice $y = y_0 - p$. V našem primeru enačbo $y = x^2 + bx + c$ preoblikujemo v
$$ \left(x-\left(-\frac{b}{2}\right)\right)^2 = 2 \cdot \frac{1}{2} \left(y - \left(c - \frac{b^2}{4}\right)\right). $$
S tem sta gorišče $A$ in premica vodnica $a$ določena:
$$ A\left(-\frac{b}{2}, c - \frac{b^2 - 1}{4}\right) \text{ in } a: y = c - \frac{b^2 + 1}{4}. $$

Naj bo $t$ ena izmed rešitev enačbe~\ref{eq:spl_kv_en}. Na premici $a$ z $A'$ označimo točko z absciso $t$. Poiščimo enačbo pregiba, ki gorišče $A$ položi v točko $A'$. Ta pregib bo tangenten na parabolo ravno v njeni ničli, njegovo presečišče z osjo parabole $ x = -b/2 $ pa nam bo določilo točko $B$.

Koeficient nosilke daljice $AA'$ je $ - 1/(2t + b)$, torej je koeficient pregiba $k = 2t + b$. Pregib je po konstrukciji tangenten na parabolo v ničli $(t, 0)$, torej je njegova enačba
$$ y = (2t + b)(x - t) = (2t + b)x - 2t^2 - bt = (2t + b)x - t^2 + c. $$
Pri tem smo upoštevali, da velja $t^2 + bt + c = 0$. Presečišče pregiba in osi parabole je tako točka $B$ z absciso $ x = -b/2 $ in ordinato
$$ y = (2t + b)\left(-\frac{b}{2}\right) - t^2 + c = - t^2 - tb + c - \frac{b^2}{2} = c + c - \frac{b^2}{2} = 2c - \frac{b^2}{2}.$$
Obe koordinati sta racionalni, torej je točka $B$ konstruktibilna točka. Ker leži na osi parabole, nam poda obe rešitvi enačbe -- pregiba sta si simetrična glede na os. Povzemimo sedaj postopek konstrukcije rešitve kvadratne enačbe~\ref{eq:spl_kv_en}:
\begin{enumerate}
    \item V koordinatnem sistemu označimo gorišče $A\left(-\frac{b}{2}, c - \frac{b^2}{4} + \frac{1}{4}\right)$, premico vodnico $a: y = c - \frac{b^2 + 1}{4}$ in točko $B(-\frac{b}{2}, 2c - \frac{b^2}{2})$.
    \item Z operacijo~\ref{op:O6} naredimo pregib skozi točko $B$, ki točko $A$ položi na premico $a$ (če je diskriminanta enačbe pozitivna, sta možna pregiba dva).
    \item Skozi sliko točke $A$ naredimo vertikalen pregib in abscisa njegovega presečišča z abscisno osjo je ničla dane enačbe.
\end{enumerate}

\textbf{Primer:} Poiščimo rešitve enačbe $x^2 - x - 1 = 0$. Določimo obe točki in premico: $A(\frac{1}{2}, -1)$, $B(\frac{1}{2}, -\frac{5}{2})$ in $a: y = -\frac{3}{2}.$. Opravimo operacijo~\ref{op:O6} in označimo presečišče abscisne osi in pravokotnice nanjo skozi sliko točke $A$. Če smo bili pri pregibanju natančni, dobimo presečišči pri $x_{1,2} = \frac{1 \pm \sqrt{5}}{2}$ (gl.\ sliko v~\cite[str.\ 37]{hull2020}).

To še zdaleč ni edini postopek za reševanje kvadratne enačbe. Kot še en lep primer Hull v~\cite[str.\ 38]{hull2020} navaja Lillovo konstrukcijo preko krožnice, lahek dokaz pa je prepuščen bralcu. Hkrati je to primer, kako za rešitev nekega problema najprej najdemo (bolj domačo) evklidsko konstrukcijo, ki jo lahko nato preko origami operacij preobrazimo v origami konstrukcijo -- saj že vemo, da lahko s prepogibanjem papirja konstruiramo vse in še več, kar se da z evklidskim orodjem. Pri obravnavi kubične enačbe bomo spoznali Belochino metodo, ki se jo da aplicirati tudi na kvadratno enačbo, in prilagojen postopek je tako opisan v razdelku~\ref{podpodl:kvadr_en_lill}.

\subsection{Kubična enačba}
\label{podpogl:kubicna_enacba}

Rešujemo enačbo oblike
$$ a x^3 + b x^2 + c x + d = 0, $$
kjer so $a, b, c, d \in \Q$ in velja $a \neq 0$. Tu je navedena ena oblika zapisa njene splošne rešitve:

\begin{align*}
    Q &= \sqrt{(2b^3 - 9abc + 27a^2d)^2 - 4(b^2 - 3ac)^3} \\
    C &= \sqrt[3]{\frac{1}{2}(Q + 2b^3 - 9abc + 27a^2d)} \\
    x_1 &= - \frac{b}{3a} - \frac{C}{3a} - \frac{b^2 - 3ac}{3aC} \\
    x_2 &= - \frac{b}{3a} + \frac{C(1 + i\sqrt{3})}{6a} + \frac{(1 - i\sqrt{3})(b^2 - 3ac)}{6aC} \\
    x_2 &= - \frac{b}{3a} + \frac{C(1 - i\sqrt{3})}{6a} + \frac{(1 + i\sqrt{3})(b^2 - 3ac)}{6aC}
\end{align*}

Operacija~\ref{op:O6} nam je preko konstrukcije tangente na parabolo pomagala rešiti kvadratno enačbo. Spomnimo se, da je Belocheva to v 30-ih letih prejšnjega stoletja nadgradila z operacijo~\ref{op:O7}, ki nam konstruira skupno tangento na dve paraboli hkrati. Po njej jo tudi imenujemo \emph{Belochin pregib}. Z njim je kot prva odkrila resnično moč origami konstrukcij, a je žal trajalo več kot pol stoletja, da so matematiki začeli ceniti njeno odkritje.

\subsubsection{Reševanje kubične enačbe z Belochinim postopkom}

Belocheva je sama odkrila naslednjo metodo reševanja kubične enačbe, kjer nam Belochin pregib poda eno izmed rešitev. Na koncu razdelka bomo videli, da nam v postopku vsaka izmed rešitev da svoj pregib (razen če gre za večkratno ničlo) in ker so rešitve kubične enačbe največ tri, so tudi pregibi oz.\ skupne tangente na dve paraboli največ tri. Z razčlenitvijo sledečega postopka bomo tako dokazali spodnjo trditev.

\begin{trditev}
    \label{trd:stevilo_skupnih_tangent}
    Število realnih rešitev kubične enačbe je enako številu Belochinih pregibov v postopku, ki je naveden v tem poglavju.
\end{trditev}

\begin{posledica}
    V evklidski ravnini imata različni paraboli največ tri skupne tangente.
\end{posledica}

Belocheva v svojem postopku izhaja iz Lillove genialne metode iskanja ničel poljubnih polinomov z realnimi koeficienti, ki si jo bomo v naslednjem razdelku podrobneje pogledali, za njeno aplikacijo pa uporabi avtorsko konstrukcijo -- Belochin kvadrat. \textcolor{red}{zaenkrat je tukaj še splošna kubična enačba, ampak a je mogoče treba predpostavit, da noben koeficient ni ničeln?}

\subsubsection*{Lillova metoda}

Njen avtor je avstrijski inženir Eduard Lill, ki jo je l.\ 1867 opisal v svojem članku~\cite{lill1867}. Gre za inovativen postopek, ki je v svoji osnovi čisto enostaven. Imejmo poljuben polinom $ p(x) = a_n x^n + a_{n-1} x^{n-1} + \ldots + a_2 x^2 + a_1 x + a_0 $ z realnimi koeficienti in iščemo njegove realne ničle, če obstajajo. Lill je iz njegovih koeficientov s sledečim postopkom v ravnini ustvaril enolično pot. Običajno se za njeno konstrukcijo uporablja figuro želve, ki nam kaže, v katero smer se premika pa tudi kam je usmerjena.

Na začetku želvo postavimo v koordinatno izhodišče $O$ tako, da gleda v pozitivno smer $x$-osi. Želva najprej v to smer prehodi razdaljo, enako koeficientu $a_n$. Nato se obrne za $90^\circ$ v nasprotno smer urinega kazalca in prehodi naslednjo razdaljo $a_{n-1}$. To ponovi za vsak koeficient polinoma in po prehojeni razdalji $a_0$ se ustavi v neki točki $T$ (slika~\ref{fig:primera_zelve}). Če je kateri od koeficientov negativen, želva hodi ritensko (primer (b) na sliki~\ref{fig:primera_zelve} za koeficiente $a_3, a_2$ in $a_0$), v primeru ničelnega koeficienta pa obstoji na mestu in se samo obrne. S potjo želve dobimo lomljeno črto iz največ $n+1$ daljic, ki jih brez škode označujmo kar z njihovimi ``pripadajočimi'' koeficienti.

\begin{figure}[h]
    \centering
    \includegraphics[width=0.9\textwidth]{images/kubična enačba/primera_zelvine_poti.png}
    \caption[Primera želvine poti]{Primera želvine poti za polinoma pete stopnje. Vzeto iz~\cite[str.\ 311]{hull2011}.}
    \label{fig:primera_zelve}
\end{figure}

Sedaj se v izhodišče $O$ postavimo še mi in z laserskim žarkom poskusimo zadeti želvo v točki $T$. Žarek najprej usmerimo daljico $a_{n-1}$, od katere se odbije v daljico $a_{n-2}$, od te v daljico $a_{n-3}$ in tako naprej. (slika~\ref{fig:primera_zelve}). Pri tem upoštevamo troje:
\begin{itemize}
    \item laserski žarek ne upošteva odbojnega zakona in se od daljice vedno odbije pod kotom $90^\circ$, zato so vpadni koti žarka na vse daljice med seboj enaki in prav tako to velja za odbojne kote;
    \item žarek se lahko odbije tudi od nosilke daljice;
    \item vsakič sta možni dve smeri odboja -- na isto stran daljice (oz.\ njene nosilke) ali skoznjo -- izberemo pa tisto, ki nam omogoči, da sploh lahko zadenemo naslednjo daljico.
\end{itemize}
ecimo, da smo zmogli zadeti želvo. Kot, ki ga v točki $O$ oklepata laserski žarek in abscisna os, označimo z $\theta$.

\begin{trditev}
    $x_{\theta} = - \tan \theta$ je ničla polinoma $p(x)$.
\end{trditev}

\begin{dokaz}
    Vzemimo primer, ko so vsi koeficienti polinoma $p(x)$ pozitivni. Želvina pot je v tem primeru sestavljena iz $n+1$ daljic, pot laserskega žarka (ki se vedno odbije od daljice in ne njene nosilke) pa iz $n$ daljic. Slednje so ravno hipotenuze pravokotnih trikotnikov. Za vsako od njih je nasprotna kateta kota $\theta$ del daljice $a_i$, priležno kateto pa označimo z $y_i$ ($ n \geq i \geq 1$). dobimo
    \begin{align*}
        y_n &= \tan \theta \cdot a_n = - x_{\theta} a_n \\
        y_{n-1} &= \tan \theta \cdot (a_{n-1} - y_n) = - x_{\theta} (a_{n-1} + x a_n) = - (a_{n-1} x_{\theta} + a_n x_{\theta}^2)\\
        y_{n-2} &= \tan \theta \cdot (a_{n-2} - y_{n-1}) = - x_{\theta} (a_{n-2} + a_{n-1} x_{\theta} + a_n x_{\theta}^2) = \\
        &= - (a_{n-2} x_{\theta} + a_{n-1} x_{\theta}^2 + a_n x_{\theta}^3) \\
        &\vdots \\
        y_1 &= - (a_1 x_{\theta} + a_2 x_{\theta}^2 + \ldots + a_{n-1} x_{\theta}^{n-1} + a_n x_{\theta}^n).
    \end{align*}
    V zadnji enakosti desno stran premaknimo na levo in upoštevamo $y_1 = a_0$. Dobimo ravno $p(x_{\theta}) = 0$, torej je $x_{\theta} = - \tan \theta$ res ničla tega polinoma.

    \textcolor{red}{Primer negativnih koeficientov:~\cite[str.\ 36]{zore2022}.}

    \textcolor{red}{Primer ničelnih koeficientov: isto kot prej, samo se spusti $y_i$ za tisti $i$, za katerega je $a_i = 0$. (\textcolor{red}{???})}
\end{dokaz}

Če pod nobenim kotom $\theta$ ne moremo zadeti želve, je polinom $p(x)$ brez realnih ničel.

Pojavi se nam vprašanje, kako določiti kot $\theta$. Za polinom tretje stopnje je Belocheva preko svojega pregiba našla zelo preprosto rešitev, ki si jo bomo sedaj pogledali.

\subsubsection*{Belochin kvadrat}

Imejmo dani točki $A$ in $B$ ter premici $r$ in $s$. Z origamijem konstruirajmo kvadrat $WXYZ$, kjer oglišče $X$ leži na premici $r$, njegovo sosednje oglišče $Y$ pa na premici $s$. Velja še, da točka $A$ leži na stranici $WX$ (ali njeni nosilki), točka $B$ pa na stranici $ZY$ (ali njeni nosilki, slika~\ref{fig:beloch_kvadrat}).

\begin{figure}[h]
    \centering
    \includegraphics[width=0.4\textwidth]{images/kubična enačba/beloch_kvadrat.png}
    \caption[Belochin kvadrat]{Belochin kvadrat. Vzeto iz~\cite[str.\ 309]{hull2011}.}
    \label{fig:beloch_kvadrat}
\end{figure}

Belocheva je iznašla naslednji postopek, ki nam konstruira ta kvadrat:
\begin{itemize}
    \item Najprej konstruiramo premico $r'$, ki je vzporedna premici $r$ in od nje enako oddaljena kot točka $A$, tako da premica $r$ leži med točko $A$ in premico $r'$. Na enak način premici $s$ konstruiramo njeno vzporednico $s'$ (slika~\ref{fig:beloch_kvadrat_konstrukcija} levo). To konstrukcijo opravimo s prepogibi iz operacije~\ref{op:O5}, zrcaljenja točke čez premico ter ponovne uporabe operacije~\ref{op:O5}. Zaradi preglednosti seveda dopuščamo, da namesto zrcaljenja preprosto prepognemo po premici in s svinčnikom označimo sliko točke.
    \item Nato opravimo Belochin pregib, ki točko $A$ slika v točko $A'$ na premici $r'$, točko $B$ pa v točko $B'$ na premici $s'$ (slika~\ref{fig:beloch_kvadrat_konstrukcija} na sredi).
    \item Naj bo točka $X$ središče daljice $AA'$ in točka $Y$ središče daljice $BB'$. Ker je pregib simetrala teh dveh daljic $AA'$ in $BB'$, sta njuni središči po konstrukciji\footnote{Gledamo lahko dva podobna pravokotna trikotnika s skupnim ogliščem v točki $A$ (oz.\ $B$), enega dvakrat večjega od drugega} ravno presečišči pregiba s premicama $r$ in $s$ (slika~\ref{fig:beloch_kvadrat_konstrukcija} desno).
    \item Daljica $XY$ -- ena izmed stranic kvadrata -- je po konstrukciji pravokotna na daljici $AX$ in $BY$, zato samo še določimo točki $W$ in $Z$ na daljicah ali njunih nosilkah in tako dobimo Belochin kvadrat.
\end{itemize}

\begin{figure}[h]
    \centering
    \includegraphics[width=0.95\textwidth]{images/kubična enačba/beloch_kvadrat_konstrukcija.png}
    \caption[Konstrukcija Belochinega kvadrata]{Konstrukcija Belochinega kvadrata z origamijem. Vzeto iz~\cite[str.\ 310]{hull2011}.}
    \label{fig:beloch_kvadrat_konstrukcija}
\end{figure}

\subsubsection*{Konstrukcija $\sqrt[3]{2}$ z Belochinim kvadratom}
\label{podpogl:beloch_kvadrat_koren}

Preden ravno naučeno znanje uporabimo za reševanje kubičnih enačb, si še na hitro poglejmo, kako lahko tudi z Belochinim kvadratom rešimo starogrški problem podvojitve kocke.

Za premico $r$ vzemimo ordinatno os, za premico $s$ pa abscisno os. Določimo še $A = (-1,0)$ in $B = (0, -2)$. Vzporednici sta torej $r': x = 1$ in $s': y = 2$. Belochin pregib seka premico $r$ v točki $X$, premico $s$ pa v točki $Y$ (slika~\ref{fig:beloch_koren}). Z $O$ označimo koordinatno izhodišče in opazimo podobne pravokotne trikotnike $OAX$, $OXY$ in $OYB$. Z upoštevanjem $|AO| = 1 $ in $|OB| = 2$ dobimo sledeča razmerja:
$$ \frac{|OX|}{|AO|} = \frac{|OY|}{|OX|} = \frac{|OB|}{|OY|} \Longrightarrow |OX| = \frac{|OY|}{|OX|} = \frac{2}{|OY|}, $$
iz česar sledi
$$ |OX|^3 = |OX| \cdot \frac{|OY|}{|OX|} \cdot \frac{2}{|OY|} = 2 \Longrightarrow |OX| = \sqrt[3]{2}. $$

\begin{figure}[h]
    \centering
    \includegraphics[width=0.5\textwidth]{images/kubična enačba/beloch_koren.png}
    \caption[Konstrukcija kubičnega korena števila dva]{Konstrukcija $\sqrt[3]{2}$ preko Belochinega kvadrata. Vzeto iz~\cite[str.\ 310]{hull2011}.}
    \label{fig:beloch_koren}
\end{figure}

Vidimo lahko, da je to enaka konstrukcija kot jo je 50 let kasneje neodvisno od Belocheve odkril G.\ Martin (razdelek~\ref{podpogl:podvojitev_kocke}), le da je za točko $B$ vzel točko $(0, -k)$ in s tem konstruiral dolžino $\sqrt[3]{k}$ za poljuben origami-konstruktibilen $k$.

\subsubsection*{Združitev Lillove metode in Belochinega kvadrata}

Za poljubno enačbo $a x^3 + b x^2 + c x + d = 0$, kjer $ a \neq 0$, povežimo sedaj Lillovo metodo s konstrukcijo primernega Belochinega kvadrata, ki nam bo natančno določil kot $\theta$. Najprej konstruiramo želvino pot za polinom $p(x) = a x^3 + b x^2 + c x + d$. V primeru neničelnih koeficientov je pot sestavljena iz štirih stranic, pot laserskega žarka pa iz treh.

Za točko $A$ vzemimo izhodišče $O$, za točko $B$ pa končno točko $T$. Premica $r$ naj bo nosilka daljice $b$, premica $s$ pa nosilka daljice $c$. Ko si določimo premici $r'$ in $s'$ ter opravimo Belochin pregib, dobimo, kot običajno, na njegovih presečiščih s premicama $r$ in $s$ točki $X$ in $Y$. Zarišemo daljice $AX$, $XY$ in $YB$. Ker po konstrukciji velja $ AX \perp XY \perp YB $, je to iskana pot laserskega žarka, ki se odbija pod pravim kotom in zadene želvo. Kot $\theta$ je kot, ki ga oklepata daljici $a_3$ in $AX$ (slika~\ref{fig:beloch_kubicna_resitev}). Rešitev je torej $x_{\theta} = - \tan \theta$.

\begin{figure}[h]
    \centering
    \includegraphics[width=0.4\textwidth]{images/kubična enačba/beloch_kubicna_resitev.png}
    \caption[Lillova metoda z Belochinim kvadratom]{Konstrukcija želvine poti za Lillovo metodo preko Belochinega kvadrata. Vzeto in preurejeno iz~\cite[str.\ 313]{hull2011}.}
    \label{fig:beloch_kubicna_resitev}
\end{figure}

Če ima enačba še eno ali dve realni rešitve, sta možna še en ali dva Belochina pregiba. S tem smo pokazali, da trditev~\ref{trd:stevilo_skupnih_tangent} (in njena posledica) velja.

\opomba{V resnici nikoli do sedaj nismo potrebovali konstruirati celega kvadrata; potrebovali smo le stranico $XY$ in dejstvo, da je pregib pravokoten na daljici $AX$ in $BY$.}

\textcolor{red}{Kaj je v primeru, ko je kakšen koeficient ničeln?}

Za konkretne primere uporabe Belochinega postopka za reševanje kubičnih enačb glej~\cite[38--44]{zore2022}.

Kot zanimivost Lavričeva v~\cite[str.\ 10--13]{lavric2013} s postopkom, ki je malo preurejen Belochin postopek, še analitično pokaže, da je ob primerno izbranih točkah $A$ in $B$ ter premicah $r$ in $s$ koeficient tangentnega pregiba rešitev kubične enačbe. Točki in premici izbere tako, da sta točki $X$ in $Y$ ravno presečišči z ordinatnima osema, iz česar lahko takoj razberemo koeficient tangente. V dokazu izpelje enačbi pripadajočih parabol in splošno enačbo njunih tangent ter iz tega dokaže rečeno. To je lahko odlična vaja za dijake, ki si želijo kakšnega izziva.

\subsubsection{Reševanje kvadratne enačbe z Lillovo metodo}
\label{podpodl:kvadr_en_lill}

Lillovo lahko uporabimo tudi za reševanje kvadratne enačbe $a x^2 + b x + c = 0, a \neq 0$. Na enak način v koordinatni sistem zarišemo želvino pot, ki se začne v točki $A$ in konča v točki $B$. Za razliko od prej tu ne uporabimo Belochinega pregiba, temveč pregib iz operacije~\ref{op:O6}. Namesto dveh premic $r$ in $s$ imamo le eno -- naj bo $r$ nosilka daljice $b$. Kot prej -- na razdalji $a$ na drugi strani točke $A$ -- označimo še njeno vzporednico $r'$. Konstruiramo pregib, ki gre skozi točko $B$ in točko $A$ položi na premico $r'$. Njegovo presečišče s premico $r$ nam določi točko $X$, kjer se žarek iz točke $A$ pod pravim kotom odbije v točko $B$. Na sliki~\ref{fig:kv_en_lill} je primer kosntrukcije pri negativnem koeficientu $c$. S tem je kot $\theta$ določen. Premislili smo tudi že, da sta možna največ dva pregiba in da je število pregibov enako številu realnih rešitev enačbe.

\begin{figure}[h]
    \centering
    \includegraphics[width=0.5\textwidth]{images/kvadratna_enacba/kvadratna_enacba_lillova_metoda.png}
    \caption[Lillova metoda za kvadratno enačbo]{Reševanje kvadratne enačbe po Lillovi metodi z operacijo~\ref{op:O6} ($c < 0$).}
    \label{fig:kv_en_lill}
\end{figure}

\subsubsection{Alperinova rešitev}

\textcolor{red}{a čmo tudi to?} (gl.\ Hull 2020, hull2013 str.\ 78 spodej) - prevedba kubične enačbe na kvadratno al neki tazga.

\subsection{Kvartična enačba}

Rešujemo enačbo oblike
$$ a x^4 + b x^3 + c x^2 + d x + e = 0, $$
kjer so $a, b, c, d, e \in \Q$ in velja $a \neq 0$.

\emph{Abel-Ruffinijev izrek}, ki temelji na Galoisovi teoriji, pravi, da je štiri najvišja stopnja polinomske enačbe, za katere rešitve še obstaja splošna formula v radikalih, t.\ j.\ izrazih, ki vsebujejo korenjenje~\cite{mrinal2019}. Iz slike~\ref{fig:kvarticna_formula} je razvidno, da splošna formula ni najbolj praktična, zato se pri reševanju enačb četrte stopnje poslužujemo prevedb na enačbe nižje stopnje (glej ). Zato v tem razdelku 

\begin{figure}[h]
    \centering
    \includegraphics[width=0.95\textwidth]{images/quartic_formula.png}
    \caption[Kvartična formula]{Splošna formula za rešitev enačbe četrte stopnje.}
    \label{fig:kvarticna_formula}
\end{figure}

Kot smo omenili že na začetku poglavja, se da kvartično enačbo z uvedbami novih spremenljivk prevesti na kubično. Načinov je več, gl.\ \cite{wikiquartic}, \cite{quartics2012}.

% kot redukcija na kubično ali kvadratno??
% citiram od začetka poglavja: Znano je tudi, da lahko reševanje kvartične enačbe prevedemo na reševanje kubične ali celo kvadratne, zato nam origami rešuje tudi enačbe četrte stopnje.

\subsection{Kvintična enačba}

\textcolor{red}{Če boš to obdelala v poglavju MULTIFOLD, potem spremeni na začetku tega poglavja, da bomo gledali samo enačbo do 4.\ stopnje.}