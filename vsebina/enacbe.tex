\section{Reševanje enačb}
\label{pogl:enacbe}

% Nazadnje pa se bomo posvetili še najbolj obsežnemu poglavju, ki deloma zapusti področje geometrije. Pogledali si bomo, kako lahko s pomočjo prepogibanja papirja rešujemo kvadratne (povezava s parabolo in operacijo O6; pa to se da tudi z evklidskim orodjem) in kubične enačbe (operacija O7!), za bolj zahtevne pa bosta zanimivi podpoglavji o reševanju enačb 4.\ in 5.\ stopnje.
% Alhazijev al kej je že uni problem!!!

% - kvadratne enačbe (te se še da samo z ravnilom in šestilom, povezava s parabolo)
% - kubične enačbe
%     - z Belochinim kvadratom
%     - Lillova metoda (z Belochinim prepogibom)
%     - Alperinova rešitev (gl. Hull 2020)
% - enačbe 4. stopnje (TEŽJE) --> Alhazen's problem (gl.\ članek od Vavšetiča in tudi~\cite[str.\ 137--139]{geometricconstructions})
% - enačbe 5. stopnje (TEŽJE)

\subsection{Enačba 2.\ stopnje (kvadratna enačba)}

Z neoznačenim ravnilom in šestilom  lahko konstruiramo natanko števila oblike $a + b\sqrt{r}$, kjer so $a, b, r \in \Q$ (gl.\ podpoglavje~\ref{podpogl:evkl_konstruktibilnost}). Take oblike je tudi splošna rešitev kvadratne enačbe. Vemo, da z origamijem lahko konstruiramo še več števil kot z evklidskim orodjem, zato nam tudi prepogibanje papirja omogoča -- preko operacij seštevanja, odštevanja, množenja, deljenja in korenjenja -- konstruirati katerokoli rešitev poljubne kvadratne enačbe z racionalnimi koeficienti.

Origami operacija~\ref{op:O6} nam podaja tangento na parabolo in lahko se vprašamo, ali je mogoče rešitve poljubne kvadratne enačbe z racionalnimi koeficienti konstruirali brez predhodnega računanja po kvadratni formuli. Odgovor je pozitiven, vendar zahteva tehten premislek (\cite[str.\ 36--38]{hull2020}) \textcolor{red}{Predelaj in zapiši, tudi omeni pretvorbo evklidske konstrukcije v origami, ki je tam navedena.}

\subsection{Enačba 3.\ stopnje (kubična enačba)}

Za kubične enačbe iz parabol (kar sledi iz operacije~\ref{op:O7}) lahko gledaš~\cite[str.\ 150]{geometricconstructions}.

\subsection{Enačba 4.\ stopnje (kvartična enačba)}

\subsection{Enačba 5.\ stopnje (kvintična enačba)}