\section{Uvod}

Začnimo z odzivom mojih prijateljev in sorodnikov, ko so izvedeli, da bom v svoji magistrski nalogi pisala o origamiju. Velika večina jih je bila zelo presenečena, saj si sploh ni predstavljala, da se v prepogibanju papirja skriva matematika. Kar je razumljivo, saj običajno ljudje, ki se s to kraljico znanosti po srednješolskem izobraževanju prenehajo aktivneje ukvarjati, njenega vpliva na vse okoli nas ne opazijo.

In resnica je, da se v origamiju razkriva toliko matematike, da je v tej nalogi ni bilo mogoče zajeti v celoti. Ne da se niti oceniti, kolikšen delež je tu opisan, saj se origami ne dotika le -- že tako izjemno širokega -- področja geometrije, temveč tudi analize, teorije števil, abstraktne algebre, diferencialne topologije \ldots Prav tako njegova uporaba zajema široko polje znanosti in inženirstva -- od arhitekture in robotike do fizike in astrofizike, če naštejemo le nekaj primerov. Kdo bi si mislil, da lahko origami uporabimo za zlaganje šotorov in ogromnih kupol nad športnimi stadioni ali celo za pošiljanje solarnih objektov v vesolje?~\cite[str.\ 3--5]{hull2020}.

Origami je umetnost prepogibanja papirja, ki se razvija že več kot tisočletje (trdnih dokazov o zlaganju papirja, kot ga poznamo danes, do pred letom 1600 po Kr.\ ni). Oblikovanje oblik iz lista papirja se je do konca 20.\ stoletja hitro razširilo po vsem svetu~\cite{robinson2024}. Matematični vidik origamija je v ospredje prišel nekoliko kasneje. V 19.\ stoletju je nemški učitelj Friedrich Froebel (1782--1852) v prepogibanju papirja opazil visoko pedagoško vrednost, kar je uporabil pri poučevanju osnovne geometrije v vrtcu. Indijski matematik Tandalam Sundara Row je nato l.\ 1893 izdal obsežno knjigo \emph{Geometric Exercises in Paper Folding}~\cite{row1917}, v kateri popisuje konstrukcije raznolikih geometrijskih likov in celo krivulj. Velik prelom je dosegla italijanska matematičarka Margherita P. Beloch, ki je v 30-ih letih 20.\ st.\ odkrila, da lahko s prepogibanjem papirja rešujemo celo kubične enačbe. Vseeno je preteklo še pol stoletja, da je origami začel zanimati tudi širšo znanost, od takrat pa se je na tem področju odprlo veliko priložnosti za raziskovanje koncepta origamija in njegovo uporabo v najrazličnejših strokah~\cite[str.\ 10]{hull2020}.

Ravno uporaba origamija v pedagoške namene je tista, ki nas v tej nalogi še posebej zanima. Prepričana sem, da lahko praktična izkušnja prepogibanja papirja za namen reševanja problemov učence bolj motivira, saj je to neka nova oblika dela, ki je niso vajeni, hkrati pa zahteva spretnost in natančnost. Poleg fine motorike z origamijem krepimo tudi raziskovalno delo učencev ter odkrivanje in uporabo geometrijskih načel in pravil v praksi. Še zdaleč ne bomo zajeli vsega, kar bi lahko v šoli s prepogibanjem papirja počeli, vendar je kljub vsemu v nalogi vključenih veliko primerov, predvsem tistih iz geometrijskega področja.

Največja motivacija za to nalogo je, da je literature v slovenskem jeziku, ki vključuje uporabo origamija pri pouku matematike, zelo malo. Na to temo je spisanih nekaj člankov, seminarskih nalog ter diplomskih in magistrskih del, strokovnih knjig v slovenščini iz tega področja pa nisem našla. Ta naloga zajema predvsem uporabo origamija za namene raziskovanja geometrije ter reševanja enačb in vključuje veliko slik z orisanimi konstrukcijami. Me ddrugim je namenjena uporabi pri pouku matematike ali matematičnem krožku. Opisane matematične teme so namreč dovolj enostavne, da se jih večinoma da predelati v eni šolski uri. Zato iskreno upam, da bo naloga koristila še kateremu pedagogu, ki bi si želel svoj pouk matematike popestriti na nov in zanimiv način.

V geometriji preko Evklidovih postulatov ter uporabe znamenitih evklidskih orodij raziskujemo, kaj vse lahko v evklidski ravnini skonstruiramo brez uporabe drugih pravil ali orodij. V prvem poglavju si bomo pogledali podobnosti med evklidskimi in origami konstrukcijami in ugotovili, da lahko z origamijem konstruiramo več, kot lahko z neoznačenim ravnilom in šestilom. To bomo pokazali tudi s spustom na algebraično ozadje konstrukcij.

V drugem poglavju bomo v roke prijeli liste papirja različnih geometrijskih oblik in spoznali nekaj osnovnih konstrukcij -- preko osnovnošolskega dokazovanje lastnosti geometrijskih likov do konstrukcije treh pravilnih $n$-kotnikov; pogledali si bomo tudi vse tri Hagove izreke, s katerimi njihov avtor raziskuje, v kakšnem razmerju lahko razdelimo stranice kvadrata, če opravimo določene prepogibe, potem pa bomo to posplošili na iskanje postopka razdelitve stranice na poljubno število enakih delov. Zaključili bomo s konstrukcijo kvadratnega korena poljubnega origami števila in z (več različnimi!) origami postopki, ki nam rešijo dva starogrška problema, nerešljiva z evklidskim orodjem -- problem trisekcije kota ter podvojitve kocke, torej konstrukcije števila $\sqrt[3]{2}$.

Sledi kratko poglavje o konstrukcijah pravilnih $n$-kotnikov (\textcolor{red}{tu še ne vem, a bom vključila ali ne}). Potem sledi didaktično zanimivo področje, v katerem spoznamo (ali osvežimo spomin), kako s prepogibanjem papirja točke na določeno premico ali krožnico konstruiramo tangente na vse štiri stožnice in s tem na papirju dobimo njihov obris.

Najbolj zanimivo in obsežno poglavje pa je reševanje enačb s prepogibanjem papirja. Ko v prvem poglavju spoznamo, katera števila lahko kosntruiramo z origamijem, tu znanje nadradimo in spoznamo več metod, s katerimi lahko rešujemo splošne kvadratne in kubične enačbe ter celo nekatere enačbe četrte stopnje.

\textcolor{red}{Potem je še poglavje o Alzhazen's problem in 2-fold origami, ampak še ne vem, koliko bom tega v ključila. Nekaj bi še, ker do zdaj ni neke pretežke matematike za FMF tukej noter, ampak bom še videla. Do zdaj se ukvarjamo samo z 1-fold origamijem, torej da prepogneš in takoj poravnaš papir nazaj.}

Ker je naloga spisana z namenom morebitne uporabe pri pouku srednješolske (lahko tudi osnovnošolske) matematike, je v njej zajetih veliko tem, ki bi bile razumljive tudi bralcem, ki nimajo veliko znanja univerzitetne matematike. Vseeno se na nekaterih področjih dotaknemo tudi konceptom, ki dijakom načeloma niso znani (npr.\ afina in projektivna geometrija pri metodi reševanja enačbe četrte stopnje). \textcolor{red}{(in mogoče še potem, če vključiš zadnji dve poglavji, ki res nista enostavni)}

Zapustimo sedaj znano jezerce umetelno prepognjenih ladjic in žerjavov ter se podajmo na širne vode globokega oceana matematičnega origamija.