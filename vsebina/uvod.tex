\section{Uvod}

Začnimo z odzivom mojih prijateljev in sorodnikov, ko so izvedeli, da bom v svoji magistrski nalogi pisala o origamiju. Velika večina jih je bila zelo presenečena, saj si sploh ni predstavljala, da se v prepogibanju papirja skriva matematika. Kar je razumljivo, saj običajno ljudje, ki se s to kraljico znanosti po srednješolskem izobraževanju prenehajo aktivneje ukvarjati, njenega vpliva na vse okoli nas ne opazijo.

In resnica je, da se v origamiju razkriva toliko matematike, da je v tej nalogi ni bilo mogoče zajeti v celoti. Ne da se niti oceniti, kolikšen delež je tu opisan, saj se origami ne dotika le -- že tako izjemno širokega -- področja geometrije, temveč tudi analize, teorije števil, abstraktne algebre, diferencialne topologije \ldots Prav tako njegova uporaba zajema široko polje znanosti in inženirstva -- od arhitekture in robotike do fizike in astrofizike, če naštejemo le nekaj primerov. Kdo bi si mislil, da lahko origami uporabimo za zlaganje šotorov in ogromnih kupol nad športnimi stadioni ali celo za pošiljanje solarnih objektov v vesolje?~\cite[str.\ 3--5]{hull2020}.

Origami je umetnost prepogibanja papirja, ki se razvija že več kot tisočletje (trdnih dokazov o zlaganju papirja, kot ga poznamo danes, pred letom 1600 po Kr.\ ni). Oblikovanje oblik iz lista papirja se je do konca 20.\ stoletja hitro razširilo po vsem svetu~\cite{robinson2024}. Matematični vidik origamija je v ospredje prišel nekoliko kasneje. V 19.\ stoletju je nemški učitelj Friedrich Froebel (1782--1852) v prepogibanju papirja opazil visoko pedagoško vrednost, kar je uporabil pri poučevanju osnovne geometrije v vrtcu. Indijski matematik Tandalam Sundara Row je nato l.\ 1893 izdal obsežno knjigo z opisanimi konstrukcijami raznolikih geometrijskih likov in celo krivulj. Velik prelom je dosegla italijanska matematičarka Margherita P. Beloch, ki je v 30-ih letih 20.\ st.\ odkrila, da lahko s prepogibanjem papirja rešujemo celo kubične enačbe. Vseeno je preteklo še pol stoletja, da je origami začel zanimati tudi širšo znanost, od takrat pa se je na tem področju odprlo veliko priložnosti za raziskovanje koncepta origamija in uporabo prepogibanja v najrazličnejših strokah~\cite[str.\ 10]{hull2020}.

Ravno uporaba origamija v pedagoške namene je tista, ki nas v tej nalogi še posebej zanima. Prepričana sem, da lahko praktična izkušnja prepogibanja papirja za namen reševanja problemov učence bolj motivira, saj je to neka nova oblika dela, ki je niso vajeni, hkrati pa vključuje neko motorično aktivnost in spretnost. Poleg fine motorike krepimo tudi raziskovalno delo učencev ter odkrivanje in uporabo geometrijskih načel in pravil v praksi. Še zdaleč ne bomo zajeli vsega, kar bi lahko v šoli s prepogibanjem papirja počeli, vendar je kljub vsemu v nalogi vključenih veliko primerov, predvsem iz geometrijskega področja.

Največja motivacija za to nalogo je, da je literature v slovenskem jeziku, ki vključuje uporabo origamija pri pouku matematike, zelo malo. Na to temo je spisanih nekaj člankov in seminarskih nalog ter diplomskih in magistrskih del, strokovnih knjig iz tega področja pa nisem našla. Ta naloga zajema predvsem uporabo origamija za namene raziskovanja geometrije ter reševanja enačb in vključuje veliko slik z orisanimi konstrukcijami. Zato je tudi daljša, vendar je tako tudi zaradi namena kasnejše uporabe pri pouku matematike ali matematičnem krožku. Opisane matematične teme so namreč dovolj enostavne, da se jih večinoma da predelati v eni šolski uri. Zato iskreno upam, da bo naloga koristila še kateremu pedagogu, ki bi si želel svoj pouk matematike popestriti na nov in zanimiv način.

V geometriji preko Evklidovih postulatov ter uporabe evklidskih orodij (neoznačeno ravnilo ter šestilo) raziskujemo, kaj vse lahko v evklidski ravnini skonstruiramo brez uporabe drugih pravil ali orodij. V prvem poglavju si bomo pogledali povezavo med evklidskimi ter origami konstrukcijami in ugotovili, da lahko z origamijem konstruiramo še kaj, česar z evklidskimi orodji ne moremo. Nato si bomo v naslednjem poglavju pogledali, kako konstruiramo tangente na stožnice in zakaj konstrukcije tako delujejo. V tretjem poglavju sledi prepogibanje kvadratnega lista papirja, ki nam lahko stranice kvadrata razdeli v zanimivih razmerjih. Pogledali si bomo Hagove izreke in se naučili, kako stranico razdelimo na poljubno število enako dolgih delov. Poleg kvadrata je zanimiva tudi konstrukcija enakostraničnega trikotnika, ki ga lahko dobimo na več načinov, poleg teh pa si bomo v četrtem poglavju pogledali še konstrukcije tudi kakih drugih pravilnih $n$-kotnikov.

Po tej bolj osnovni geometriji se bomo v petem poglavju podali na vznemirljivo reševanje dveh starogrških problemov, ki ju z evklidskimi orodji -- dokazano -- ne znamo rešiti; to sta \emph{podvojitev kocke} (oz. konstrukcija $\sqrt[3]{2}$) in \emph{trisekcija kota}. Izkaže se, da se da vsakega od njiju rešiti celo na več kot en način!

Nazadnje pa se bomo posvetili še najbolj obsežnemu poglavju, ki deloma zapusti področje geometrije. Pogledali si bomo, kako lahko s pomočjo prepogibanja papirja rešujemo kvadratne in kubične enačbe, za bolj zahtevne pa bosta zanimivi podpoglavji o reševanju enačb 4.\ in 5.\ reda.

Zapustimo sedaj malo jezerce umetelno zloženih ladjic in žerjavov ter se podajmo na širne vode globokega oceana matematičnega origamija.