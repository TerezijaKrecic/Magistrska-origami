\section{Uvod}

Začnimo z odzivom mojih prijateljev in sorodnikov, ko so izvedeli, da bom v svoji magistrski nalogi pisala o origamiju. Velika večina jih je bila zelo presenečena, saj si sploh ni predstavljala, da se v prepogibanju papirja skriva matematika. Kar je razumljivo, saj običajno ljudje, ki se s to kraljico znanosti po srednješolskem izobraževanju prenehajo aktivneje ukvarjati, njenega vpliva na vse okoli nas ne opazijo.

In resnica je, da se v origamiju razkriva toliko matematike, da je v tej nalogi ni bilo mogoče zajeti v celoti. Ne da se niti oceniti, kolikšen delež je tu opisan, saj se origami ne dotika le -- že tako izjemno širokega -- področja geometrije, temveč tudi analize, teorije števil, abstraktne algebre, diferencialne topologije \ldots Prav tako njegova uporaba zajema široko polje znanosti in inženirstva -- od arhitekture in robotike do fizike in astrofizike, če naštejemo le nekaj primerov. Kdo bi si mislil, da lahko origami uporabimo za zlaganje šotorov in ogromnih kupol nad športnimi stadioni ali celo za pošiljanje solarnih objektov v vesolje~\cite[str.\ 3--5]{hull2020}?

Origami je umetnost prepogibanja papirja, ki se razvija že več kot tisočletje (trdnih dokazov o zlaganju papirja, kot ga poznamo danes, do pred letom 1600 po Kr.\ ni). Oblikovanje figur iz lista papirja se je do konca 20.\ stoletja hitro razširilo po vsem svetu~\cite{robinson2024}. Matematični vidik origamija je v ospredje prišel nekoliko kasneje. V 19.\ stoletju je nemški učitelj Friedrich Froebel (1782--1852) v prepogibanju papirja opazil visoko pedagoško vrednost in ga vljučil v svoj pouk osnovne geometrije v vrtcu. Indijski matematik Tandalam Sundara Row je leta 1893 izdal obsežno knjigo \emph{Geometric Exercises in Paper Folding}~\cite{row1917}, v kateri popisuje konstrukcije raznolikih geometrijskih likov in celo krivulj. Velik prelom je dosegla italijanska matematičarka Margherita P. Beloch, ki je v tridesetih letih 20.\ stoletja odkrila postopek, s katerim lahko preko prepogibanja papirja rešujemo celo kubične enačbe. Vseeno je preteklo še pol stoletja, da je origami začel zanimati tudi širšo znanost, od takrat pa se je na tem področju odprlo veliko priložnosti za raziskovanje koncepta origamija in njegovo uporabo v najrazličnejših strokah~\cite[str.\ 10]{hull2020}.

V tej nalogi nas med drugim zanima tudi uporaba origamija v pedagoške namene. Prepričana sem, da lahko praktična izkušnja prepogibanja papirja za namen reševanja problemov učence bolj motivira, saj to ni običajna oblika dela pri pouku, hkrati pa zahteva spretnost in natančnost. Poleg fine motorike lahko z origamijem krepimo predvsem raziskovalno delo učencev ter odkrivanje in uporabo geometrijskih načel in pravil v praksi. V nalogi je vključenih veliko primerov, predvsem tistih iz geometrijskega področja, vendar še zdaleč ne bomo zajeli vsega, kar bi lahko v šoli s prepogibanjem papirja počeli.

Največja motivacija za nastanek te naloge je, da ni veliko literature v slovenskem jeziku, ki opisuje matematični pogled na origami. Moč je najti nekaj člankov, seminarskih nalog ter diplomskih in magistrskih del, strokovnih knjig v slovenščini iz tega področja pa nisem našla. Ta naloga zajema predvsem uporabo origamija za namene raziskovanja geometrije ter reševanja enačb in vključuje veliko slik z orisanimi konstrukcijami. Med drugim je namenjena uporabi pri pouku matematike ali matematičnem krožku. Opisane matematične teme so namreč dovolj enostavne, da se jih večinoma da predelati v eni šolski uri. Zato iskreno upam, da bo naloga koristila še kateremu pedagogu, ki bi si želel svoj pouk matematike popestriti na nov in zanimiv način.

Najprej bomo definirali ter poiskali povezavo in razliko med evklidskimi in origami konstrukcijami. S spustom na algebraično ozadje konstrukcij bomo dokazali, katera števila (oz.\ razdalje) in kateri pravilni $n$-kotniki so konstruktibilni z origamijem in tako pokazali, da lahko z origamijem kosntruiramo več kot samo z evklidskim orodjem.

Nato bomo v roke prijeli liste papirja v obliki trikotnika, kvadrata ali pravokotnika ter spoznali nekaj osnovnih konstrukcij -- od osnovnošolskega dokazovanja lastnosti geometrijskih likov do konstrukcije enakostraničnega trikotnika in pravilnega šest- in osemkotnika; pogledali si bomo vse tri Hagove izreke, s katerimi njihov avtor raziskuje, v kakšnem razmerju lahko z določenimi prepogibi razdelimo stranice kvadrata, potem pa bomo to posplošili na iskanje postopka razdelitve stranice na poljubno število enakih delov. Na kratko bomo spoznali tudi $X$-pregibe.

V naslednjem poglavju bomo po konstrukciji kvadratnega korena poljubnega števila spoznali več različnih origami postopkov, ki nam rešijo dva starogrška problema, nerešljiva z evklidskim orodjem -- problem trisekcije kota ter podvojitve kocke, torej konstrukcije števila $\sqrt[3]{2}$.

Sledi kratko poglavje, v katerem spoznamo (ali osvežimo spomin), kako s prepogibanjem papirja točke na določeno premico ali krožnico konstruiramo tangente na vse štiri stožnice in s tem na papirju dobimo njihov obris.

Najbolj zanimivo in obsežno poglavje je reševanje enačb s prepogibanjem papirja. Tu spoznamo več metod, s katerimi lahko rešujemo splošne kvadratne in kubične enačbe ter celo nekatere enačbe četrte stopnje. V tem poglavju so zbrane metode in postopki več avtorjev, kot so Lill, Beloch, Alperin in Geretschläger, od katerih slednja dva rešitve enačb iščeta preko presečišč dualnih stožnic v projektivni ravnini.

Na koncu je predstavljen še optični Alhazenov problem, ki se ga da reševati tako na algebrski kot geometrijski način. Pogledali si bomo, kako lahko z origamijem poiščemo točko odboja na sferičnem zrcalu, v kateri se svetlobni žarek iz ene izbrane točke odbije natanko v drugo izbrano točko.

Ker je naloga spisana z namenom morebitne uporabe pri pouku srednješolske (lahko tudi osnovnošolske) matematike, je v njej zajetih veliko tem, ki bi bile razumljive tudi bralcem, ki nimajo veliko znanja univerzitetne matematike. Vseeno se na nekaterih področjih dotaknemo tudi konceptov, ki dijakom načeloma niso znani, kot so npr.\ polja, razširitve polj, Galoisova teorija, projektivna geometrija.

Zapustimo sedaj znano jezerce umetelno prepognjenih ladjic in žerjavov ter se podajmo na širne vode globokega oceana matematičnega origamija.