\section{Zlaganje stožnic}
\label{pogl:stoznice}

% kako konstruiramo tangente na stožnice in zakaj konstrukcije tako delujejo
% parabola, elipsa, hiperbola
% povezava z O5
% pa omeni tudi O6, kjer je skupna tangenta na dve paraboli
% a za krožnico se tudi da?

Iz didaktičnega vidika zelo zanimivo poglavje nam predstavlja konstrukcije tangent na stožnice s prepogibanjem papirja. \textcolor{red}{pogruntej in poblefirej še kej za uvod}

\subsection{Parabola}

V prejšnjem poglavju smo spoznali aksiom~\ref{aks:O5}, ki nam je podal tangento na parabolo z goriščem $A$ in premico vodnico $a$ (slika~\ref{fig:O5_parabola}). Tangenta je bila enolično določena s točko $B$. V splošnem te točke ne potrebujemo -- katerikoli pregib, ki točko $A$ preslika na premico $a$, je neka tangenta na parabolo.

Na sliki \textcolor{red}{REFERENCA} je narisan potek konstrukcije, ki nam poda množico pregibov -- tangent na parabolo. Za premico vodnico si zaradi enostavnosti izberimo kar en rob lista papirja ter s svinčnikom nekje označimo gorišče. List prepogibamo tako, da izbrani rob pokrije točko. Rob nato premikamo po malih korakih v obe strani in tako se nam po vedno več pregibih prikaže vedno bolj gladek obris parabole.

\textcolor{red}{SLIKA korakov (kot v Zore2022)}

