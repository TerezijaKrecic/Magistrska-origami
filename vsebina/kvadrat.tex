\section{Prepogibanje kvadrata}
\label{pogl:prepog_kvadrata}

Vzemimo v roke kvadraten list papirja in poglejmo, kaj lahko z njegovim prepogibanjem dobimo. V tem poglavju se bomo ukvarjali s preprostimi konstrukcijami, kot so konstrukcija enakostraničnega trikotnika in števila $\sqrt{r}$ za poljuben $r \in \mathcal{O}$. Pogledali si bomo vse tri Hagove izreke o razmerjih, na katere specifični pregibi kvadratnega lista papirja razdelijo njegove stranicE, nato pa to posplošili na iskanje metod za razdelitev daljice na poljubno število skladnih delov. Na koncu pa bomo končno preko več različnih konstrukcij rešili dva starogrška problema, zaradi katerih smo se sploh začeli ukvarjati s temo origamija.

\subsection{Nekaj kratkih in zanimivih konstrukcij za uvod}

\subsubsection*{Konstrukcija enakostraničnega trikotnika}

- kot 60° (smo že pri algebri)
- enakostranični trikotnik (smo že pri algebri, ma lahko še enkrat, pa več načinov)

\subsubsection*{Konstrukcija števila $\sqrt{r}$}

\textcolor{red}{Tu ne rabiš kvadratnega lista!}

Vzemimo $r \in \Q^+$ in si poglejmo naslednjo konstrukcijo (vzeto iz~\cite[str.\ 58]{hull2013}):
Imejmo točko $A (0, 1) $ in premico $y = -1$. Na ordinatni osi označimo točko $B (0, -r/4)$ in z operacijo~\ref{op:O6} skoznjo naredimo pregib, ki točko $A$ položi na premico $y = -1$. Njena zrcalna slika je $A' (t, 0) $ za nek $t \in \R$ (slika~\ref{fig:konstrukcija_korena}).

\begin{figure}[h]
    \centering
    \includegraphics[width=0.5\textwidth]{images/kvadratni_koren.png}
    \caption[Konstrukcija korena]{Konstrukcija števila $\sqrt{r}$ za poljuben $r \in \Q^{+}$.}
    \label{fig:konstrukcija_korena}
\end{figure}

Pregib po konstrukciji poteka skozi točko $B$ in razpolovišče daljice $AA'$, torej je njegov koeficient $k_B = \frac{r}{2t}$ (izpeljavo prepuščamo bralcu). Ker je pregib simetrala daljice $AA'$, njena nosilka pa ima koeficient $k_A = - \frac{2}{t}$, dobimo
\begin{align*}
    k_B &= - \frac{1}{k_A},\\
    \frac{r}{2t} &= \frac{t}{2},\\
    r &= t^2 \text{ oz. } t = \sqrt{r}.
\end{align*}
Na koncu le še prepognemo pravokotnico na abscisno os skozi točko $A'$ in tako dobimo točko $(\sqrt{r}, 0)$. Torej smo konstruirali število $\sqrt{r}$ za poljuben $r \in \Q^{+}$.

\textcolor{red}{Še kakšna konstrukcija?}

\subsection{Hagovi izreki}

S prepogibanjem kvadratnega lista papirja se je veliko ukvarjal Kazuo Haga, sicer japonski profesor biologije. V svojem delu \emph{Origamics: Mathematical Explorations Through Paper Folding}~\cite{haga2008} je tako med drugim formuliral tri izreke, ki jih poznamo pod imenom \emph{Hagovi izreki}. Pri vsakem od njih gre za konstrukcijo specifičnega pregiba, ki povzroči delitev stranic kvadrata v različnih razmerjih. Vsak izrek posebej bomo najprej formulirali, si slikovno ogledali konstrukcijo in ga dokazali, nato pa si pogledali še nekaj dodatnih lastnosti, ki sledijo iz njega.

Da si olajšamo računanje, predpostavimo, da ima kvadrat, ki predstavlja naš list papirja, stranico dolžine $1$. Njegova oglišča označimo s črkami $A, B, C$ in $D$, začenši v zgornjem desnem oglišču in sledečimi v pozitivni smeri, torej nasprotni smeri urinega kazalca.

\textcolor{red}{Konstrukcija poljubnega $a/b \in \Q$~\cite[str.\ 20--21]{lang2013}}

\textcolor{red}{Hull2013, activity 11 (str. 103--)}

\subsubsection{Prvi Hagov izrek}

\begin{izrek}[Prvi Hagov izrek]
    Zgornjo stranico $AD$ kvadrata $ABCD$ razpolovimo v točki $E$ in s pregibom nanjo položimo oglišče $C$. S tem na levi in desni stranici kvadrata dobimo tri točke, ki jih označimo z $F, G$ in $H$ (slika~\ref{fig:hagov_izrek1}). Za te točke velja:
    \begin{itemize}
        \item točka $F$ deli desno stranico v razmeru $3:5$,
        \item točka $H$ deli levo stranico v razmerju $2:1$,
        \item točka $H$ deli spodnjo stranico v razmerju $1:5$,
        \item točke $G$ deli levo stranico v razmerju $7:1$.
    \end{itemize}
\end{izrek}

\begin{figure}[h]
    \centering
    \includegraphics[width=0.4\textwidth]{images/hagovi_izreki/hagov_izrek1.png}
    \caption[Pregib iz prvega Hagovega izreka]{Konstrukcija pregiba iz prvega Hagovega izreka. Vzeto iz~\cite[str. 4]{haga2008}.}
    \label{fig:hagov_izrek1}
\end{figure}

\begin{dokaz}
    Kot kaže slika~\ref{fig:hagov_izrek1}, označimo še točki $I$ in $J$. Najprej lahko opazimo, da pregib iz izreka povzroči nastanek treh podobnih pravokotnih trikotnikov, ki so na sliki~\ref{fig:hagov_izrek1} pobarvani sivo. Za vsakega od njih lahko določimo dolžine njegovih stranic.

    Začnimo s trikotnikom $\triangle DEF$. Ker je $E$ razpolovišče stranice $AD$, je $|DE| = 1/2$. Če drugo kateto $DF$ označimo z $a$, je hipotenuza $EF$ dolga $1-a$, saj $|DF| + |EF| = |DF| + |FC| = 1$ po konstrukciji. Iz Pitagorovega izreka nato izračunamo $a = 3/8$. Torej točka $F$ res deli stranico $CD$ v razmerju $3:5$.

    Iz razmerja podobnih trikotnikov $\triangle DEF$ in $\triangle AHE$ dobimo
    $$ \frac{|AH|}{|AE|} = \frac{|DE|}{|DF|}, \; \text{ torej } \; |AH| = \frac{|AE|\cdot|DE|}{|DF|} = \frac{1/2 \cdot 1/2}{3/8} = \frac{2}{3}.$$
    Točka $H$ res deli stranico $AB$ v razmerju $2:1$. Drugače povedano -- s prvim Hagovim izrekom znamo poljubno daljico razdeliti na tri skladne dele.

    Sedaj lahko izračunamo dolžino hipotenuze $EH$ trikotnika $\triangle AHE$. Iz Pitagorovega izreka sledi $|EH| = 5/6$ (in posledično iz $|EI| = 1$ še $|HI| = 1/6$), torej točka $H$ res deli spodnjo stranico v razmerju $1:5$.

    Za izračun dolžine daljice $BG$, ki je po konstrukciji enaka dolžini katete $GI$, si spet pomagamo z razmerji podobnih trikotnikov; tokrat vzamemo trikotnika $\triangle IHG$ in $\triangle AHE$. Iz razmerja
    $$ \frac{|GI|}{|HI|} = \frac{|AE|}{|AH|} \; \text{ sledi } \; |BG| = |GI| = \frac{|AE|\cdot|HI|}{|AH|} = \frac{1/2 \cdot 1/6}{2/3} = \frac{1}{8},$$
    torej točka $G$ res deli stanico $AB$ v razmerju $7:1$.
\end{dokaz}

Za vajo lahko izračunamo še preostale dolžine daljic:
\begin{align*}
    |GH| &= |AB| - |AH| - |BG| = 1 - \frac{2}{3} - \frac{1}{8} = \frac{5}{24},\\
    |CJ| &= |BG| = \frac{1}{8},\\
    |FJ| &= |CD| - |DF| - |CJ| = 1 - \frac{3}{8} - \frac{1}{8} = \frac{1}{2},\\
    |FG| &= \sqrt{|GJ|^2 + |FJ|^2} = \sqrt{1^2 + \left(\frac{1}{2}\right)^2} = \frac{\sqrt{5}}{2}.
\end{align*}

S tem so znane vse dolžine daljic, na katere pregib iz prvega Hagovega izreka razdeli stranice enotskega kvadrata. Na sliki~\ref{fig:hagov_izrek1_st} je tako povzetek naših ugotovitev.

\begin{figure}[h]
    \centering
    \includegraphics[width=0.4\textwidth]{images/hagovi_izreki/hagov_izrek1_stevilke.png}
    \caption[Prvi Hagov izrek v številkah]{Dolžine daljic po prvem Hagovem izreku. Vzeto iz~\cite[str. 7]{haga2008}.}
    \label{fig:hagov_izrek1_st}
\end{figure}

Izrek lahko tudi posplošimo, če za točko $E$ ne vzamemo razpolovišča, temveč poljubno točko na daljici $AD$. Naj bo $x = |ED|$. Nastale točke označimo kot prej, dolžine nastalih daljic pa z $y_1$ do $y_6$, kot kaže slika~\ref{fig:hagov_izrek1_splosen}.

\begin{figure}[h]
    \centering
    \includegraphics[width=0.5\textwidth]{images/hagovi_izreki/hagov_izrek1_splosen.png}
    \caption[Prvi Hagov izrek v splošnem]{Oznake dolžin iz prvega Hagovega izreka v splošnem. Vzeto iz~\cite[str. 9]{haga2008}.}
    \label{fig:hagov_izrek1_splosen}
\end{figure}

Za vsak $i \in \{1,2,3,4,5,6\}$ poiščimo sedaj vrednost $y_i$ v odvisnosti od $x$. Kot prej najprej opazimo, da imamo zopet tri podobne pravokotne trikotnike. Iz Pitagorovega izreka za pravokotni trikotnik $\triangle EDF$ sledi
$$y_1 = (1-x^2)/2,$$
iz razmerja podobnih pravokotnih trikotnikov $\triangle EDF$ in $\triangle HAE$ pa izračunamo
$$ y_2 = \frac{x(1-x)}{y_1} = \frac{2x}{1+x} \; \text{ in } \; y_3 = \frac{(1-y_1)(1-x)}{y_1} = \frac{1+x^2}{1+x}.$$
Pregib $FG$ je po konstrukciji simetrala daljice $CE$, torej pravokotna nanjo, iz česar sledi, da sta trikotnika $\triangle CKF$ in $\triangle CDE$ podobna in kota $\angle DEC$ in $\angle KFC$ skladna. Zato sta skladna tudi trikotnika $\triangle CDE$ in $\triangle GJF$, torej $|FJ| = x$. Posledično je
$$y_4 = |CJ| = 1 - (y_1 + x) = \frac{(1-x)^2}{2} \; \text{ in } \; y_5 = 1 - y_2 - y_4 = \frac{(1-x)(1+x^2)}{2(1+x)}.$$
Na koncu še s ponovno uporabo Pitagorovega izreka izračunamo
$$ y_6 = \sqrt{|GJ|^2 + |FJ|^2} = \sqrt{1 + x^2}.$$

Splošne vrednosti dolžin $y_i$ mogoče niso najlepše, vendar pri marsikateri izbiri števila $x \in (0,1)$ dobimo lepe številke. Najbolj so zanimiva recipročna števila naravnih števil. Vemo že, da pri izbiri $x = 1/2$ lahko dobimo števila $1/3, 1/6, 1/8$, pri izbiri $x = 1/4$ in $x = 3/4$ dobimo še $2/5$ (in iz tega z razpolovitvijo $1/5$) in $1/7$. Računanje prepuščamo bralcu, se pa na tej točki lahko vprašamo, ali za vsak $n \in \N$ obstaja primeren $x$, da lahko preko neke dolžine $y_i$ ali $1-y_1$ in preko postopkov za konstrukcijo že znanih razmerij konstruiramo dolžino $1/n$. \textcolor{red}{dokaz, da se da??? al pej spusti, če se ti ne da \ldots}

\subsubsection{Drugi Hagov izrek}

\begin{izrek}[Drugi Hagov izrek]
    Zgornjo stranico $AD$ kvadrata $ABCD$ razpolovimo v točki $E$ in opravimo pregib skozi točko $E$ ter oglišče $C$. Točka $D$ se tako preslika v točko $F$ (slika~\ref{fig:hagov_izrek2}). Če stranico $EF$ podaljšamo do leve stranice kvadrata, jo presečišče $G$ razdeli v razmerju $2:1$.
\end{izrek}

\begin{figure}[h]
    \centering
    \includegraphics[width=0.4\textwidth]{images/hagovi_izreki/hagov_izrek2.png}
    \caption[Pregib iz drugega Hagovega izreka]{Konstrukcija pregiba iz drugega Hagovega izreka. Vzeto iz~\cite[str. 12]{haga2008}.}
    \label{fig:hagov_izrek2}
\end{figure}

\begin{dokaz}
    Opazimo lahko, da sta trikotnika $\triangle BCG$ in $\triangle FCG$ skladna, saj imata skladni daljšo kateto in hipotenuzo ter pravi kot nasproti hipotenuze. Označimo $x = |GB| = |GF|$. Zapišimo Pitagorov izrek za pravokotni trikotnik $\triangle AGE$:
    $$ \left(x + \frac{1}{2}\right)^2 = (1-x)^2 + \left(\frac{1}{2}\right)^2 \; \text{ in izračunamo } \; x = \frac{1}{3},$$
    torej točka $G$ res deli stranico $AB$ v razmerju $2:1$.
\end{dokaz}

S tem smo zopet dobili način razdelitve daljice na tri enake dele, a tu zanimivih razmerij še ni konec. Poglejmo si še, v kakšen razmerju nam stranice deli točka $F$ in točke, ki jih dobimo s podaljšanjem daljic $FD$ in $FC$ do leve stranice. Označimo nove točke $H, I, J, K$ in $M$, kot kaže slika~\ref{fig:hagov_izrek2_st}.

\begin{figure}[h]
    \centering
    \includegraphics[width=0.45\textwidth]{images/hagovi_izreki/hagov_izrek2_stevilke.png}
    \caption[Drugi Hagov izrek v številkah]{Dolžine daljic po drugem Hagovem izreku. Vzeto in preurejeno iz~\cite[str. 15]{haga2008}.}
    \label{fig:hagov_izrek2_st}
\end{figure}

Po konstrukciji pregiba velja $FD \perp CE$, iz česar dobimo podobne pravokotne trikotnike $\triangle CDE$, $\triangle CFE$, $\triangle DAK$, $\triangle DHE$, $\triangle FHE$, $\triangle DIF$. Prvi trije od naštetih so celo skladni, prav tako je skladen tudi sledeči par. Le trikotnik $\triangle DIF$ nima skladnega para. Iz sledečih razmerij izračunamo
\begin{align*}
    |DH| &= \frac{|DE| \cdot |CD|}{|CE|} = \frac{1}{\sqrt{5}}, \; \text{ torej } \; |DF| = 2|FH| = \frac{2}{\sqrt{5}}, \\
    |DI| &= \frac{|DF| \cdot |CD|}{|CE|} = \frac{4}{5}, \; \text{ torej } \; |AI| = \frac{1}{5}, \\
    |FI| &= \frac{|DI| \cdot |DE|}{|CD|} = \frac{2}{5} \; \text{ in} \\
    |AK| &= |DE| = \frac{1}{2}.
\end{align*}

Iz podobnih trikotnikov $\triangle BCL$ in $\triangle MCF$ sledi še
$$ |BL| = \frac{|BC| \cdot |FM|}{|CM|} = \frac{3}{4}, \; \text{ torej } \; |AL| = |AB| - |BL| = \frac{1}{4}. $$

Torej nam drugi Hagov izrek poleg konstrukcije števil $1/3, 2/3$ poda tudi diretkno konstrukcijo števil $1/5, 2/5, 3/5$ in $4/5$.

\subsubsection{Tretji Hagov izrek}

\begin{izrek}[Tretji Hagov izrek]
    Zgornjo stranico $AD$ kvadrata $ABCD$ razpolovimo v točki $E$ in opravimo pregib, ki točko $E$ položi na desno stranico in hkrati oglišče $C$ na levo stranico (slika~\ref{fig:hagov_izrek3}). Njena slika $H$ levo stranico deli v razmerju $2:1$.
\end{izrek}

\begin{figure}[h]
    \centering
    \includegraphics[width=0.4\textwidth]{images/hagovi_izreki/hagov_izrek3.png}
    \caption[Pregib iz tretjega Hagovega izreka]{Konstrukcija pregiba iz tretjega Hagovega izreka. Vzeto iz~\cite[str. 18]{haga2008}.}
    \label{fig:hagov_izrek3}
\end{figure}

\begin{dokaz}
    Označimo še točke $E, F, G,$ in $I$ ter uvedimo $x = |BH|$ in $y = |BG|$, kot kaže slika~\ref{fig:hagov_izrek3}. Zaradi prepogiba je $|GH| = |CG| = 1-y$. Iz Pitagorovega izreka za pravokotni trikotnik $\triangle BGH$ ter razmerja za podobna trikotnika $\triangle BGH$ in $\triangle AHE$ dobimo sistem enačb
    $$ x^2 + y^2 = (1-y)^2 \; \text{ in } \; \frac{1/2}{1-x} = \frac{x}{y}, $$
    iz katerih izračunamo $x = \frac{1}{3}$ in $y = \frac{4}{9}$. Torej točka $H$ res deli stranico $AB$ v razmerju $2:1$.
\end{dokaz}

Kot pri prejšnjih dveh izrekih bi lahko poračunali še preostale dolžine daljic. To za vajo prepuščamo bralcu, ki se lahko o svojih rezultatih prepriča s sliko~\ref{fig:hagov_izrek3_st}.

\begin{figure}[h]
    \centering
    \includegraphics[width=0.4\textwidth]{images/hagovi_izreki/hagov_izrek3_stevilke.png}
    \caption[Tretji Hagov izrek v številkah]{Dolžine daljic po tretjem Hagovem izreku. Vzeto iz~\cite[str. 19]{haga2008}.}
    \label{fig:hagov_izrek3_st}
\end{figure}


\textcolor{red}{Kakšen zaključek teh treh izrekov? Skupno -- vsi trije uporabljajo središče $E$ zgornje stranice $AD$. Prvi izrek nanj položi oglišče $C$, drugi izrek naredi pregib skoznjo in oglišče $C$, tretji pa nanjo položi desno stranico tako, da $C$ leži na levi stranici. Kej skupnega, bi se dalo še kakšen drug pregib blablabla}

\textcolor{red}{Lahko omeniš še srebrne pravokotnike (npr.\ A4 list papirja, stranici sta v razmerju $1 : \sqrt{2}$ in vsakič, ko daš pravokotnik po kratki stranici na pol, dobiš spet srebrn pravokotnik; pač isti princip kot pri zlatem pravokotniku za zlati rez), pa da se da tudi na njih naredit te Hagove izreke. Katere razdelitve dobiš? Na 9, 14, 16 delov itd., poglej vir. Ampak to nej gledajo v~\cite[str.\ 21--32]{haga2008}.}



\subsection{Razdelitev daljice na $n$ enakih delov}

Stranico kvadrata želimo razdeliti na $n$ enakih delov, kjer je $n \in \N$ poljuben. Za $n = 2^t, t \in \N_0$ je to čisto enostavno, saj samo prepolavljamo razdalje med pregibi, dokler ne dosežemo cilja. Če je $n$ sod, vendar ni potenca $2$, torej $n = 2^t(2m + 1)$, kjer sta $t, m \in \N$, stranico najprej razdelimo na $2^t$ delov, nato pa moramo vsakega izmed njih razdeliti na $2m + 1$ (liho število) delov. Izziv tega problema je torej v razdelitvi daljice na liho število delov. Ko bomo zo zmogli, jo bomo znali razdeliti na $n$ delov za vsak $n \in \N$.

V prejšnjem razdelku so nam Hagovi izreki podali razdelitev stranice kvadrata na tri, pet, sedem in devet delov. Vendar iščemo metodo, ki nam stranico razdeli na $n$ delov za splošen lih $n \in \N$. Spomnimo se, da smo en tak postopek že spoznali -- v dokazu izreka~\ref{izr:podpolje} smo za poljuben $a \in \R$ znali konstruirati razdaljo $1/a$, kar bi lahko uporabili za razdelitev neke daljice na $a$ enakih delov -- konstruirano razdaljo $1/a$ bi $a$-krat prenesli naprej. Načinov reševanja tega problema pa se je skozi zadnja desetletja oblikovalo še veliko več; tu si bomo pogledali še \textcolor{red}{koliko?} metode.

\subsubsection*{Metoda križajočih se diagonal}

Metoda nima uradnega prevoda niti uradnega imena, jo pa tako imenuje Robert J.\ Lang v svojem članku~\cite{lang1988}. Njena konstrukcija je prikazana na sliki~\ref{fig:kriz_diag_3}. Najprej kvadrat dvakrat prepognemo na pol -- enkrat po diagonali skozi oglišči $A$ in $C$ in drugič po vertikali. Nato prepognemo po diagonali (skozi oglišče $B$) še desni pokončen pravokotnik. Presečišče obeh diagonal označimo s točko $P$ in naredimo skoznjo prepogib, ki je pravokoten na horizontalno stranico kvadrata.

\begin{figure}[h]
    \centering
    \includegraphics[width=0.9\textwidth]{images/tretjinjenje_stranice1.png}
    \caption[Razdelitev stranice na tri dele]{Konstrukcija po metodi križajoih se diagonal za $n=3$.}
    \label{fig:kriz_diag_3}
\end{figure}

\begin{trditev}
    \label{trd:kriz_diag_3}
    Zadnji pregib iz zgornjega opisa konstrukcije razdeli horizontalno stranico kvadrata v razmerju $2:1$.
\end{trditev}

\begin{dokaz}
    Dokazujemo lahko na več načinov:
    \begin{enumerate}
        \item \textit{Analitičen pristop:} Kvadrat postavimo v evklidsko ravnino tako, da je spodnje levo oglišče kvadrata v koordinatnem izhodišču in spodnje desno v točki $(1, 0)$. Obe diagonali izrazimo z enačbama premic. Glavna diagonala ima enačbo $y = x$, diagonala pravokotnika pa $y = -2x + 2$. Točka $P$ je njuno presečišče in ima tako koordinati $(2/3, 2/3)$.
        \item \textit{Preko podobnih trikotnikov:} Z opisanimi prepogibi v tem kvadratu konstruiramo več trikotnikov. Njihova oglišča označimo tako, kot kaže slika \textcolor{red}{naredi in referiraj sliko, str. 38 v hull2013; ampak predrugači imena oglišč (glej nadaljevanje tega dokaza)}. Iz podobnosti trikotnikov $\triangle AGP$ in $\triangle ABC$ sledi, da je trikotnik $\triangle AGP$ enakokrak. Naj bo dolžina njegovih krakov $x$. Potem je $|AG| = |GP| = x$ in $|GB| = 1 - x$. Iz podobnosti trikotnikov $\triangle EFB$ in $\triangle PGB$ sledi
        \begin{align*}
            \frac{|EF|}{|FB|} &= \frac{|PG|}{|GB|}, \\
            \frac{1}{\frac{1}{2}} &= \frac{x}{1 - x}, \\
            x &= \frac{2}{3}.
        \end{align*}
    \end{enumerate}
\end{dokaz}

Po konstrukciji pregiba, ki zgornjo stranico kvadrata razdeli v razmerju $2:1$, desni pravokotnik (s stranico $DE$) po vertikali prepognemo še na pol. S tem smo stranico kvadrata razdelili na tri enake dele.

Razdelitev stranice kvadrata na štiri dele je očitna -- kvadrat v vertikalni smeri dvakrat prepognemo na pol.

Stranico razdelimo na pet delov na podoben način kot na tri. Naredimo enak pregib po glavni diagonali, nato pa zgornjo stranico razdelimo v razmerju $3:1$ (to znamo storiti). S tem smo na desni strani kvadrata dobili pokončen pravokotnik s horizontalno stranico, dolgo četrt stranice kvadrata. Naslednji pregib je, kot prej, diagonala tega pravokotnika (tista skozi oglišče $B$). Presečišče te in glavne diagonale je točka, ki je od desne stranice oddaljena za $1/5$ (dokaz je analogen tistemu za trditev~\ref{trd:kriz_diag_3}). Naredimo vertikalen pregib skozi točko $P$ in s tem zgornjo stranico kvadrata razdelimo v razmerju $4:1$. Na koncu še levi del te stranice razdelimo na štiri dele. S tem smo celotno stranico razdelili na pet skladnih delov.

Zgornji postopek lahko posplošimo na poljuben $n \in \N$. Kot smo videli v konkretnih primerih za $n = 3, 4$ in $5$, smo si pomagali z vnajprejšnjo razdelitvijo stranice na $n-1$ število enakih delov (kar smo zmogli storiti). Dokaz naslednje trditve bo tako temeljil na indukciji.

\begin{trditev}[Metoda križajočih se diagonal za splošen $n$]
    Naj bo $n \in \N, n > 2$. Kvadrat $ABCD$ s stranico dolžine $1$ prepognemo po diagonali $AC$, potem pa stranico $DC$ s točko $E$ razdelimo v razmerju $(n-2):1$. Naredimo pregib novonastalega pravokotnika skozi točki $B$ in $E$. Presečišče te in glavne diagonale je točka $P$, ki je od desne stranice kvadrata oddaljena za $1/n$.
\end{trditev}

\begin{dokaz}
    Za $n = 1$ in $n = 2$ ni kaj dokazovati -- v prvem primeru pregiba sploh ni, v drugem primeru stranico prepolovimo.

    \textit{Baza indukcije:} Vemo že, da trditev drži za $n = 3, 4, 5$.

    \textit{Indukcijska predpostavka:} Predpostavimo, da znamo stranico razdeliti na $n$ enakih delov.

    \textit{Indukcijski korak:} Dokazujemo, da znamo stranico razdeliti na $n+1$ enakih delov. Po navodilih za konstrukcijo najprej stranico $DC$ s točko $E$ razdelimo v razmerju $(n-1):1$ (kot če bi jo razdelili na $n$ skladnih delov, ampak označimo le zadnji prpogib). To po indukcijski predpostavki znamo storiti. S prepogibom skozi oglišče $B$ in točko $E$ dobimo, kot presečišči obeh diagonal, točko $P$. Potem pa podobno kot pri dokazu trditve~\ref{trd:kriz_diag_3} dokažimo, da leži točka $P$ na želeni razdalji od desne stranice kvadrata:
    \begin{enumerate}
        \item \textit{Analitičen pristop:} Naj bo oglišče $A$ koordinatno izhodišče in oglišče $B$ točka $(1, 0)$. Premica, ki je nosilka diagonale $AC$, ima tako enačbo $y = x$, nosilka diagonale $CE$ pa $y = -nx + n$. Točka $P$ je njuno presečišče in ima tako koordinate $(n/(n+1), n/(n+1))$. Torej je od desne stranice kvadrata res oddaljena za $1/(n+1)$.
        \item \textit{Preko podobnih trikotnikov:} Označimo oglišča trikotnikov, kot kaže slika \textcolor{red}{SLIKA plus REFERENCA}. Trikotnik $\triangle AGP$ je enakokrak in njegova kraka označimo z $x$. Iz razmerij dolžin stranic podobnih trikotnikov $\triangle EFB$ in $\triangle PGB$ sledi
        $$ \frac{|EF|}{|FB|} = \frac{|PG|}{|GB|} \Rightarrow \frac{1}{\frac{1}{n}} = \frac{x}{1 - x} \Rightarrow n - nx = x \Rightarrow x = \frac{n}{n+1}. $$
        Točka $P$ je od desne stranice kvadrata res oddaljena za $x - 1 = 1/(n + 1)$.
    \end{enumerate}

\end{dokaz}

\begin{posledica}
    Poljubno daljico znamo razdeliti na $n$ skladnih delov za vsak $n \in \N$.
\end{posledica}

\begin{dokaz}
    Vzemimo neko daljico poljubne dolžine. Ker znamo konstruirati pravokotnice skozi točke in prenašati razdalje, lahko konstruiramo kvadrat, katereda zgornja stranica dana daljica \textcolor{red}{(spet kakšna slikca več korakov)}. Po zgornji trditvi jo znamo razdeliti v razmerju $(n-1) : 1$ za vsak $n \in \N$. Potem moramo njen daljši del razdeliti na $n-1$ skladnih delov. To storimo na enak način kot prej -- konstruiramo manjši kvadrat s to novo stranico in ponovimo postopek. Ustavimo se, ko na nekem koraku stranico kvadrata razdelimo v ramerju $1:1$. Takrat bo zgornja stranica oz. dana daljica razdeljena na $n$ skladnih delov.
\end{dokaz}

\subsubsection*{Metoda2}

\subsubsection*{Metoda3}

\subsubsection*{Metoda4}

\textcolor{red}{Več metod (vsaj tri?), Hull2013 (str.\ 36--40).}

\textcolor{red}{Do zdaj smo imeli metode z razdelitvijo preko prepogibanja kvadratnega lista papirja. Za razdelitev preko pravokotnika pa glej~\cite[str.\ 107--134]{haga2008}.}

\subsection{$X$-pregibi}

\textcolor{red}{Glej~\cite[str.\ 33--44]{haga2008}}


\subsection{Reševanje nerešljivih starogrških problemov}
\label{podpogl:starogrskiproblemi}

Z evklidskimi konstrukcijami se je seveda pojavilo konstruktibilnih ugank -- vprašanj, ali je specifično število konstruktibilno (in na kakšnen način) ali ne. Zelo znani so trije t.\ i.\ ``starogrški' problemi, ki so matematike bremenili več kot tisočletje, začenši s časom Evklida (300 pr.\ Kr.), končno pa sta nanje dokončno odgovorila Niels Henrik Abel (1802--1829) in Evariste Galois (1811--1832) v začetku 19.\ stoletja. Gre za sledeče tri probleme:
\begin{itemize}
    \item \textbf{Podvojitev kocke} Imejmo že konstruktibilno kocko. Konstruiraj novo kocko, ki ima dvakrat večji volumen od prve (problem se poenostavi na iskanje konstrukcije števila $\sqrt[3]{2}$).
    \item \textbf{Trisekcija kota} Dan je poljuben konstruktibilen kot. Konstruiraj kot, ki prvega deli na tri skladne dele.
    \item \textbf{Kvadratura kroga} Za dan konstruktibilen krog konstruiraj kvadrat, ki ima enako ploščino kot dani krog (problem se poenostavi na konstrukcije števila $\sqrt{\pi}$).
\end{itemize}

Z znanjem, ki sta ga znanosti posredovala Abel in Galois, se da pokazati, da ti trije problemi z evklidskim orodjem niso rešljivi. V nalogi smo do sedaj že večkrat omenili, da pa obstajajo origami konstrukcije (celo več metod za isti problem!), ki nam konstruirajo kubični koren origami števila ter razdelijo kot na tri skladne dele. Vse metode, ki bodo sedaj naštete, zahtevajo uporabo Belochinega pregiba (operacije~\ref{op:O7}), kar je logično, saj so vse ostale origami operacije dovolj za vse evklidske konstrukcije. Žal pa tudi tu ostajamo nemočni glede konstrukcije števila $\sqrt{\pi}$, saj je transcedentno.

\subsubsection{Podvojitev kocke}
\label{podpogl:podvojitev_kocke}

Po legendi iz grške mitologije je bog Apolon po oraklju prebivalcem svojega rojstnega otoka Delosa sporočil, da mu morajo, če se želijo znebiti smrtonosne kuge, zgraditi nov oltar v obliki kocke, ki je enak prejšnjemu, le da mora biti dvakrat večji po prostornini. Torej je bilo potrebno konstruirati kocko s stranico, ki je za faktor $\sqrt[3][2]$ večja od stranice originalne kocke. Po drugi legendi pa naj bi Platon izjavil, da je ta problem, ki so ga prejeli na njegovi Akademiji v Atenah, poslan od bogov samih z namenom osramotiti Grke zaradi njihovega zanemarjanja in prezira do matematike (ker z evklidskim orodjem niso znali konstruirati poljubnih dolžin)~\cite[str.\ 29]{geometricconstructions}.

Ne vemo, ali so bili Grki prepričani, da se problema z neoznačenim ravnilom in šestilom ne da rešiti. Vsekako pa jim je manjkalo algebrsko znanje. Če je stranica kocke dolga $1$, je stranica podvojene kocke dolga $\sqrt[3]{2}$ in ker je obseg $\Q(\sqrt[3]{2})$ vektorski prostor razsežnosti $3$ nad obsegom $\Q$ (enačba $ x^3 - 2 = 0 $ nima racionalne rešitve), podvojitev kocke po izreku~\ref{izr:evkl_konstr} z evkliskim orodjem res ni mogoča~\cite[str. 78]{jerman1998}.

\subsubsection*{Starogrška rešitev preko presečišča dveh parabol}

Mogoče Grkom ni uspelo priti do tega premisleka, vendar so problem vseeno uspeli rešiti, čeprav po drugi poti; uporabili so še eno močno matematično orodje -- stožnice. Videla v~\cite{videla1997} dokaže izrek, ki je identičen izreku~\ref{izr:origami_konstr} (ki govori, katera števila so origami števila), le da namesto origamija uporabi stožnice. V bistvu s tem dokaže, da so origami kosntrukcije ekvivalentne konstrukcijam s stožnicami!

V istem viru Videla tudi navaja konstrukcijo s parabolami, ki za dano dolžino $a$ podajo dolžino $c$, za katero velja $c^3 = a$. Njen avtor je Menehmo (prb.\ 350 pr.\ Kr.), tutor Aleksandra Velikega. Vzel je sledeči paraboli (slika~\ref{fig:videla}):
\begin{itemize}
    \item $\mathcal{P}_1: y = x^2$ z goriščem v točki $(0, \frac{1}{4})$ in premico vodnico $y = - \frac{1}{4}$ in
    \item $\mathcal{P}_1: x = \frac{y^2}{a}$ z goriščem v točki $(\frac{a}{4}, 0)$ in premico vodnico $x = - \frac{a}{4}$.
\end{itemize}
\begin{figure}[h]
    \centering
    \includegraphics[width=0.7\textwidth]{images/starogr_problemi/cube_parabola.png}
    \caption[Menehmova konstrukcija kubičnega korena]{Menehmova konstrukcija števila $\sqrt[3]{2}$ preko parabol. Vzeto iz~\cite[str.\ 6]{videla1997}.}
    \label{fig:videla}
\end{figure}
Presečišči teh dveh parabol dobimo preko enakosti
$$y = x^2 = y^4/a^2,$$
kar nam da enačbo $y(a^2-y^3) = 0$ z rešitvama $y=0$ in $y = \sqrt[3]{a^2}$. Presečišči sta torej koordinatno izhodišče in točka $Q = (\sqrt[3]{a}, \sqrt[3]{a^2}) $. Njena abscisa je naša rešitev.
\opomba{Čeprav je konstrukcija enostavna in logična, je praktično težje izvedena, saj z roko ne znamo natančno risati parabol. \textcolor{red}{mal lepše to napiši. Pa bodi ziher da se ne da. Pač elipso se da mehanično.}}

\subsubsection*{Martinova konstrukcija}

George E.\ Martin v~\cite[str.\ 156--157]{geometricconstructions} poda preprosto konstrukcijo števila $\sqrt[3]{k}$ za poljubno origami število $k$. Tudi on pri tem uporabi dve paraboli, vendar pri postopku potrebujemo le njuni gorišči in premici vodnici. Ne bomo iskali njunih presečišč, temveč bomo z Belochinim pregibom konstruirali njuno skupno tangento, ki nam bo podala željeni rezultat.

Naj bo $k \in \mathcal{O}$ poljuben. Vzemimo paraboli z goriščema v točkah $P = (-1, 0)$ in $Q = (0, -k)$ ter premici vodnici $p: x = 1$ in $q: y = k$. Paraboli imata skupno gorišče v koordinatnem izhodišču in sta pravokotni druga na drugo, torej imata eno samo skupno tangento. Prepognimo točko $P$ na premico $p$ in točko $Q$ na premico $q$. Pregib seka $y$-os v točki $R$ (slika~\ref{fig:martin}).
\begin{figure}[h]
    \centering
    \includegraphics[width=0.6\textwidth]{images/starogr_problemi/cube_martin.png}
    \caption[Martinova konstrukcija kubičnega korena]{Martinova konstrukcija števila $\sqrt[3]{k}$.}
    \label{fig:martin}
\end{figure}
\begin{trditev}
    Točka $R$ iz zgornje konstrukcije ima koordinate $(0, \sqrt[3]{k})$.
\end{trditev}
\begin{dokaz}
    Označimo z $O$ koordinatno izhodišče in s $S$ presečišče pregiba z $x$-osjo. Točki $R$ in $S$ sta zaradi take izbire gorišč in premic vodnic ravno središči daljic z enim krajiščem v točkah $P$ in $Q$ ter drugim krajiščem v njunih slikah. Torej velja $PR \perp RS \perp SQ$. Zato so trikotniki $\triangle POR, \triangle ROS$ in $\triangle SOQ$ podobni. Iz tega ob upoštevanju $|OP| = 1$ in $|OQ| = k$ dobimo razmerje
    $$ \frac{|OR|}{|OP|} = \frac{|OS|}{|OR|} = \frac{|OQ|}{|OS|} \Longrightarrow |OR| = \frac{|OS|}{|OR|} = \frac{k}{|OS|}, $$
    iz česar sledi
    $$ |OR|^3 = |OR| \cdot \frac{|OS|}{|OR|} \cdot \frac{k}{|OS|} = k \Longrightarrow |OR| = \sqrt[3]{k}.$$
\end{dokaz}
\begin{opomba}
    V razdelku~\ref{podpogl:beloch_kvadrat_koren} bomo spoznali konstrukcijo števila $\sqrt[3]{2}$ preko Belochinega kvadrata, ki je v bistvu poseben primer Martinove konstrukcije.
\end{opomba}

\subsubsection*{Messerjeva konstrukcija}

Peter Messer je v~\cite{messer1986} podal naslednjo preprosto konstrukcijo, ki ne konstruira števila $\sqrt[3]{2}$ kot razdaljo, temveč kot razmerje: kvadrat po horizontali razdelimo na tri dele (to sedaj že znamo storiti) ter točki $p_1$ in $p_2$ s prepogibom položimo na premici $L_1$ in $L_2$, kot kaže slika~\ref{fig:messer} (levo).
\begin{figure}[h]
    \centering
    \includegraphics[width=0.68\textwidth]{images/starogr_problemi/messer1.png}
    \includegraphics[width=0.3\textwidth]{images/starogr_problemi/messer2.png}
    \caption[Messerjeva konstrukcija]{Messerjeva konstrukcija razmerja $\sqrt[3]{2}$. Vzeto iz~\cite[str.\ 67--68]{hull2013}.}
    \label{fig:messer}
\end{figure}
\begin{trditev}
    Slika točke $p_1$ deli levo stranico kvadrata v razmerju $\sqrt[3]{2}$.
\end{trditev}
\begin{dokaz}
    Vpeljimo oznake $X, Y, A, B, C, D, E$ ter $d = |BC|$, kot kaže slika~\ref{fig:messer}. Dokazati moramo $X/Y = \sqrt[3]{2}$, za lažje računanje pa brez škode privzemimo $Y=1$. Stranica kvadrata je tako dolga $X+1$, zato je $|AC| = X+1-d$ in $|AE| = (X+1)/3$.

    Opazimo podobna pravokotnika $\triangle ABC$ in $\triangle ADE$. Iz trikotnika $\triangle ABC$ s pomočjo Pitagorovega izreka izrazimo $d = (X^2+2X)/(2X+2)$, preko leve stranice pa še $|AD| = X - (X+1)/3 = (2X-1)/3$. Iz podobnosti omenjenih trikotnikov izrazimo razmerje katete in hipotenuze z enačbo $|BC|/|AC| = |AD|/|AE|$. Ko vstavimo noter vse vrednosti, odvisne od $X$, dobimo enačbo
    $$ \frac{X^2 + 2X}{X^2 + 2X + 2} = \frac{2X - 1}{X + 1},$$
    ki se nam poenostavi prav v $X^3 = 2$. Torej je $X = \sqrt[3]{2}$.
\end{dokaz}

Še ena konstrukcija $\sqrt[3]{a/b}$ je v~\cite[str.\ 366--367]{geret1995}.



\subsubsection{Trisekcija kota}
\label{podpogl:trisekcija}

Kot $90^\circ$ znamo tretjiniti z neoznačenim ravnilom in šestilom, saj znamo konstruirati kot $30^\circ$. Težava je, da ne obstaja konstrukcija, s katero na tri skladne kote razdelimo \emph{poljuben} kot. V~\cite[str.\ 77--78]{jerman1998} je dokaz o neobstoju konstrukcije za trisekcijo kota $60^\circ$. Avtor se pri tem sklicuje na izrek~\ref{izr:evkl_konstr} in opombo~\ref{op:razseznost_obsega_evkl} iz razdelka~\ref{podpogl:evkl_konstruktibilnost}. Na kratko -- iz zveze $ 1/2 = \cos 60° = \cos(3 \cdot 20°) = 4 \cos^3 20° - 3 \cos 20°, $ z zamenjavo $x = \cos 20^\circ$ dobimo enačbo $8 x^3 - 6x - 1 = 0$, ki nima racionalne rešitve. Razsežnost prostora $\Q(x)$ nad obsegom $\Q$ je tako enaka $3$ in števila $ \cos 20° $ se ne da narisati le z ravnilom in šestilom. Zato trisekcija z evklidskim orodjem v splošnem ni mogoča.

% SPODNJI DOKAZ JE PREPISAN IZ VIRA IN SAMO CITIRAN V prejšnjem odstavku

% Algebraični dokaz, da z evklidskim orodjem ne moremo tretjiniti poljubnega kota -- dokažimo za kot $60°$. (iz~\cite[str.\ 77--78]{jerman1998})

% Kot $60°$ znamo narisati. Če bi ga znali razdeliti na tri enake dele, bi potemtake znali narisati tudi kot $20°$, s tem pa (ker znamo risati pravokotnice) tudi $\cos 20°$ \textcolor{red}{slika z enotsko krožnico}. Pokažimo, da to ne gre.

% Izračunajmo minimalni polinom števila $\cos 20°$. Ker je
% $$ \frac{1}{2} = \cos 60° = \cos(3 \cdot 20°) = 4 \cos^3 20° - 3 \cos 20°, $$
% ima polinom $ p(x) = 8 x^3 - 6x - 1 $ ničlo $ \cos 20°$. Minimalni polinom števila $ \cos 20°$ deli polinom $p$. Če polinom $p$ razpade na produkt dveh polinomov s koeficienti v $\Q$, je eden od polinomov zagotovo linearen. To pa bi pomenilo, da ima polinom $p$ vsaj eno racionalno ničlo. Edini kandidatki za racionalne ničle polinoma $p$ so števila iz množice
% $$ \{\pm 1, \pm \frac{1}{2}, \pm \frac{1}{4}, \pm \frac{1}{8} \}. $$ Nobeno od teh števil ni ničla polinoma $p$, zato se $p$ ne da razcepiti na produkt dveh polinomov z racionalnimi koeficienti. Minimalni polinom števila $ \cos 20°$ je torej enaka
% $$ m(x) = \frac{1}{8} p(x) = x^3 - \frac{3}{4} x - \frac{1}{8}. $$
% Tako je razsežnost prostora $\Q(\cos 20°)$  nad obsegom $\Q$ enaka $3$ in števila $ \cos 20° $ se ne da narisati le z ravnilom in šestilom.

% Zato trisekcija kota v splošnem ni mogoča.

\subsubsection*{Starogrška rešitev preko presečišča krožnice in hiperbole}

Grki so tudi ta problem uspeli rešiti s stožnicami. Videla v~\cite[str.\ 6--7]{videla1997} opisuje Pappusovo konstrukcijo iz $3$.\ stoletja po Kr., ki je tu ne bomo dokazali. Gre za sledeč postopek: Na kraku $BA$ poljubnega kota $\angle ABC$ izberemo poljubno točko $F$ in zarišemo krožnico s središčem v točki $B$ in polmerom $BF$. Naj bo $BD$ simetrala kota $\angle ABC$. Naj bo presečišče krožnice in hiperbole z ekscentičnostjo $2$, ki ima gorišče v točki $F$ in premico vodnico $BD$, točka $E$ (slika~\ref{fig:trisection_gr}). Potem poltrak $BE$ tretjini kot $\angle ABC$.

\begin{figure}[h]
    \centering
    \includegraphics[width=0.6\textwidth]{images/starogr_problemi/trisection_grska.png}
    \caption[Pappusova trisekcija kota]{Pappusova trisekcija kota preko stožnic. Vzeto iz~\cite[str.\ 7]{videla1997}.}
    \label{fig:trisection_gr}
\end{figure}

\subsubsection*{Abejeva metoda}

Sledeča metoda ima ime po japonskemu matematiku Hisashiju Abeju, ki jo je zapisal v svojem članku leta 1980. Postopek vključuje Belochin pregib, torej se ga ne da izvesti z evklidskim orodjem, edina pomankljivost metode pa je, da deluje le za ostre kote. Postopek je sledeč:

\begin{enumerate}
    \item Na kvadratnem listu papirja konstruiramo poljuben kot $\theta$, ki ima vrh v spodnjem desnem vogalu in en krak na spodnji stranici. Nato konstruiramo še dva horizontalna in ekvidistančna pregiba na dnu papirja (slika~\ref{fig:abe_1} levo).
    \item Točko $p_1$ prepognemo na spodnji horizontalen pregib, označen $L_1$, točko $p_2$ pa na poševen krak kota, označen z $L_2$ (slika~\ref{fig:abe_1} na sredi).
    \item Preden pregib razgrnemo, podaljšamo pregib $L_1$ do konca in nov pregib označimo z $L_3$ (slika~\ref{fig:abe_1} desno).
    \item Papir razgrnemo in tokrat v spodnji levi kot podaljšamo pregib $L_3$.
\end{enumerate}

\opomba{V $3$.\ koraku opravimo pregib še preden smo razgrnili prvega. To je za nas načeloma prepovedana poteza, vendar bi se dalo $L_3$ konstruirati tudi po klasični poti z enkratnimi prepogibi -- označili bi sliko točke, ki leži hkrati na $L_1$ in levi stranici kvadrata, ter točko v pregibu iz $2$.\ koraka, ki leži na $L_1$ in skozinju naredili pregib $L_3$ -- zato zaradi lažje izvedbe brez škode dopustimo tak postopek.}

\begin{figure}[h]
    \centering
    \includegraphics[width=0.95\textwidth]{images/starogr_problemi/abe_nastavek1.png}
    \caption[Abejeva metoda ($1$.\ del)]{Trisekcija kota po Abejevi metodi. Vzeto iz~\cite[str.\ 64]{hull2013}.}
    \label{fig:abe_1}
\end{figure}

\begin{trditev}
    Pregib $L_3$ poteka skozi točko $p_1$. Kot s krakoma $L_2$ in $L_3$ ter vrhom v točki $p_1$ je velik $\theta/3$.
\end{trditev}
\begin{posledica}
    Ko spodnji rob kvadrata prepognemo na pregib $L_3$, razdelimo kot $\theta$ na tri skladne kote.
\end{posledica}

\begin{dokaz}
    Posledica logično sledi, zato dokazujemo le trditev. Označimo z $x$ točko, ki leži na presečišču pregiba $L_1$ in pregiba iz $2$.\ koraka Abejeve metode. Z $A, B$, in $C$ označimo še slike točk z leve stranice kvadrata, kot kaže slika~\ref{fig:abe_2}.
    \begin{figure}[h]
        \centering
        \includegraphics[width=0.7\textwidth]{images/starogr_problemi/abe_trisekcija.png}
        \caption[Abejeva metoda ($2$.\ del)]{Dokazovanje Abejeve metode. Vzeto in predelano iz~\cite[str.\ 65]{hull2013}.}
        \label{fig:abe_2}
    \end{figure}
    Ker je točka $C$ slika točke $p_1$ in $x$ leži na $L_1$, daljica $xC$ leži na $L_1$. Po konstrukciji daljica $xB$ leži na $L_3$, zato sta kota ob $x$, ko papir razgrnemo, skladna. Zaradi sovršnosti kotov daljica $p_1x$ leži na $L_3$, s čimer je prvi del trditve dokazan.
    
    Na razgrnjenem papirju zarišemo (ali prepognemo) še nekaj daljic (slika~\ref{fig:abe_2} desno). Ker velja $|AB| = |BC| = |CD|$ in imata pravokotna trikotnika $\triangle p_1AB$ in $\triangle p_1BC$ skupno še drugo kateto, trikotnika $\triangle p_1BC$ in $\triangle p_1CD$ pa skupno hipotenuzo, so vsi trije trikotniki skladni z enakim kotom v točki $p_1$, torej nam pregiba skozi daljici $p_1B$ (kar je ravno $L_3$) in $p_1C$ kot $\theta$ razdelijo na tri skladne kote.
\end{dokaz}

Ker ta postopek deluje le za ostre kote, si poglejmo naslednjo metodo, ki jo lahko uporabljamo tako za ostre kot tudi tope kote.

\subsubsection*{Justinova metoda}

Avtor Justin:
\begin{itemize}
    \item \cite[str.\ 34]{lang2013}
\end{itemize}

\subsubsection*{Še nekaj metod brez znanih avtorjev}

Gleason's trisection~\cite{gleason1988}, ma mogoče ne teve

\cite[str.\ 154--155]{geometricconstructions}
\cite[str.\ 158 nal. 10.14]{geometricconstructions}