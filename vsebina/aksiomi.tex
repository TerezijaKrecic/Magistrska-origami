\section{Origami aksiomi in povezava z evlikdskimi konstrukcijami}

\textbf{Najprej malo od origamija (zgodovina na splošno). Pravila (enkratni prepogibi, nič rezanja, 2D, ravni prepogibi).}

Kraj in čas izvora origamija nista jasno določena. Nekateri viri zatrjujejo, da izhaja iz Japonske, drugi ga pripisujejo Kitajski, tretji se ne strinjajo z nobeno od teh dveh možnosti. Možno je, da so umetnost zlaganja odkrili še pred izumom papirja, za katerega je l. 105 po Kr. poskrbel kitajski dvorni uradnik Cai Lun, saj se da npr. zlagati tudi robce iz blaga. Je pa papir idealen material za zlaganje. Japonska beseda \emph{origami} kot umetnost zgibanja papirja (``oru'' -- prepogibati, ``kami'' -- papir) se je na Daljnem vzhodu začela uporabljati proti koncu 19. stoletja.