\section{Origami aksiomi in povezava z evlikdskimi konstrukcijami}

Kraj in čas izvora origamija nista jasno določena. Nekateri viri zatrjujejo, da izhaja iz Japonske, drugi ga pripisujejo Kitajski, tretji se ne strinjajo z nobeno od teh dveh možnosti. Možno je, da so umetnost zlaganja odkrili še pred izumom papirja, za katerega je l. 105 po Kr. poskrbel kitajski dvorni uradnik Cai Lun, saj se da npr.\ zlagati tudi robce iz blaga. Je pa papir idealen material za zlaganje. Japonska beseda \emph{origami} kot umetnost zgibanja papirja (``oru'' -- prepogibati, ``kami'' -- papir) se je na Daljnem vzhodu začela uporabljati proti koncu 19. stoletja. Povečano zanimanje za origami v matematiki se je začelo v 2.\ pol.\ 20.\ stoletja in s seboj prineslo množično izhajanje literature o povezavi origamija z matematiko, fiziko, astronomijo, računalništvom, kemijo in še mnogimi drugimi vedami~\cite{zore2022}

V nalogi se bomo omejili le na prepogibanje v ravnini, tj.\ list papirja vzamemo za model evklidske ravnine, s prepogibanjem pa v tej ravnini tudi ostanemo. Nadalje pregibe konsktruiramo le po enega naenkrat in v ravni črti, prepovedana pa je uporaba katerega drugega orodja (npr.\ škarje in lepilo). Bralec je ob branju povabljen, da opisane konstrukcije tudi sam preizkusi na listu papirja, sicer pa se jih da brez večjih težav predstavljati tudi brez materiala.

\textbf{povezava me dprepogibi in premicami}