% naložite dodatne pakete, ki jih potrebujete
\usepackage{units}        % fizikalne enote kot \unit[12]{kg} s polovico nedeljivega presledka, glej primer v kodi
\usepackage{graphicx}     % za slike
% \usepackage{tikz}
% VEČ ZANIMIVIH PAKETOV
% \usepackage{array}      % več možnosti za tabele
% \usepackage[list=true,listformat=simple]{subcaption}  % več kot ena slika na figure, omogoči slika 1a, slika 1b
% \usepackage[all]{xy}    % diagrami
% \usepackage{doi}        % za clickable DOI entrye v bibliografiji
% \usepackage{enumerate}     % več možnosti za sezname

% Za barvanje source kode
% \usepackage{minted}
% \renewcommand\listingscaption{Program}

% Za pisanje psevdokode
% \usepackage{algpseudocode}  % za psevdokodo
% \usepackage{algorithm}
% \floatname{algorithm}{Algoritem}
% \renewcommand{\listalgorithmname}{Kazalo algoritmov}

% deklarirajte vse matematične operatorje, da jih bo LaTeX pravilno stavil
% \DeclareMathOperator{\...}{...}

% vstavite svoje definicije ...
\newcommand{\R}{\mathbb R}
\newcommand{\N}{\mathbb N}
\newcommand{\Z}{\mathbb Z}
% Lahko se zgodi, da je ukaz \C definiral že paket hyperref,
% zato dobite napako: Command \C already defined.
% V tem primeru namesto ukaza \newcommand uporabite \renewcommand
\newcommand{\C}{\mathbb C}
\newcommand{\Q}{\mathbb Q}
% Oznaki za množici evklidsko-konstruktibilnih in origami števil:
\newcommand{\OR}{\mathbb O}
\newcommand{\E}{\mathbb E}
