% \documentclass[mat2, tisk]{fmfdelo}
% \documentclass[fin2, tisk]{fmfdelo}
% \documentclass[isrm2, tisk]{fmfdelo}
\documentclass[ped, tisk]{fmfdelo}
% Če pobrišete možnost tisk, bodo povezave obarvane,
% na začetku pa ne bo praznih strani po naslovu, …

%%%%%%%%%%%%%%%%%%%%%%%%%%%%%%%%%%%%%%%%%%%%%%%%%%%%%%%%%%%%%%%%%%%%%%%%%%%%%%%
% METAPODATKI
%%%%%%%%%%%%%%%%%%%%%%%%%%%%%%%%%%%%%%%%%%%%%%%%%%%%%%%%%%%%%%%%%%%%%%%%%%%%%%%

% - vaše ime
\avtor{Terezija Krečič}

% - naslov dela v slovenščini
\naslov{Origami konstrukcije in reševanje enačb s prepogibanjem papirja}

% - naslov dela v angleščini
\title{Origami constructions and solving equations by paper-folding}

% - ime mentorja/mentorice s polnim nazivom:
%   - doc.~dr.~Ime Priimek
%   - izr.~prof.~dr.~Ime Priimek
%   - prof.~dr.~Ime Priimek
%   za druge variante uporabite ustrezne ukaze
\mentor{prof.~dr.~Aleš Vavpetič}
% \somentor{...}
% \mentorica{...}
% \somentorica{...}
% \mentorja{...}{...}
% \somentorja{...}{...}
% \mentorici{...}{...}
% \somentorici{...}{...}

% - leto magisterija
\letnica{2025}

% - povzetek v slovenščini
%   V povzetku na kratko opišite vsebinske rezultate dela. Sem ne sodi razlaga
%   organizacije dela, torej v katerem razdelku je kaj, pač pa le opis vsebine.
\povzetek{Prepogibanje papirja je v zadnjih petdesetih letih dobilo veliko vrednost v znanosti. Z ravnimi in enkratnimi prepogibi ter z njihovimi presečišči so določene premice in točke v modelu evklidske ravnine, vendar origamika, kot matematična veja raziskovanja preko prepogibanja papirja, poleg evklidske geometrije temelji tudi na področju algebre, projektivne geometrije, analize, teorije števil in še mnogo drugih. V nalogi bomo definirali konstrukcije, ki jih lahko opravljamo s prepogibanjem papirja, ter definirali množico origami števil. Pogledali si bomo, kako lahko z origamijem rešujemo probleme, izmed katerih lahko nekatere rešimo tudi z evklidskim orodjem, ostalih pa ne. Spoznali bomo Hagove izreke, konstruirali razdaljo $\sqrt[3]{2}$ in tretjinili poljubni kot, prepogibali tangente na stožnice, reševali enačbe kubične in četrte stopnje, na koncu pa spoznali znan optični Alhazenov problem.}

% - povzetek v angleščini
\abstract{An abstract of the work is written here. This includes a short description of
the content and not the structure of your work. NAPIŠI!}

% - klasifikacijske oznake, ločene z vejicami
%   Oznake, ki opisujejo področje dela, so dostopne na strani https://www.ams.org/msc/
\klasifikacija{11R04, 11R11, 11R32, 51M05, 47L50, 11D41}

% - ključne besede, ki nastopajo v delu, ločene s \sep
\kljucnebesede{prepogibanje papirja\sep origami operacije\sep Belochin pregib\sep origami konstrukcije\sep origami števila\sep Hagovi izreki\sep $X$-pregibi\sep trisekcija kota\sep podvojitev kocke\sep prepogibanje tangent na stožnice\sep Lillova metoda\sep reševanje enačb\sep dualne stožnice\sep Alhazenov problem}

% - angleški prevod ključnih besed
\keywords{Dopolni po slovensko in prevedi tukej}

% - neobvezna zahvala
\zahvala{
  Neobvezno.
  Zahvaljujem se \dots \textcolor{red}{DOPIŠI}
}

% - program dela, ki ga napiše mentor z osnovno literaturo
\programdela{
  Mentor naj napiše program dela skupaj z osnovno literaturo. \textcolor{red}{ZA DOPISAT!}
}

\osnovnaliteratura{
% Literatura mora biti tukaj posebej samostojno navedena (po pomembnosti) in ne
% le citirana. V tem razdelku literature ne oštevilčimo po svoje, ampak uporabljamo
% ukaz \vnosliterature, v katerega vpišemo citat
  \vnosliterature{hull2020}
  \vnosliterature{haga2008}
  \vnosliterature{geret1995}
  \vnosliterature{nishimura2018}
}

% - ime datoteke z viri (vključno s končnico .bib), če uporabljate BibTeX
\literatura{literatura.bib}

%%%%%%%%%%%%%%%%%%%%%%%%%%%%%%%%%%%%%%%%%%%%%%%%%%%%%%%%%%%%%%%%%%%%%%%%%%%%%%%
% DODATNE DEFINICIJE
%%%%%%%%%%%%%%%%%%%%%%%%%%%%%%%%%%%%%%%%%%%%%%%%%%%%%%%%%%%%%%%%%%%%%%%%%%%%%%%

\input{definicije.tex}

%%%%%%%%%%%%%%%%%%%%%%%%%%%%%%%%%%%%%%%%%%%%%%%%%%%%%%%%%%%%%%%%%%%%%%%%%%%%%%%
% ZAČETEK VSEBINE
%%%%%%%%%%%%%%%%%%%%%%%%%%%%%%%%%%%%%%%%%%%%%%%%%%%%%%%%%%%%%%%%%%%%%%%%%%%%%%%

\begin{document}

\section{Uvod}

Začnimo z odzivom mojih prijateljev in sorodnikov, ko so izvedeli, da bom v svoji magistrski nalogi pisala o origamiju. Velika večina jih je bila zelo presenečena, saj si sploh ni predstavljala, da se v prepogibanju papirja skriva matematika. Kar je razumljivo, saj običajno ljudje, ki se s to kraljico znanosti po srednješolskem izobraževanju prenehajo aktivneje ukvarjati, njenega vpliva na vse okoli nas ne opazijo.

In resnica je, da se v origamiju razkriva toliko matematike, da je v tej nalogi ni bilo mogoče zajeti v celoti. Ne da se niti oceniti, kolikšen delež je tu opisan, saj se origami ne dotika le -- že tako izjemno širokega -- področja geometrije, temveč tudi analize, teorije števil, abstraktne algebre, diferencialne topologije \ldots Prav tako njegova uporaba zajema široko polje znanosti in inženirstva -- od arhitekture in robotike do fizike in astrofizike, če naštejemo le nekaj primerov. Kdo bi si mislil, da lahko origami uporabimo za zlaganje šotorov in ogromnih kupol nad športnimi stadioni ali celo za pošiljanje solarnih objektov v vesolje?~\cite[str.\ 3--5]{hull2020}.

Origami je umetnost prepogibanja papirja, ki se razvija že več kot tisočletje (trdnih dokazov o zlaganju papirja, kot ga poznamo danes, do pred letom 1600 po Kr.\ ni). Oblikovanje oblik iz lista papirja se je do konca 20.\ stoletja hitro razširilo po vsem svetu~\cite{robinson2024}. Matematični vidik origamija je v ospredje prišel nekoliko kasneje. V 19.\ stoletju je nemški učitelj Friedrich Froebel (1782--1852) v prepogibanju papirja opazil visoko pedagoško vrednost, kar je uporabil pri poučevanju osnovne geometrije v vrtcu. Indijski matematik Tandalam Sundara Row je nato l.\ 1893 izdal obsežno knjigo \emph{Geometric Exercises in Paper Folding}~\cite{row1917}, v kateri popisuje konstrukcije raznolikih geometrijskih likov in celo krivulj. Velik prelom je dosegla italijanska matematičarka Margherita P. Beloch, ki je v 30-ih letih 20.\ st.\ odkrila, da lahko s prepogibanjem papirja rešujemo celo kubične enačbe. Vseeno je preteklo še pol stoletja, da je origami začel zanimati tudi širšo znanost, od takrat pa se je na tem področju odprlo veliko priložnosti za raziskovanje koncepta origamija in njegovo uporabo v najrazličnejših strokah~\cite[str.\ 10]{hull2020}.

Ravno uporaba origamija v pedagoške namene je tista, ki nas v tej nalogi še posebej zanima. Prepričana sem, da lahko praktična izkušnja prepogibanja papirja za namen reševanja problemov učence bolj motivira, saj je to neka nova oblika dela, ki je niso vajeni, hkrati pa zahteva spretnost in natančnost. Poleg fine motorike z origamijem krepimo tudi raziskovalno delo učencev ter odkrivanje in uporabo geometrijskih načel in pravil v praksi. Še zdaleč ne bomo zajeli vsega, kar bi lahko v šoli s prepogibanjem papirja počeli, vendar je kljub vsemu v nalogi vključenih veliko primerov, predvsem tistih iz geometrijskega področja.

Največja motivacija za to nalogo je, da je literature v slovenskem jeziku, ki vključuje uporabo origamija pri pouku matematike, zelo malo. Na to temo je spisanih nekaj člankov, seminarskih nalog ter diplomskih in magistrskih del, strokovnih knjig v slovenščini iz tega področja pa nisem našla. Ta naloga zajema predvsem uporabo origamija za namene raziskovanja geometrije ter reševanja enačb in vključuje veliko slik z orisanimi konstrukcijami. Me ddrugim je namenjena uporabi pri pouku matematike ali matematičnem krožku. Opisane matematične teme so namreč dovolj enostavne, da se jih večinoma da predelati v eni šolski uri. Zato iskreno upam, da bo naloga koristila še kateremu pedagogu, ki bi si želel svoj pouk matematike popestriti na nov in zanimiv način.

V geometriji preko Evklidovih postulatov ter uporabe znamenitih evklidskih orodij raziskujemo, kaj vse lahko v evklidski ravnini skonstruiramo brez uporabe drugih pravil ali orodij. V prvem poglavju si bomo pogledali podobnosti med evklidskimi in origami konstrukcijami in ugotovili, da lahko z origamijem konstruiramo več, kot lahko z neoznačenim ravnilom in šestilom. To bomo pokazali tudi s spustom na algebraično ozadje konstrukcij.

V drugem poglavju bomo v roke prijeli liste papirja različnih geometrijskih oblik in spoznali nekaj osnovnih konstrukcij -- preko osnovnošolskega dokazovanje lastnosti geometrijskih likov do konstrukcije enakostraničnega trikotnika in kvadratnega korena poljubnega števila; pogledali si bomo tudi vse tri Hagove izreke, s katerimi njihov avtor raziskuje, v kakšnem razmerju lahko razdelimo stranice kvadrata, če opravimo določene prepogibe, potem pa bomo to posplošili na iskanje postopka razdelitve stranice na poljubno število enakih delov. Zaključili bomo z (več različnimi!) origami konstrukcijami, ki nam rešijo dva starogrška problema, ki ju z evklidskim orodjem ne moremo rešiti -- problem trisekcije kota ter podvojitve kocke, torej konstrukcije števila $\sqrt[3]{2}$.

Sledi kratko poglavje o konstrukcijah pravilnih $n$-kotnikov (\textcolor{red}{tu še ne vem, a bom vključila ali ne}). Potem sledi didaktično zanimivo področje, v katerem spoznamo (ali osvežimo spomin), kako s prepogibanjem papirja točke na določeno premico ali krožnico konstruiramo tangente na vse štiri stožnice in s tem na papirju dobimo njihov obris.

Najbolj zanimivo in obsežno poglavje pa je reševanje enačb s prepogibanjem papirja. Ko v prvem poglavju spoznamo, katera števila lahko kosntruiramo z origamijem, tu znanje nadradimo in spoznamo več metod, s katerimi lahko rešujemo splošne kvadratne in kubične enačbe ter celo nekatere enačbe četrte stopnje.

\textcolor{red}{Potem je še poglavje o Alzhazen's problem in 2-fold origami, ampak še ne vem, koliko bom tega v ključila. Nekaj bi še, ker do zdaj ni neke pretežke matematike za FMF tukej noter, ampak bom še videla. Do zdaj se ukvarjamo samo z 1-fold origamijem, torej da prepogneš in takoj poravnaš papir nazaj.}

Ker je naloga spisana z namenom morebitne uporabe pri pouku srednješolske (lahko tudi osnovnošolske) matematike, je v njej zajetih veliko tem, ki bi bile razumljive tudi bralcem, ki nimajo veliko znanja univerzitetne matematike. Vseeno se na nekaterih področjih dotaknemo tudi konceptom, ki dijakom načeloma niso znani (npr.\ afina in projektivna geometrija pri metodi reševanja enačbe četrte stopnje). \textcolor{red}{(in mogoče še potem, če vključiš zadnji dve poglavji, ki res nista enostavni)}

Zapustimo sedaj znano jezerce umetelno prepognjenih ladjic in žerjavov ter se podajmo na širne vode globokega oceana matematičnega origamija.
\newpage
\section{Evklidske in origami konstrukcije}

Kraj in čas izvora origamija nista jasno določena. Nekateri viri zatrjujejo, da izhaja iz Japonske, drugi ga pripisujejo Kitajski, tretji se ne strinjajo z nobeno od teh dveh možnosti. Možno je, da so umetnost zlaganja odkrili še pred izumom papirja, za katerega je l. 105 po Kr. poskrbel kitajski dvorni uradnik Cai Lun, saj se da npr.\ zlagati tudi robce iz blaga. Je pa papir idealen material za zlaganje. Japonska beseda \emph{origami} kot umetnost zgibanja papirja (``oru'' -- prepogibati, ``kami'' -- papir) se je na Daljnem vzhodu začela uporabljati proti koncu 19. stoletja.

Povečano zanimanje za origami v matematiki se je začelo v 2.\ pol.\ 20.\ stoletja in s seboj prineslo množično izhajanje literature o povezavi origamija z matematiko, fiziko, astronomijo, računalništvom, kemijo in še mnogimi drugimi vedami~\cite{zore2022}. V angleščini je tako za matematično raziskovanje s prepogibanjem papirja nastal izraz ``\emph{origamics}'', ki bi ga lahko po zgledu poimenovanj veliko znanstvenih disciplin (\emph{mathematics} -- matematika, \emph{physisc} -- fizika itd.) prevedli kot ``origamika''~\cite{sgv2016} (uradnega izraza v slovenščini še ni).

\subsection{Evklidovi postulati in evklidske konstrukcije}

Preden si pogledamo, kaj lahko s prepogibanjem papirja konstruiramo, se spomnimo, na čem temelji evklidska geometrija. Za njenega očeta štejemo grškega matematika Evklida\footnote{O življenju tega aleksandrijskega učenjaka ne vemo nič gotovega, je pa zelo verjetno živel za časa prvega Ptolemaja (faraon v času 306--283 pr.\ Kr.)~\cite[str.\ 61]{struik1986}.}, ki je napisal zelo znano zbirko trinajstih knjig pod skupnim imenom \emph{Elementi}. V njih obravnavana snov temelji na strogo logični izpeljavi izrekov iz definicij\footnote{\emph{Definicija} je nedvoumno jasna opredelitev novega pojma.}, aksiomov\footnote{\emph{Aksiom} je temeljna resnica ali načelo, ki ne potrebuje dokazov (oz.\ dokaz sploh ne obstaja) in vedno velja.} in postulatov\footnote{\emph{Postulat} je predpostavka oz.\ zahteva. Evklid med aksiomi in postulati ni postavil jasne razlike, Aristotel pa je postulat od aksioma ločil po tem, da gre pri prvem bolj za hipotezo kot temeljno resnico, vendar se njene veljavnosti ne dokazuje, temveč privzame kot veljavno~\cite[str.\ 122]{euclidI}. V primeru petega Evklidovega postulata se bomo spomnili, da nam to, ali ga privzamemo ali ne, poda različne geometrije. Danes med pojmoma ne ločujemo~\cite[str.\ 2]{geometricconstructions}.}. Še danes večina osnovno- in srednješolske geometrije izvira prav iz prvih šestih knjig Elementov.

Prva knjiga nas še posebej zanima. V njej je Evklid najprej definiral osnovne pojme -- točka, premica, površina, ravnina, ravninski kot, pravi kot, ostri kot, topi kot, krog, središče kroga, premer, enakostranični in enakokraki trikotnik, kvadrat \ldots ter nazadnje upeljal še pojem vzporednih premic. Nato je zapisal znamenitih pet postulatov~\cite{euclidI}, iz katerih izhaja vsa evklidska geometrija:

\begin{postulat}
    \label{post:P1}
    Med dvema poljubnima točkama je mogoče narisati ravno črto.
\end{postulat}
\begin{postulat}
    \label{post:P2}
    Vsako ravno črto je mogoče na obeh koncih podaljšati.
\end{postulat}
\begin{postulat}
    \label{post:P3}
    Mogoče je narisati krožnico s poljubnim središčem in poljubnim polmerom.
\end{postulat}
\begin{postulat}
    \label{post:P4}
    Vsi pravi koti so med seboj skladni.
\end{postulat}
\begin{postulat}
    \label{post:P5}
    Če poljubni ravni črti sekamo s tretjo ravno črto (prečnico) in je vsota notranjih kotov eni strani prečnice manjša od dveh pravih kotov, potem se dani premici, če ju dovolj podaljšamo, sekata na tej strani prečnice.
    \opomba{Vemo že, da je postulat~\ref{post:P5} ekvivalenten \emph{aksiomu o vzporednicah}, ki pravi, da skozi dano točko, ki ne leži na dani premici, poteka natanko ena vzporednica k tej premici.}
\end{postulat}

\emph{Evklidske konstrukcije} so konstrukcije premic, kotov, krožnic in drugih geometrijskih figur, ki jih je mogoče konstruirati le z uporabo t.\i.\ \emph{evklidskih orodij}:

\begin{itemize}
    \item neoznačeno in neskončno dolgo ravnilo (anlg.\ \emph{straightedge})
    \item šestilo, ki ne prenaša razdalj (ko ga dvignemo od podlage, se njegova kraka zložita skupaj)
\end{itemize}

Formalno so torej edine dovoljene konstrukcije tiste iz postulatov~\ref{post:P1}--~\ref{post:P3}. Seveda privzamemo, da so konstruktibilne tudi točke.

Velika motivacija za uvedbo origami konstrukcij je ta, da lahko z njimi rešimo kar dva od treh znamenitih problemov, ki jih z evklidskim orodjem ne moremo. Več o tem sledi v poglavju~\ref{pogl:starogrskiproblemi}.

\subsection{Origami aksiomi}

V nalogi se bomo omejili le na prepogibanje v ravnini, tj.\ list papirja vzamemo za model evklidske ravnine, s prepogibanjem pa v tej ravnini tudi ostanemo. Nadalje pregibe konsktruiramo le po enega naenkrat in v ravni črti, prepovedana pa je uporaba kakršnegakoli orodja (npr.\ škarje in lepilo).  Bralec je ob branju povabljen, da opisane konstrukcije tudi sam preizkusi na listu papirja, sicer pa se jih da brez večjih težav predstavljati tudi brez materiala.

Ker so pregibi torej ravne črte, nam služijo kot modeli premic. Kaj so potem modeli točk? Ravno presečišča premic, torej presečišča pregibov.

Japonski matematik Humiaki Huzita je l.\ 1992 (\textcolor{red}{REFERENCA 26 v uni knjigi, pa v bistvu ugotovi, kdo je original avtor in kdo je kasneje koliko aksiomov rediscoveral!}) zapisal seznam pravil, s katerimi lahko opišemo vse operacije, ki jih lahko naredimo s prepogibanjem papirja. \textcolor{red}{(najprej 6, pol 7, kako je to blo?????)}
. Ta pravila so znana pod imenom \textcolor{red}{\ldots} aksiomi, vendar je izbira izraza ``aksiom'' mogoče manj primerna, saj za nekatere od njih opisana konstrukcija ob neprimerno izbranih točkah in premicah sploh ne obstaja. Kljub temu bomo zaradi razširjenosti rabe to ime ohranili~\cite[str.\ 7]{zore2022}. Sedaj pa si jih podrobneje poglejmo:

\begin{aksiom}[O1]
    \label{aks:O1}
    Za poljubni točki $A$ in $B$ obstaja natanko en pregib $p$, ki gre skoznju.
\end{aksiom}
\begin{aksiom}[O2]
    \label{aks:O2}
    Za poljubni točki $A$ in $B$ obstaja natanko en pregib $p$, da se točki pokrijeta.
\end{aksiom}
\begin{aksiom}[O3]
    \label{aks:O3}
    Za poljubni premici $a$ in $b$ obstaja pregib, ki ju položi eno na drugo.
    \opomba{Če sta premici vzporedni, je tak pregib en sam, sicer pa sta pregiba dva.}
\end{aksiom}
\begin{aksiom}[O4]
    \label{aks:O4}
    Za poljubno točko $A$ in premico $a$ obstaja natanko en pregib skozi točko $A$, ki je pravokoten na premico $a$.
\end{aksiom}
\begin{aksiom}[o5]
    \label{aks:O5}
    Za primerno izbrani točki $A$ in $B$ ter premico $a$ obstaja pregib $p$ skozi točko $B$, ki točko $A$ postavi na premico $a$.
\end{aksiom}
\begin{aksiom}[06]
    \label{aks:O6}
    Za primerno izbrani točki $A$ in $B$ ter premici $a$ in $b$ obstaja pregib $p$, ki točko $A$ postavi na premico $a$ ter točko $B$ na premico $b$.
\end{aksiom}
\begin{aksiom}[O7]
    \label{aks:O7}

\end{aksiom}

Hitro lahko vidimo, da je aksiom O1 ekvivalenten postulatu~\ref{post:P1}. \textcolor{red}{popravi reference za aksiome ... PLUS Še ostala povezava s postulati!!!!}. Aksiom O2 nam poda konstrukcijo simetrale daljice $AB$, aksiom O3 pa simetrali kota, ki ga določata premici in njuno presečišče (v primeru vzporednosti dobimo premico, ki je enako oddaljena od obeh premic). Aksiom O4 nam podaja konstrukcijo pravokotnice na premico skozi dano točko. \textcolor{red}{tukej še manjka povezava prou z EVKLIDSKIMI konstrukcijami}.

\newpage
\section{Prepogibanje kvadrata}
\label{pogl:prepog_kvadrata}

Spomnimo se, da smo pri dokazovanju trditve~\ref{izr:origami_konstruktibilnost} izpustili dokaz, da lahko vsako origami-konstruktibilno število delimo s poljubnim naravnim številom. Ob predpostavki, da to velja, smo dokazali, da so origami-konstruktibilna števila zaprta za deljenje. Cilj tega poglavja je tako pokazati, kako neko dolžino s prepogibanjem razdeliti na $n$ delov za poljubni $n \in \N$. Poleg tega pa si bomo pogledali še konstrukcijo raznovrstnih razmerij, ki jih lahko poda neka točka na poljubni dolžini.

Tako kot z evklidskim orodjem tudi s prepogibanjem papirja težko ostanemo na enem pregibu in si pri konstrukciji točk pomagamo s skokom v dvodimenzionalni svet. Zato vzemimo v roke kvadraten list papirja in raziskujmo, kako ga lahko prepogibamo in v kakšnih razmerjih lahko pregibi ali zrcaljene točke delijo njegove stranice.

\subsection{Razdelitev dolžine na $n$ enakih delov}

\textcolor{red}{Več metod (vsaj tri?), na koncu daš Hagovo metodo in iz tega začneš z novim podpoglavjem -- Hagovi izreki. Matode pred tem: Hull2013 (str.\ 36--40)}


\subsection{Hagovi izreki}

\subsubsection{Prvi Hagov izrek}

\textcolor{red}{PLUS poslošitev.}

\subsubsection{Drugi Hagov izrek}

\textcolor{red}{PLUS poslošitev.}

\subsubsection{Tretji Hagov izrek}

\textcolor{red}{PLUS poslošitev.}





\subsection{Reševanje nerešljivih starogrških problemov}
\label{podpogl:starogrskiproblemi}

% - trisekcija kota (Abe, Justin)
% - podvojitev kocke (Messer, Beloch)

% Naštej vse tri. Onega od $\pi$ se ne da rešiti?
% reševanje dveh starogrških problemov, ki ju z evklidskimi orodji -- dokazano -- ne znamo rešiti; to sta \emph{podvojitev kocke} (oz. konstrukcija $\sqrt[3]{2}$) in \emph{trisekcija kota}. Izkaže se, da se da vsakega od njiju rešiti celo na več kot en način! (to je omenjeno v 2.3, tko da trisekcija mora imet več načinov, čene popravi)


% Geometric Constructions str. 29 -- legenda od Apolonu, ki je zahteval 2x večji oltar, da prežene kugo

\subsubsection{Trisekcija kota}

% SPODNJI DOKAZ JE PREPISAN IZ VIRA IN SAMO CITIRAN V POGLAVJU 2.3

% Algebraični dokaz, da z evklidskim orodjem ne moremo tretjiniti poljubnega kota -- dokažimo za kot $60°$. (iz~\cite[str.\ 77--78]{jerman1998})

% Kot $60°$ znamo narisati. Če bi ga znali razdeliti na tri enake dele, bi potemtake znali narisati tudi kot $20°$, s tem pa (ker znamo risati pravokotnice) tudi $\cos 20°$ \textcolor{red}{slika z enotsko krožnico}. Pokažimo, da to ne gre.

% Izračunajmo minimalni polinom števila $\cos 20°$. Ker je
% $$ \frac{1}{2} = \cos 60° = \cos(3 \cdot 20°) = 4 \cos^3 20° - 3 \cos 20°, $$
% ima polinom $ p(x) = 8 x^3 - 6x - 1 $ ničlo $ \cos 20°$. Minimalni polinom števila $ \cos 20°$ deli polinom $p$. Če polinom $p$ razpade na produkt dveh polinomov s koeficienti v $\Q$, je eden od polinomov zagotovo linearen. To pa bi pomenilo, da ima polinom $p$ vsaj eno racionalno ničlo. Edini kandidatki za racionalne ničle polinoma $p$ so števila iz množice
% $$ \{\pm 1, \pm \frac{1}{2}, \pm \frac{1}{4}, \pm \frac{1}{8} \}. $$4Nobeno od teh števil ni ničla polinoma $p$, zato se $p$ ne da razcepiti na produkt dveh polinomov z racionalnimi koeficienti. Minimalni polinom števila $ \cos 20°$ je torej enaka
% $$ m(x) = \frac{1}{8} p(x) = x^3 - \frac{3}{4} x - \frac{1}{8}. $$
% Tako je razsežnost prostora $\Q(\cos 20°)$  nad obsegom $\Q$ enaka $3$ in števila $ \cos 20° $ se ne da narisati le z ravnilom in šestilom.

% Zato trisekcija kota v splošnem ni mogoča.

Neka origami konstrukcija je v~\cite[str.\ 155]{geometricconstructions}, pa tudi (avtor Abe) v nalogi 10.14~\cite[str.\ 158 spodaj]{geometricconstructions}. .

\subsubsection{Podvojitev kocke}

V prostoru imamo kocko. Ali se da samo z ravnilom in šestilom narisati stranico kocke, ki ima dvakrat večjo prostornino kot dana kocka?

Če je stranica kocke dolga 1, je stranica podvojene kocke dolga $\sqrt[3]{2}$. Ker je obseg $\Q(\sqrt[3]{2})$ vektorski prostor razsežnosti $3$ nad obsegom $\Q$ (enačba $ x^3 - 2 = 0 $ nima racionalne rešitve), podvojitev kocke ni mogoča~\cite[str. 78]{jerman1998}.

Ena konstrukcija je v~\cite[str.\ 156]{geometricconstructions} (za poljuben $k$!).
\newpage
\section{Reševanje nerešljivih starogrških problemov}
\label{pogl:starogrskiproblemi}

Z evklidskimi konstrukcijami se je seveda pojavilo konstruktibilnih ugank -- vprašanj, ali sta specifična razdalja oz.\ kot konstruktibilna (in na kakšnen način) ali ne. Stremeli so k iskanju konstrukcij z evklidskim orodjem in če so kakšen problem uspeli rešiti le z neoznačenim ravnilom in šestilom, so te konstrukcije obravnavali kot ``boljše rešitve''~\cite[str. 36]{royster2002}. Grki pa seveda niso delali le s tem orodjem, ampak so se na primer poslužili tudi označenega ravnila\footnote{Martin v trditvi $10.4$ in preko poglavja $9$ v~\cite{geometricconstructions} dokaže, da so origami števila natanko tista množica števil, ki se jih da konstruirati z ravnilom, ki ima na robu dve oznaki.}, s katerim so lahko rešili probleme, ki jih sicer niso mogli. Zelo znani so trije t.\ i.\ ``starogrški' problemi, ki so matematike bremenili več kot tisočletje, začenši s časom Evklida (300 pr.\ Kr.), končno pa sta nanje dokončno odgovorila Niels Henrik Abel (1802--1829) in Evariste Galois (1811--1832) v začetku 19.\ stoletja. Gre za sledeče tri probleme:
\begin{itemize}
    \item \textbf{Podvojitev kocke} Imejmo že konstruktibilno kocko. Konstruiraj novo kocko, ki ima dvakrat večji volumen od prve (problem se poenostavi na iskanje konstrukcije števila $\sqrt[3]{2}$).
    \item \textbf{Trisekcija kota} Dan je poljuben konstruktibilen kot. Konstruiraj kot, ki prvega deli na tri skladne dele.
    \item \textbf{Kvadratura kroga} Za dan konstruktibilen krog konstruiraj kvadrat, ki ima enako ploščino kot dani krog (problem se poenostavi na konstrukcije števila $\sqrt{\pi}$).
\end{itemize}

Z znanjem, ki sta ga znanosti posredovala Abel in Galois, se da pokazati, da ti trije problemi z evklidskim orodjem niso rešljivi. V knjigi o starogrški matematiki od Talesa do Evklida~\cite[str.\ 218--270]{heath1921} je zbrano veliko zamisli in konstrukcij grških matematikov, ki so se ukvarjali s temi tremi problemi in evklidsko orodje ne zadostuje za nobeno od najdenih rešitev.

V nalogi smo do sedaj že večkrat omenili, da pa obstajajo origami konstrukcije (celo več metod za isti problem!), ki nam konstruirajo kubični koren origami števila ter razdelijo kot na tri skladne dele. Vse metode, ki bodo sedaj naštete, zahtevajo uporabo Belochinega pregiba (operacije~\ref{op:O7}), kar je logično, saj so vse ostale origami operacije dovolj za vse evklidske konstrukcije. Žal pa tudi tu ostajamo nemočni glede konstrukcije števila $\sqrt{\pi}$, saj je transcedentno.

\subsection*{Konstrukcija števila $\sqrt{r}$}

Preden si pogledamo konstrukcijo kubičnega korena, vzemimo origami število $r \in $ in kosntruirajmo njegov kvadratni koren (postopek je vzet iz~\cite[str.\ 58]{hull2013}).

Imejmo točko $A (0, 1) $ in premico $y = -1$. Na ordinatni osi označimo točko $B (0, -r/4)$ in z operacijo~\ref{op:O6} skoznjo naredimo pregib, ki točko $A$ položi na premico $y = -1$. Njena zrcalna slika je $A' (t, 0) $ za nek $t \in \R$ (slika~\ref{fig:konstrukcija_korena}).

\begin{figure}[h]
    \centering
    \includegraphics[width=0.5\textwidth]{images/kvadratni_koren.png}
    \caption[Konstrukcija korena]{Konstrukcija števila $\sqrt{r}$ za poljuben $r \in \Q^{+}$.}
    \label{fig:konstrukcija_korena}
\end{figure}

Pregib po konstrukciji poteka skozi točko $B$ in razpolovišče daljice $AA'$, torej je njegov koeficient $k_B = \frac{r}{2t}$ (izpeljavo prepuščamo bralcu). Ker je pregib simetrala daljice $AA'$, njena nosilka pa ima koeficient $k_A = - \frac{2}{t}$, dobimo
\begin{align*}
    k_B &= - \frac{1}{k_A},\\
    \frac{r}{2t} &= \frac{t}{2},\\
    r &= t^2 \text{ oz. } t = \sqrt{r}.
\end{align*}
Na koncu le še prepognemo pravokotnico na abscisno os skozi točko $A'$ in tako dobimo točko $(\sqrt{r}, 0)$. Torej smo konstruirali število $\sqrt{r}$ za poljuben $r \in $.

\subsection{Podvojitev kocke}
\label{podpogl:podvojitev_kocke}

Po legendi iz grške mitologije je bog Apolon po oraklju prebivalcem svojega rojstnega otoka Delosa sporočil, da mu morajo, če se želijo znebiti smrtonosne kuge, zgraditi nov oltar v obliki kocke, ki je enak prejšnjemu, le da mora biti dvakrat večji po prostornini. Torej je bilo potrebno konstruirati kocko s stranico, ki je za faktor $\sqrt[3][2]$ večja od stranice originalne kocke. Po drugi legendi pa naj bi Platon izjavil, da je ta problem, ki so ga prejeli na njegovi Akademiji v Atenah, poslan od bogov samih z namenom osramotiti Grke zaradi njihovega zanemarjanja in prezira do matematike (ker z evklidskim orodjem niso znali konstruirati poljubnih dolžin)~\cite[str.\ 29]{geometricconstructions}.

Ne vemo, ali so bili Grki prepričani, da se problema z neoznačenim ravnilom in šestilom ne da rešiti. Vsekako pa jim je manjkalo algebrsko znanje iz razdelka~\ref{podpogl:evkl_konstruktibilnost}.

\subsubsection*{Starogrška rešitev preko presečišča dveh parabol}

Mogoče Grkom ni uspelo priti do tega premisleka, vendar so problem vseeno uspeli rešiti, čeprav po drugi poti; uporabili so še eno močno matematično orodje -- stožnice. Videla v~\cite{videla1997} dokaže izrek, ki je identičen izreku~\ref{izr:orig_razp_polje} (ki govori, katera števila so origami števila), le da namesto origamija uporabi stožnice. V bistvu s tem dokaže, da so origami kosntrukcije ekvivalentne konstrukcijam s stožnicami!

V istem viru Videla tudi navaja konstrukcijo s parabolami, ki za dano dolžino $a$ podajo dolžino $c$, za katero velja $c^3 = a$. Njen avtor je Menehmo (prb.\ 350 pr.\ Kr.), tutor Aleksandra Velikega. Vzel je sledeči paraboli (slika~\ref{fig:videla}):
\begin{itemize}
    \item $\mathcal{P}_1: y = x^2$ z goriščem v točki $(0, \frac{1}{4})$ in premico vodnico $y = - \frac{1}{4}$ in
    \item $\mathcal{P}_1: x = \frac{y^2}{a}$ z goriščem v točki $(\frac{a}{4}, 0)$ in premico vodnico $x = - \frac{a}{4}$.
\end{itemize}
\begin{figure}[h]
    \centering
    \includegraphics[width=0.7\textwidth]{images/starogr_problemi/cube_parabola.png}
    \caption[Menehmova konstrukcija kubičnega korena]{Menehmova konstrukcija števila $\sqrt[3]{2}$ preko parabol. Vzeto iz~\cite[str.\ 6]{videla1997}.}
    \label{fig:videla}
\end{figure}
Presečišči teh dveh parabol dobimo preko enakosti
$$y = x^2 = y^4/a^2,$$
kar nam da enačbo $y(a^2-y^3) = 0$ z rešitvama $y=0$ in $y = \sqrt[3]{a^2}$. Presečišči sta torej koordinatno izhodišče in točka $Q = (\sqrt[3]{a}, \sqrt[3]{a^2}) $. Njena abscisa je naša rešitev.
\opomba{Čeprav je konstrukcija enostavna in logična, je izvedljiva le v teoriji, saj z roko ne znamo natančno risati parabol.}

\subsubsection*{Martinova konstrukcija}

George E.\ Martin v~\cite[str.\ 156--157]{geometricconstructions} poda preprosto konstrukcijo števila $\sqrt[3]{k}$ za poljubno origami število $k$. Tudi on pri tem uporabi dve paraboli, vendar pri postopku potrebujemo le njuni gorišči in premici vodnici. Ne bomo iskali njunih presečišč, temveč bomo z Belochinim pregibom konstruirali njuno skupno tangento, ki nam bo podala željeni rezultat.

Naj bo $k \in $ poljuben. Vzemimo paraboli z goriščema v točkah $P = (-1, 0)$ in $Q = (0, -k)$ ter premici vodnici $p: x = 1$ in $q: y = k$. Paraboli imata skupno gorišče v koordinatnem izhodišču in sta pravokotni druga na drugo, torej imata eno samo skupno tangento. Prepognimo točko $P$ na premico $p$ in točko $Q$ na premico $q$. Pregib seka $y$-os v točki $R$ (slika~\ref{fig:martin}).
\begin{figure}[h]
    \centering
    \includegraphics[width=0.6\textwidth]{images/starogr_problemi/cube_martin.png}
    \caption[Martinova konstrukcija kubičnega korena]{Martinova konstrukcija števila $\sqrt[3]{k}$.}
    \label{fig:martin}
\end{figure}
\begin{trditev}
    Točka $R$ iz zgornje konstrukcije ima koordinate $(0, \sqrt[3]{k})$.
\end{trditev}
\begin{dokaz}
    Označimo z $O$ koordinatno izhodišče in s $S$ presečišče pregiba z $x$-osjo. Točki $R$ in $S$ sta zaradi take izbire gorišč in premic vodnic ravno središči daljic z enim krajiščem v točkah $P$ in $Q$ ter drugim krajiščem v njunih slikah. Torej velja $PR \perp RS \perp SQ$. Zato so trikotniki $\triangle POR, \triangle ROS$ in $\triangle SOQ$ podobni. Iz tega ob upoštevanju $|OP| = 1$ in $|OQ| = k$ dobimo razmerje
    $$ \frac{|OR|}{|OP|} = \frac{|OS|}{|OR|} = \frac{|OQ|}{|OS|} \Longrightarrow |OR| = \frac{|OS|}{|OR|} = \frac{k}{|OS|}, $$
    iz česar sledi
    $$ |OR|^3 = |OR| \cdot \frac{|OS|}{|OR|} \cdot \frac{k}{|OS|} = k \Longrightarrow |OR| = \sqrt[3]{k}.$$
\end{dokaz}
\begin{opomba}
    V razdelku~\ref{podpogl:beloch_kvadrat_koren} bomo spoznali konstrukcijo števila $\sqrt[3]{2}$ preko Belochinega kvadrata, ki je v bistvu poseben primer Martinove konstrukcije.
\end{opomba}

\subsubsection*{Messerjeva konstrukcija}

Peter Messer v~\cite{messer1986} navaja avtorski postopek, ki sicer ne konstruira števila $\sqrt[3]{2}$ kot razdaljo, temveč kot razmerje: kvadraten list papirja po horizontali razdelimo na tri dele (to sedaj že znamo storiti) ter točki $p_1$ in $p_2$ s prepogibom položimo na premici $L_1$ in $L_2$, kot kaže slika~\ref{fig:messer} (levo).
\begin{figure}[h]
    \centering
    \includegraphics[width=0.68\textwidth]{images/starogr_problemi/messer1.png}
    \includegraphics[width=0.3\textwidth]{images/starogr_problemi/messer2.png}
    \caption[Messerjeva konstrukcija]{Messerjeva konstrukcija razmerja $\sqrt[3]{2}$. Vzeto iz~\cite[str.\ 67--68]{hull2013}.}
    \label{fig:messer}
\end{figure}
\begin{trditev}
    Slika točke $p_1$ deli levo stranico kvadrata v razmerju $\sqrt[3]{2}$.
\end{trditev}
\begin{dokaz}
    Vpeljimo oznake $X, Y, A, B, C, D, E$ ter $d = |BC|$, kot kaže slika~\ref{fig:messer}. Dokazati moramo $X/Y = \sqrt[3]{2}$, za lažje računanje pa brez škode privzemimo $Y=1$. Stranica kvadrata je tako dolga $X+1$, zato je $|AC| = X+1-d$ in $|AE| = (X+1)/3$.

    Opazimo podobna pravokotnika $\triangle ABC$ in $\triangle ADE$. Iz trikotnika $\triangle ABC$ s pomočjo Pitagorovega izreka izrazimo $d = (X^2+2X)/(2X+2)$, preko leve stranice pa še $|AD| = X - (X+1)/3 = (2X-1)/3$. Iz podobnosti omenjenih trikotnikov izrazimo razmerje katete in hipotenuze z enačbo $|BC|/|AC| = |AD|/|AE|$. Ko vstavimo noter vse vrednosti, odvisne od $X$, dobimo enačbo
    $$ \frac{X^2 + 2X}{X^2 + 2X + 2} = \frac{2X - 1}{X + 1},$$
    ki se nam poenostavi prav v $X^3 = 2$. Torej je $X = \sqrt[3]{2}$.
\end{dokaz}
\opomba{Lahko bi rekli, da poleg razmerja v primeru $Y=1$ Messer konstruira razdaljo $\sqrt[3]{2}$, vendar je razdalja $Y$ odvisna od stranice kvadrata. V tem primeru bi morali vzeti kvadraten list papira s stranico $1 + \sqrt[3]{2}$, za kar bi pač potrebovali že konstrukcijo kubičnega korena števila $2$. Da bi pri splošnem kvadratnem listu papirja dobili to dolžino, moramo razdalji $X$ in $Y$ z origamijem še deliti, to pa že znamo.}

\subsection{Trisekcija kota}
\label{podpogl:trisekcija}

Kot $90^\circ$ znamo tretjiniti z neoznačenim ravnilom in šestilom, saj znamo konstruirati kot $30^\circ$. Vem pa že, da ne obstaja konstrukcija, s katero na tri skladne kote razdelimo \emph{poljuben} kot.

% SPODNJI DOKAZ JE PREPISAN IZ VIRA IN SAMO CITIRAN V prejšnjem odstavku

% Algebraični dokaz, da z evklidskim orodjem ne moremo tretjiniti poljubnega kota -- dokažimo za kot $60°$. (iz~\cite[str.\ 77--78]{jerman1998})

% Kot $60°$ znamo narisati. Če bi ga znali razdeliti na tri enake dele, bi potemtake znali narisati tudi kot $20°$, s tem pa (ker znamo risati pravokotnice) tudi $\cos 20°$ \textcolor{red}{slika z enotsko krožnico}. Pokažimo, da to ne gre.

% Izračunajmo minimalni polinom števila $\cos 20°$. Ker je
% $$ \frac{1}{2} = \cos 60° = \cos(3 \cdot 20°) = 4 \cos^3 20° - 3 \cos 20°, $$
% ima polinom $ p(x) = 8 x^3 - 6x - 1 $ ničlo $ \cos 20°$. Minimalni polinom števila $ \cos 20°$ deli polinom $p$. Če polinom $p$ razpade na produkt dveh polinomov s koeficienti v $\Q$, je eden od polinomov zagotovo linearen. To pa bi pomenilo, da ima polinom $p$ vsaj eno racionalno ničlo. Edini kandidatki za racionalne ničle polinoma $p$ so števila iz množice
% $$ \{\pm 1, \pm \frac{1}{2}, \pm \frac{1}{4}, \pm \frac{1}{8} \}. $$ Nobeno od teh števil ni ničla polinoma $p$, zato se $p$ ne da razcepiti na produkt dveh polinomov z racionalnimi koeficienti. Minimalni polinom števila $ \cos 20°$ je torej enaka
% $$ m(x) = \frac{1}{8} p(x) = x^3 - \frac{3}{4} x - \frac{1}{8}. $$
% Tako je razsežnost prostora $\Q(\cos 20°)$  nad obsegom $\Q$ enaka $3$ in števila $ \cos 20° $ se ne da narisati le z ravnilom in šestilom.

% Zato trisekcija kota v splošnem ni mogoča.

\subsubsection*{Starogrška rešitev preko presečišča krožnice in hiperbole}

Grki so tudi ta problem uspeli rešiti s stožnicami. Videla v~\cite[str.\ 6--7]{videla1997} opisuje Pappusovo konstrukcijo iz $3$.\ stoletja po Kr., ki je tu ne bomo dokazali. Gre za sledeč postopek: Na kraku $BA$ poljubnega kota $\angle ABC$ izberemo poljubno točko $F$ in zarišemo krožnico s središčem v točki $B$ in polmerom $BF$. Naj bo $BD$ simetrala kota $\angle ABC$. Naj bo presečišče krožnice in hiperbole z ekscentičnostjo $2$, ki ima gorišče v točki $F$ in premico vodnico $BD$, točka $E$ (slika~\ref{fig:trisection_gr}). Potem poltrak $BE$ tretjini kot $\angle ABC$.

\begin{figure}[h]
    \centering
    \includegraphics[width=0.5\textwidth]{images/starogr_problemi/trisection_grska.png}
    \caption[Pappusova trisekcija kota]{Pappusova trisekcija kota preko stožnic. Vzeto iz~\cite[str.\ 7]{videla1997}.}
    \label{fig:trisection_gr}
\end{figure}

\subsubsection*{Abejeva metoda}

Sledeča metoda ima ime po japonskemu matematiku Hisashiju Abeju, ki jo je odkril v $80$-ih letih prejšnjega stoletja. Postopek vključuje Belochin pregib, torej se ga ne da izvesti z evklidskim orodjem, edina pomankljivost metode pa je, da deluje le za ostre kote. Postopek je sledeč:

\begin{enumerate}
    \item Na kvadratnem listu papirja konstruiramo poljuben kot $\theta$, ki ima vrh v spodnjem desnem vogalu in en krak na spodnji stranici. Nato konstruiramo še dva horizontalna in ekvidistančna pregiba na dnu papirja (slika~\ref{fig:abe_1} levo).
    \item Točko $p_1$ prepognemo na spodnji horizontalen pregib, označen $L_1$, točko $p_2$ pa na poševen krak kota, označen z $L_2$ (slika~\ref{fig:abe_1} na sredi).
    \item Preden pregib razgrnemo, podaljšamo pregib $L_1$ do konca in nov pregib označimo z $L_3$ (slika~\ref{fig:abe_1} desno).
    \item Papir razgrnemo in tokrat v spodnji levi kot podaljšamo pregib $L_3$.
\end{enumerate}
\begin{figure}[h]
    \centering
    \includegraphics[width=0.95\textwidth]{images/starogr_problemi/abe_nastavek1.png}
    \caption[Abejeva metoda ($1$.\ del)]{Trisekcija kota po Abejevi metodi. Vzeto iz~\cite[str.\ 64]{hull2013}.}
    \label{fig:abe_1}
\end{figure}

\opomba{V $3$.\ koraku opravimo pregib še preden smo razgrnili prvega. To je za nas načeloma prepovedana poteza, vendar bi se dalo $L_3$ konstruirati tudi po klasični poti z enkratnimi prepogibi -- označili bi sliko točke, ki leži hkrati na $L_1$ in levi stranici kvadrata, ter točko v pregibu iz $2$.\ koraka, ki leži na $L_1$ in skozinju naredili pregib $L_3$ -- zato zaradi lažje izvedbe brez škode dopustimo tak postopek.}

\begin{trditev}
    Pregib $L_3$ poteka skozi točko $p_1$. Kot s krakoma $L_2$ in $L_3$ ter vrhom v točki $p_1$ je velik $\theta/3$.
\end{trditev}
\begin{posledica}
    Ko spodnji rob kvadrata prepognemo na pregib $L_3$, razdelimo kot $\theta$ na tri skladne kote.
\end{posledica}

\begin{dokaz}
    Posledica logično sledi, zato dokazujemo le trditev. Označimo z $x$ točko, ki leži na presečišču pregiba $L_1$ in pregiba iz $2$.\ koraka Abejeve metode. Z $A, B$, in $C$ označimo še slike točk z leve stranice kvadrata, kot kaže slika~\ref{fig:abe_2}.
    \begin{figure}[h]
        \centering
        \includegraphics[width=0.7\textwidth]{images/starogr_problemi/abe_trisekcija.png}
        \caption[Abejeva metoda ($2$.\ del)]{Dokazovanje Abejeve metode. Vzeto in predelano iz~\cite[str.\ 65]{hull2013}.}
        \label{fig:abe_2}
    \end{figure}
    Ker je točka $C$ slika točke $p_1$ in $x$ leži na $L_1$, daljica $xC$ leži na $L_1$. Po konstrukciji daljica $xB$ leži na $L_3$, zato sta kota ob $x$, ko papir razgrnemo, skladna. Zaradi sovršnosti kotov daljica $p_1x$ leži na $L_3$, s čimer je prvi del trditve dokazan.
    
    Na razgrnjenem papirju zarišemo (ali prepognemo) še nekaj daljic (slika~\ref{fig:abe_2} desno). Ker velja $|AB| = |BC| = |CD|$ in imata pravokotna trikotnika $\triangle p_1AB$ in $\triangle p_1BC$ skupno še drugo kateto, trikotnika $\triangle p_1BC$ in $\triangle p_1CD$ pa skupno hipotenuzo, so vsi trije trikotniki skladni z enakim kotom v točki $p_1$, torej nam pregiba skozi daljici $p_1B$ (kar je ravno $L_3$) in $p_1C$ kot $\theta$ razdelijo na tri skladne kote.
\end{dokaz}

Ker ta postopek deluje le za ostre kote, si poglejmo naslednjo metodo, ki jo lahko uporabljamo tako za ostre kot tudi tope kote.

\subsubsection*{Justinova metoda}

Francoski matematik Jacques Justin za svojo metodo trisekcije ne zahteva kvadratnega lista papirja, ampak je dovolj kakršenkoli list. Lang v~\cite[str.\ 34]{lang2013} takole navaja njegovo kosntrukcijo:

Na sredo narišemo poljuben kot $\angle ABC$ (oster ali top) in njuna kraka podaljšamo skozi vrh $B$. Skozi vrh konstruiramo poltrak, pravokoten na krat $BA$. Točko $C$ prezrcalimo čez vrh v točko $D$ ter nato obe točki prepognemo na nosilko kraka $BA$ in ravno konstruirano pravokotnico, kot kaže slika~\ref{fig:justin} (levo). Nazadnje na Belochin pregib konstruiramo še pravokotnico skozi točko $B$.
\begin{figure}[h]
    \centering
    \includegraphics[width=0.45\textwidth]{images/starogr_problemi/justin_trisection.png}
    \includegraphics[width=0.45\textwidth]{images/starogr_problemi/justin_trisection_dokaz.png}
    \caption[Justinova trisekcija kota]{Justinova trisekcija kota (levo) in njen geometrijski dokaz (desno).}
    \label{fig:justin}
\end{figure}

\begin{trditev}
    Kot, ki v točki $B$ oklepata zadnja pravokotnica iz zgornje konstrukcije in krak $BA$, je tretjina kota $\angle ABC$.
\end{trditev}
\begin{dokaz}
    Označimo z $\alpha$ kot iz trditve. Naj bosta točki $C'$ in $D'$ sliki točk $C$ in $D$, točka $B'$ pa presečišče daljice $C'D'$ s pravokotnico iz trditve. Po konstrukciji Belochinega pregiba je daljica $C'D'$ slika daljice $CD$, torej je točka $B'$ slika točke $B$. Točka $B'$ je tako središče daljice $C'D'$ in zato je vzporednica k poltraku $BD'$ skozi točko $B'$ simetrala daljice $C'B$. Trikotnik $\triangle C'B'B$ je tako enakokrak in velja $\angle C'BB' = \angle B'C'B = \alpha$.
    
    Ker sta trikotnika $\triangle C'B'B$ in $\triangle CBB'$ zaradi simetričnosti glede Belochin pregib skladna, velja tudi $\angle B'CB = \angle CB'B = \alpha$. Iz vsote notranjih kotov trikotnike $\triangle CB'B$ sledi $\angle C'BC = 180^\circ - 3\alpha$, torej je $\angle ABC = 180^\circ - \angle C'BC = 3\alpha$. 
\end{dokaz}

\subsubsection*{Martinovi konstrukciji za trisekcijo ostrega kota}

George E.\ Martin v~\cite[poglavje 10]{geometricconstructions} navaja še dve metodi za trisekcijo ostrega kota.

Pri prvi vzamemo oster kot $\angle PQR$ in s točko $M$ označimo središče daljice $PQ$. Skozi $M$ konstruiramo pravokotnico $p$ na $QR$, nato pa še pravokotnico na $p$. Opravimo tisti Belochin pregib (od treh možnih), ki seka daljico $PM$ in točko $Q$ položi na premico $q$ (v točko $Q'$), točko $P$ pa na premico $p$ (v točko $P'$). S $T$ označimo presečišče daljice $QQ'$ s premico $p$ in s $S$ presečišče pregiba s premico $q$ (slika~\ref{fig:trisection_10.4}).

\begin{figure}[h]
    \centering
    \includegraphics[width=0.5\textwidth]{images/starogr_problemi/trisection_10.4.png}
    \caption[Martinova trisekcija ostrega kota (metoda $1$)]{Martinov postopek za trisekcijo kota iz~\cite[str.\ 154]{geometricconstructions}.}
    \label{fig:trisection_10.4}
\end{figure}

\begin{trditev}
    Daljici $QT$ in $QS$ tretjinita kot $\angle PQR$.
\end{trditev}
\begin{dokaz}
    Ker velja $|QM| = |MP|$, $\angle PMP' = \angle TMQ$ in $QT \parallel PP'$, sta trikotnika $\triangle QMT$ in $\triangle PMP'$ skladna in je $|TM| = |MP'|$. Potem sta skladna tudi pravokotna trikotnika $\triangle TMQ'$ in $\triangle P'MQ'$, zato je $\angle MQ'P' = \angle TQ'M = \angle RQQ'$ (zaradi izmeničnih kotov ob vzporednicah $QR$ in $q$) $= \angle Q'QS$ (ker je trikotnik $\triangle QSQ'$ enakokrak).
    
    Bralec se lahko hitro prepriča, da daljica $Q'P'$ seka pregib ravno v njegovem presečišču z daljico $MP$ (vsi koti ob tem presečišču so zaradi konstrukcija pregiba in sovršnosti skladni). Zato velja še $\angle PQQ' = \angle P'Q'Q$, iz česar sledi
    $$ \angle MQS = \angle SQT = \angle TQR.$$
\end{dokaz}

Druga metoda je prvi zelo podobna. Zopet vzamemo oster kot $\angle PQR$ in s točko $M$ označimo središče daljice $PQ$. Naj bo $l$ pravokotnica na $QR$ skozi točko $Q$, točka $N$ nožišče pravokotnice na premico $l$ skozi točko $P$, s $q$ označimo pa še pravokotnico na premico $l$ skozi točko $M$. Opravimo tisti Belochin pregib, ki točko $Q$ položi na premico $q$ (v točko $Q'$) in točko $N$ na poltrak $QR$. Naj bo točka $S$ presečišče pregiba s premico $q$ (slika~\ref{fig:trisection_10.14}).

\begin{figure}[h]
    \centering
    \includegraphics[width=0.6\textwidth]{images/starogr_problemi/trisection_10.14.png}
    \caption[Martinova trisekcija ostrega kota (metoda $2$)]{Martinov postopek za trisekcijo kota iz~\cite[str.\ 158--159]{geometricconstructions}.}
    \label{fig:trisection_10.14}
\end{figure}

\begin{trditev}
    Daljici $QQ'$ in $QS$ tretjinita kot $\angle PQR$.
\end{trditev}
\begin{dokaz}
    Zopet premislimo, da se pregib, poltrak $QP$ in daljica $NQ'$ sekajo v isti točki. Zato je $\angle QQ'N = \angle PQQ'$. Zaradi vzporednosti kraka $QR$ in premice $q$ sta skladna tudi izmenična kota $\angle RQQ'$ in $\angle QQ'S$, z njima pa je zaradi enakokrakosti trikotnika $\triangle QSQ'$ skladen tudi kot $\angle SQQ'$.

    Ker velja $|QM| = |MP|$ in $q \parallel NP$, je premica $q$ simetrala daljice $QN$, torej tudi simetrala kota $\angle QQ'N$. Iz tega sledi
    $$ \angle MQS = \angle SQQ' = \angle Q'QR.$$
\end{dokaz}
\newpage
\section{Zlaganje stožnic}
\label{pogl:stoznice}

% kako konstruiramo tangente na stožnice in zakaj konstrukcije tako delujejo
% parabola, elipsa, hiperbola
% povezava z O5
% pa omeni tudi O6, kjer je skupna tangenta na dve paraboli
% a za krožnico se tudi da?
\newpage
\section{Reševanje enačb}
\label{pogl:enacbe}

\textcolor{red}{popravi da so captioni slik, ki so v več kot eni vrstici, poravbnani na sredini}

Zapustimo deloma področje geometrije in si poglejmo, kako lahko s prepogibanjem papirja rešujemo enačbe z racionalnimi koeficienti.

Spomnimo se še, da smo origami števila definirali kot vsa števila, ki jih lahko s prepogibanjem konstruiramo preko na začetku danega izhodišča $O$ in števila $1$ na realni osi (definicija~\ref{def:origami_stevilo}). V evklidski ravnini (ki je v bijekciji s kompleksno ravnino, dano v definiciji) bomo konstruirali rešitve naših enačb, da pa bo pregibov čim manj in s tem preglednost večja, za označbo pomožnih točk in premic dopuščamo uporabo pisala (saj bi jih tako ali tako znali konstruirati s pregibi).

Začnimo z najbolj osnovno, t.\ j.\ linearno enačbo. Enačba $ax + b = 0$, kjer $a, b \in \Q$ in $a \neq 0$ ima rešitev $x = -b/a$, ki je racionalno število, torej origami število in samo po sebi konstruktibilno. Če bi želeli rešitev konstruirati geometrijsko preko pregibov, v ravnini prepognemo premico $y = ax + b$ (napravimo pregib npr.\ skozi točki (0, b) in (1, a+b)) in njeno presečišče z abscisno osjo nam da iskano rešitev.

Uporaba origamija je za reševanje linearne enačbe očitno manj praktična kot računanje rešitve. Bolj zanimivo je reševanje kvadratne in kubične enačbe. Ker za njune rešitve obstajata splošni formuli, bi jih lahko najprej izračunali in nato preko operacij seštevanja, odštevanja, množenja, deljenja in korenjenja konstruirali s prepogibanjem, vendar je to časovno preveč potratno. Pogledali si bomo, kako se z origamijem lahko temu izognemo in rešitev konstruiramo brez uporabe računskih operacij.

Ključno vlogo bosta v nadaljevanju odigrali origami operaciji~\ref{op:O6} in~\ref{op:O7}. Prva nam hkrati s konstrukcijo tangente na parabolo določi tudi točko na paraboli, skozi katero je pregib tangenten na stožnico, to pa je ekvivalentno reševanju kvadratne enačbe. Druga s konstrukcijo skupne tangente na dve paraboli omogoča reševanje kubične enačbe. Alperin v~\cite[str.\ 129]{alperin2000} pokaže, kako izpeljati koeficient skupne tangente na dani paraboli in izkaže se, da je iskani koeficient rešitev kubične enačbe. Število skupnih tangent je torej enako številu rešitev kubične enačbe, kar pomeni, da imata paraboli v evklidski ravnini največ tri skupne tangente.

V teoriji bi nam prepogibanje papirja pomagalo tudi pri reševanju kvartičnih enačb, saj zanje še obstaja splošna formula (vendar zaradi dolžine praktično neuporabna) in tudi vemo, da lahko enačbo četrte stopnje prevedemo na enačbe nižje stopnje (gl.\ \cite{wikiquartic}, \cite{quartics2012}). Geometrično pa je to reševanje potem težje izvedljivo in zato manj motivacijsko, saj bi postopek reševanja zahteval veliko več pregibov kot pri reševanju ene kubične ali kvadratne enačbe, pa tudi vmesne rezutate bi morali računati. Tako bi se lahko tu z reševanjem enačb preko origamija ustavili, vendar obstajajo alternativne rešitve. V~\cite{edwards2001} je opisan postopek, ki preko \textcolor{red}{projektivne geometrije in dualnih stožnic} ter z Belochinim pregibom reši splošno kubično in nato tudi kvartično enačbo neke določene oblike. Postopek si bomo tudi sami pogledali, saj je zelo zanimiv iz vidika projektivne geometrije.

Za enačbe pete in višjih stopenj pa splošna formula za rešitve ne obstaja več (\emph{Abel-Ruffinijev} izrek, gl.\ \cite{mrinal2019}). Kljub temu se da z origamijem še vedno konstruirati rešitve nekaterih enačb višjih stopenj, vendar ne obstaja postopek z enkratnimi prepogibi -- potrebno se je poslužiti dvojnih (\emph{two fold}) ali večkratnih (\emph{multi-fold}) prepogibov (gl.\ poglavje~\ref{pogl:multifold}).

\subsection{Reševanje kvadratne enačbe preko tangente na parabolo}
\label{podpogl:kvadratna_enacba}

Rešujemo enačbo oblike
$$ a x^2 + b x + c = 0, $$
kjer so $a, b, c \in \Q$ in velja $a \neq 0$.  Njeni splošni rešitvi sta
$$ x_{1,2} = \frac{-b \pm \sqrt{b^2 - 4ac}}{2a}.$$

Postopek, ki si ga bomo pogledali v nadaljevanju, predpostavlja $a = 1$. Ker je vodilni koeficient neničeln, lahko z njim enačbo delimo in pri tem še vedno dobimo racionalne koeficiente, zato lahko predpostavko brez škode za splošnost sprejmemo. Nova oblika enačbe je tako
\begin{equation}
    \label{eq:spl_kv_en}
    x^2 + bx + c = 0.
\end{equation}
Predpostavimo, da ima enačba dve različni realni rešitvi oz.\ da je diskriminanta enačbe pozitivna, t.\ j.\ $D = b^2 - 4c > 0$. Če realnih ničel ni, o origami konstrukciji rešitev namreč nima smisla razpravljati. Če je rešitev ena, je podana kot $x = -b/2$, kar je origami-konstruktibilno število in se ga lahko takoj konstruira.

Enačba~\ref{eq:spl_kv_en} nam poda pokončno parabolo $y = x^2 + bx + c$ z vodoravno premico vodnico in dvema ničlama, ki sta rešitvi naše enačbe. Iščemo absciso presečišča parabole z abscisno osjo.

Zopet se bomo poslužili dosedanjega znanja o operaciji~\ref{op:O6}. Ta nam s pregibom skozi dano točko $B$, ki točko $A$ položi na premico $a$, konstruira tangento na parabolo z goriščem v točki $A$ in premico vodnico $a$.

Naša parabola je z enačbo seveda natančno določena. Ideja iskane konstrukcije rešitev enačbe je določiti tako točko $B$ (najlažje kar na osi parabole), da bi nam izvedba operacije~\ref{op:O6} podala tangento na parabolo ravno v njeni ničli. Želeni pregib mora potekati skozi točko $B$ in gorišče $A$ položiti na tisto točko $A'$ na premici vodnici $a$, ki ima enako absciso kot ničla parabole. (gl.\ sliko~\ref{fig:tockaB_in_O6}). Taka točka $B$ je z osjo parabole in katerokoli izmed ničlama (zaradi simetrije) natanko določena.

\begin{figure}[h]
    \centering
    \includegraphics[width=0.5\textwidth]{images/kvadratna_enacba/tockaB_in_O6.png}
    \caption[Iskanje točke $B$]{Operacijo~\ref{op:O6} skozi iskano točko $B$ poda rešitev kvadratne enačbe.}
    \label{fig:tockaB_in_O6}
\end{figure}

Edina nevarnost, da ta konstrukcija ne bo delovala, je možnost, da točka $B$ kdaj ne bo origami-konstruktibilna točka. Zato sedaj izračunajmo njene koordinate in se prepričajmo, da se to nikoli ne bo zgodilo.

Najprej iz dane enačbe parabole določimo njeno gorišče $A$ in premico vodnico $a$. Spomnimo se, da iz enačbe parabole oblike
$$ (x - x_0)^2 = 2p(y - y_0) $$
takoj razberemo koordinati gorišča $(x_0, y_0)$ in enačbo premice vodnice $y = y_0 - p$. V našem primeru enačbo $y = x^2 + bx + c$ preoblikujemo v
$$ \left(x-\left(-\frac{b}{2}\right)\right)^2 = 2 \cdot \frac{1}{2} \left(y - \left(c - \frac{b^2}{4}\right)\right). $$
S tem sta gorišče $A$ in premica vodnica $a$ določena:
$$ A\left(-\frac{b}{2}, c - \frac{b^2 - 1}{4}\right) \text{ in } a: y = c - \frac{b^2 + 1}{4}. $$

Naj bo $t$ ena izmed rešitev enačbe~\ref{eq:spl_kv_en}. Na premici $a$ z $A'$ označimo točko z absciso $t$. Poiščimo enačbo pregiba, ki gorišče $A$ položi v točko $A'$. Ta pregib bo tangenten na parabolo ravno v njeni ničli, njegovo presečišče z osjo parabole $ x = -b/2 $ pa nam bo določilo točko $B$.

Koeficient nosilke daljice $AA'$ je $ - 1/(2t + b)$, torej je koeficient pregiba $k = 2t + b$. Pregib je po konstrukciji tangenten na parabolo v ničli $(t, 0)$, torej je njegova enačba
$$ y = (2t + b)(x - t) = (2t + b)x - 2t^2 - bt = (2t + b)x - t^2 + c. $$
Pri tem smo upoštevali, da velja $t^2 + bt + c = 0$. Presečišče pregiba in osi parabole je tako točka $B$ z absciso $ x = -b/2 $ in ordinato
$$ y = (2t + b)\left(-\frac{b}{2}\right) - t^2 + c = - t^2 - tb + c - \frac{b^2}{2} = c + c - \frac{b^2}{2} = 2c - \frac{b^2}{2}.$$
Obe koordinati sta racionalni, torej je točka $B$ konstruktibilna točka. Ker leži na osi parabole, nam poda obe rešitvi enačbe -- pregiba sta si simetrična glede na os. Povzemimo sedaj postopek konstrukcije rešitve kvadratne enačbe~\ref{eq:spl_kv_en}:
\begin{enumerate}
    \item V koordinatnem sistemu označimo gorišče $A\left(-\frac{b}{2}, c - \frac{b^2}{4} + \frac{1}{4}\right)$, premico vodnico $a: y = c - \frac{b^2 + 1}{4}$ in točko $B(-\frac{b}{2}, 2c - \frac{b^2}{2})$.
    \item Z operacijo~\ref{op:O6} naredimo pregib skozi točko $B$, ki točko $A$ položi na premico $a$ (če je diskriminanta enačbe pozitivna, sta možna pregiba dva).
    \item Skozi sliko točke $A$ naredimo vertikalen pregib in abscisa njegovega presečišča z abscisno osjo je ničla dane enačbe.
\end{enumerate}

\textbf{Primer:} Poiščimo rešitve enačbe $x^2 - x - 1 = 0$. Določimo obe točki in premico: $A(\frac{1}{2}, -1)$, $B(\frac{1}{2}, -\frac{5}{2})$ in $a: y = -\frac{3}{2}.$. Opravimo operacijo~\ref{op:O6} in označimo presečišče abscisne osi in pravokotnice nanjo skozi sliko točke $A$. Če smo bili pri pregibanju natančni, dobimo presečišči pri $x_{1,2} = \frac{1 \pm \sqrt{5}}{2}$ (gl.\ sliko v~\cite[str.\ 37]{hull2020}).

To še zdaleč ni edini postopek za reševanje kvadratne enačbe. Kot še en lep primer Hull v~\cite[str.\ 38]{hull2020} navaja Lillovo konstrukcijo preko krožnice, lahek dokaz pa je prepuščen bralcu. Hkrati je to primer, kako za rešitev nekega problema najprej najdemo (bolj domačo) evklidsko konstrukcijo, ki jo lahko nato preko origami operacij preobrazimo v origami konstrukcijo -- saj že vemo, da lahko s prepogibanjem papirja konstruiramo vse in še več, kar se da z evklidskim orodjem. Pri obravnavi kubične enačbe bomo spoznali Belochino metodo, ki se jo da aplicirati tudi na kvadratno enačbo, in prilagojen postopek je tako opisan v razdelku~\ref{podpodl:kvadr_en_lill}.

\subsection{Lillova metoda in Belochin pregib}
\label{podpogl:kubicna_enacba}

V tem poglavju se bomo spoznali z Lillovo metodo, s katero lahko v teoriji rešimo enačbo poljubne stopnje. V središču naše pozornosti bo origami postopek za reševanje kubične enačbe, ki ga je odkrila že večkrat omenjena Belocheva, vendar se da po Lillovi metodi z uporabo operacije~\ref{op:O6} rešiti tudi kvadratno enačbo, kar si bomo tudi na hitro pogledali proti koncu razdelka. Poleg tega bomo spoznali tudi več sadov Belochinega pregiba.

Sedaj pa vzemimo enačbo oblike
$$ a x^3 + b x^2 + c x + d = 0, $$
kjer so $a, b, c, d \in \Q$ in velja $a \neq 0$. Tu je navedena ena oblika zapisa njene splošne rešitve:

\begin{align*}
    Q &= \sqrt{(2b^3 - 9abc + 27a^2d)^2 - 4(b^2 - 3ac)^3} \\
    C &= \sqrt[3]{\frac{1}{2}(Q + 2b^3 - 9abc + 27a^2d)} \\
    x_1 &= - \frac{b}{3a} - \frac{C}{3a} - \frac{b^2 - 3ac}{3aC} \\
    x_2 &= - \frac{b}{3a} + \frac{C(1 + i\sqrt{3})}{6a} + \frac{(1 - i\sqrt{3})(b^2 - 3ac)}{6aC} \\
    x_2 &= - \frac{b}{3a} + \frac{C(1 - i\sqrt{3})}{6a} + \frac{(1 + i\sqrt{3})(b^2 - 3ac)}{6aC}
\end{align*}

Operacija~\ref{op:O6} nam je preko konstrukcije tangente na parabolo pomagala rešiti kvadratno enačbo. Spomnimo se, da je Belocheva to v tridesetih letih prejšnjega stoletja nadgradila z operacijo~\ref{op:O7}, ki nam konstruira skupno tangento na dve paraboli hkrati. Po njej jo tudi imenujemo \emph{Belochin pregib}. Z njim je kot prva odkrila resnično moč origami konstrukcij, a je žal trajalo več kot pol stoletja, da so matematiki začeli ceniti njeno odkritje.

\subsubsection{Reševanje kubične enačbe z Belochinim postopkom}

Belocheva je sama odkrila naslednjo metodo reševanja kubične enačbe, kjer nam vsak Belochin pregib poda eno izmed rešitev. Iz začetka poglavja že vemo, da je število rešitev enako številu skupnih tangent, torej številu možnih Belochinih pregibov.

Belocheva v svojem postopku izhaja iz Lillove genialne metode iskanja ničel poljubnih polinomov z realnimi koeficienti, ki si jo bomo v naslednjem razdelku podrobneje pogledali, za njeno aplikacijo pa uporabi avtorsko konstrukcijo -- Belochin kvadrat.

\subsubsection*{Lillova metoda}

Njen avtor je avstrijski inženir Eduard Lill, ki jo je l.\ 1867 opisal v svojem članku~\cite{lill1867}. Gre za inovativen postopek, ki je v svoji osnovi čisto enostaven. Imejmo poljuben polinom $ p(x) = a_n x^n + a_{n-1} x^{n-1} + \ldots + a_2 x^2 + a_1 x + a_0 $ z realnimi koeficienti in iščemo njegove realne ničle, če obstajajo. Lill je iz njegovih koeficientov s sledečim postopkom v ravnini ustvaril enolično pot. Običajno se za njeno konstrukcijo uporablja figuro želve, ki nam kaže, v katero smer se premika pa tudi kam je usmerjena.

Na začetku želvo postavimo v koordinatno izhodišče $O$ tako, da gleda v pozitivno smer $x$-osi. Želva najprej v to smer prehodi razdaljo, enako koeficientu $a_n$. Nato se obrne za $90^\circ$ v nasprotno smer urinega kazalca in prehodi naslednjo razdaljo $a_{n-1}$. To ponovi za vsak koeficient polinoma in po prehojeni razdalji $a_0$ se ustavi v neki točki $T$ (slika~\ref{fig:primera_zelve}). Če je kateri od koeficientov negativen, želva hodi ritensko (primer (b) na sliki~\ref{fig:primera_zelve} za koeficiente $a_3, a_2$ in $a_0$), v primeru ničelnega koeficienta pa obstoji na mestu in se samo obrne. S potjo želve dobimo lomljeno črto iz največ $n+1$ daljic, ki jih brez škode označujmo kar z njihovimi ``pripadajočimi'' koeficienti.

\begin{figure}[h]
    \centering
    \includegraphics[width=0.9\textwidth]{images/kubična enačba/primera_zelvine_poti.png}
    \caption[Primera želvine poti]{Primera želvine poti za polinoma pete stopnje. Vzeto iz~\cite[str.\ 311]{hull2011}.}
    \label{fig:primera_zelve}
\end{figure}

Sedaj se v izhodišče $O$ postavimo še mi in z laserskim žarkom poskusimo zadeti želvo v točki $T$. Žarek najprej usmerimo daljico $a_{n-1}$, od katere se odbije v daljico $a_{n-2}$, od te v daljico $a_{n-3}$ in tako naprej. (slika~\ref{fig:primera_zelve}). Pri tem upoštevamo troje:
\begin{itemize}
    \item laserski žarek ne upošteva odbojnega zakona in se od daljice vedno odbije pod kotom $90^\circ$, zato so vpadni koti žarka na vse daljice med seboj enaki in prav tako to velja za odbojne kote;
    \item žarek se lahko odbije tudi od nosilke daljice;
    \item vsakič sta možni dve smeri odboja -- na isto ali drugo stran daljice oz.\ njene nosilke -- izberemo pa tisto, ki nam omogoči, da sploh lahko zadenemo naslednjo daljico.
\end{itemize}
Recimo, da smo zmogli zadeti želvo. Kot, ki ga v točki $O$ oklepata laserski žarek in abscisna os, označimo z $\theta$.

\begin{trditev}
    Število $x_{\theta} = - \tan \theta$ je ničla polinoma $p(x)$.
\end{trditev}

\begin{dokaz}
    Vzemimo primer, ko so vsi koeficienti polinoma $p(x)$ pozitivni. Želvina pot je v tem primeru sestavljena iz $n+1$ daljic, pot laserskega žarka (ki se vedno odbije od daljice in ne njene nosilke) pa iz $n$ daljic. Slednje so ravno hipotenuze pravokotnih trikotnikov. Za vsako od njih je nasprotna kateta kota $\theta$ del daljice $a_i$, priležno kateto pa označimo z $y_i$ ($ n \geq i \geq 1$). dobimo
    \begin{align*}
        y_n &= \tan \theta \cdot a_n = - x_{\theta} a_n \\
        y_{n-1} &= \tan \theta \cdot (a_{n-1} - y_n) = - x_{\theta} (a_{n-1} + x a_n) = - (a_{n-1} x_{\theta} + a_n x_{\theta}^2)\\
        y_{n-2} &= \tan \theta \cdot (a_{n-2} - y_{n-1}) = - x_{\theta} (a_{n-2} + a_{n-1} x_{\theta} + a_n x_{\theta}^2) = \\
        &= - (a_{n-2} x_{\theta} + a_{n-1} x_{\theta}^2 + a_n x_{\theta}^3) \\
        &\vdots \\
        y_1 &= - (a_1 x_{\theta} + a_2 x_{\theta}^2 + \ldots + a_{n-1} x_{\theta}^{n-1} + a_n x_{\theta}^n).
    \end{align*}
    V zadnji enakosti desno stran premaknimo na levo in upoštevamo $y_1 = a_0$. Dobimo ravno $p(x_{\theta}) = 0$, torej je $x_{\theta} = - \tan \theta$ res ničla tega polinoma.

    \textcolor{red}{Primer negativnih koeficientov:~\cite[str.\ 36]{zore2022}.}

    \textcolor{red}{Primer ničelnih koeficientov: isto kot prej, samo se spusti $y_i$ za tisti $i$, za katerega je $a_i = 0$. (\textcolor{red}{???})}
\end{dokaz}

Če pod nobenim kotom $\theta$ ne moremo zadeti želve, je polinom $p(x)$ brez realnih ničel.

Pojavi se nam vprašanje, kako določiti kot $\theta$. Za polinom tretje stopnje je Belocheva preko svojega pregiba našla zelo preprosto rešitev, ki si jo bomo sedaj pogledali.

\subsubsection*{Belochin kvadrat}

Imejmo dani točki $A$ in $B$ ter premici $r$ in $s$. Z origamijem konstruirajmo kvadrat $WXYZ$, kjer oglišče $X$ leži na premici $r$, njegovo sosednje oglišče $Y$ pa na premici $s$. Velja še, da točka $A$ leži na stranici $WX$ (ali njeni nosilki), točka $B$ pa na stranici $ZY$ (ali njeni nosilki, slika~\ref{fig:beloch_kvadrat}).

\begin{figure}[h]
    \centering
    \includegraphics[width=0.4\textwidth]{images/kubična enačba/beloch_kvadrat.png}
    \caption[Belochin kvadrat]{Belochin kvadrat. Vzeto iz~\cite[str.\ 309]{hull2011}.}
    \label{fig:beloch_kvadrat}
\end{figure}

Belocheva je iznašla naslednji postopek, ki nam konstruira ta kvadrat:
\begin{itemize}
    \item Najprej konstruiramo premico $r'$, ki je vzporedna premici $r$ in od nje enako oddaljena kot točka $A$, tako da premica $r$ leži med točko $A$ in premico $r'$. Na enak način premici $s$ konstruiramo njeno vzporednico $s'$ (slika~\ref{fig:beloch_kvadrat_konstrukcija} levo). To konstrukcijo opravimo s prepogibi iz operacije~\ref{op:O5}, zrcaljenja točke čez premico ter ponovne uporabe operacije~\ref{op:O5}. Zaradi preglednosti seveda dopuščamo, da namesto zrcaljenja preprosto prepognemo po premici in s svinčnikom označimo sliko točke.
    \item Nato opravimo Belochin pregib, ki točko $A$ slika v točko $A'$ na premici $r'$, točko $B$ pa v točko $B'$ na premici $s'$ (slika~\ref{fig:beloch_kvadrat_konstrukcija} na sredi).
    \item Naj bo točka $X$ središče daljice $AA'$ in točka $Y$ središče daljice $BB'$. Ker je pregib simetrala teh dveh daljic $AA'$ in $BB'$, sta njuni središči po konstrukciji\footnote{Gledamo lahko dva podobna pravokotna trikotnika s skupnim ogliščem v točki $A$ (oz.\ $B$), enega dvakrat večjega od drugega} ravno presečišči pregiba s premicama $r$ in $s$ (slika~\ref{fig:beloch_kvadrat_konstrukcija} desno).
    \item Daljica $XY$ -- ena izmed stranic kvadrata -- je po konstrukciji pravokotna na daljici $AX$ in $BY$, zato samo še določimo točki $W$ in $Z$ na daljicah ali njunih nosilkah in tako dobimo Belochin kvadrat.
\end{itemize}

\begin{figure}[h]
    \centering
    \includegraphics[width=0.95\textwidth]{images/kubična enačba/beloch_kvadrat_konstrukcija.png}
    \caption[Konstrukcija Belochinega kvadrata]{Konstrukcija Belochinega kvadrata z origamijem. Vzeto iz~\cite[str.\ 310]{hull2011}.}
    \label{fig:beloch_kvadrat_konstrukcija}
\end{figure}

\subsubsection*{Konstrukcija $\sqrt[3]{2}$ z Belochinim kvadratom}
\label{podpogl:beloch_kvadrat_koren}

Preden ravno naučeno znanje uporabimo za reševanje kubičnih enačb, si še na hitro poglejmo, kako lahko tudi z Belochinim kvadratom rešimo starogrški problem podvojitve kocke.

Za premico $r$ vzemimo ordinatno os, za premico $s$ pa abscisno os. Določimo še $A = (-1,0)$ in $B = (0, -2)$. Vzporednici sta torej $r': x = 1$ in $s': y = 2$. Belochin pregib seka premico $r$ v točki $X$, premico $s$ pa v točki $Y$ (slika~\ref{fig:beloch_koren}). Z $O$ označimo koordinatno izhodišče in opazimo podobne pravokotne trikotnike $OAX$, $OXY$ in $OYB$. Z upoštevanjem $|AO| = 1 $ in $|OB| = 2$ dobimo sledeča razmerja:
$$ \frac{|OX|}{|AO|} = \frac{|OY|}{|OX|} = \frac{|OB|}{|OY|} \Longrightarrow |OX| = \frac{|OY|}{|OX|} = \frac{2}{|OY|}, $$
iz česar sledi
$$ |OX|^3 = |OX| \cdot \frac{|OY|}{|OX|} \cdot \frac{2}{|OY|} = 2 \Longrightarrow |OX| = \sqrt[3]{2}. $$

\begin{figure}[h]
    \centering
    \includegraphics[width=0.5\textwidth]{images/kubična enačba/beloch_koren.png}
    \caption[Konstrukcija kubičnega korena števila dva]{Konstrukcija $\sqrt[3]{2}$ preko Belochinega kvadrata. Vzeto iz~\cite[str.\ 310]{hull2011}.}
    \label{fig:beloch_koren}
\end{figure}

Vidimo lahko, da je to enaka konstrukcija, kot jo je 50 let kasneje neodvisno od Belocheve odkril G.\ Martin (razdelek~\ref{podpogl:podvojitev_kocke}), le da je za točko $B$ vzel točko $(0, -k)$ in s tem konstruiral dolžino $\sqrt[3]{k}$ za poljubno origami število $k$.

\subsubsection*{Združitev Lillove metode in Belochinega kvadrata}

Za poljubno enačbo $a x^3 + b x^2 + c x + d = 0$, kjer $ a \neq 0$, povežimo sedaj Lillovo metodo s konstrukcijo primernega Belochinega kvadrata, ki nam bo natančno določil kot $\theta$. Postopek je sledeč (gl.\ tudi sliko~\ref{fig:beloch_kubicna_resitev}):

\begin{enumerate}
    \item Za točko $A$ vzemimo izhodišče $O$. Začrtamo želvino pot za polinom $p(x) = a x^3 + b x^2 + c x + d$, ki se začne v točki $A$ in konča v točki $B$. V primeru neničelnih koeficientov je pot sestavljena iz štirih stranic, pot laserskega žarka pa iz treh.
    \item Premica $r$ naj bo nosilka daljice $b$ ($r: x = a$), premica $s$ pa nosilka daljice $c$ ($s: y = b$).
    \item Določimo premici $r': x = 2a$ in $s': y = b + d$ ter opravimo Belochin pregib, ki točko $A$ položi na premico $r'$ in točko $B$ na premico $s'$.
    \item Presečišči pregiba s premicama $r$ in $s$ zaporedoma označimo s točkama $X$ in $Y$.
    \item Zarišemo daljice $AX$, $XY$ in $YB$.
\end{enumerate}

Ker po konstrukciji velja $ AX \perp XY \perp YB $, je to iskana pot laserskega žarka, ki se odbija pod pravim kotom in zadene želvo. Kot $\theta$ je kot, ki ga oklepata daljici $a_3$ in $AX$. Rešitev je torej $x_{\theta} = - \tan \theta$.

\begin{figure}[h]
    \centering
    \includegraphics[width=0.4\textwidth]{images/kubična enačba/beloch_kubicna_resitev.png}
    \caption[Lillova metoda z Belochinim kvadratom]{Konstrukcija želvine poti za Lillovo metodo preko Belochinega kvadrata. Vzeto in preurejeno iz~\cite[str.\ 313]{hull2011}.}
    \label{fig:beloch_kubicna_resitev}
\end{figure}

Če ima enačba še dve realni rešitvi, sta možna tudi še dva Belochina pregiba (enačba namreč ne more imeti točno dveh realnih rešitev, saj kompleksne rešitve nastopajo v konjugiranih parih).

\opomba{V resnici nikoli do sedaj nismo potrebovali konstruirati celega kvadrata; potrebovali smo le stranico $XY$ in dejstvo, da je pregib pravokoten na daljici $AX$ in $BY$.}

\opomba{Enačbe premic $r, s, r'$ in $s'$ so univerzalne in zgornja konstrukcija tako deluje tudi v primeru, ko je kakšen od koeficientov $b, c, d$ ničeln.}

Kot zanimivost Lavričeva v~\cite[str.\ 10--13]{lavric2013} s postopkom, ki je malo preurejen Belochin postopek, še analitično pokaže, da je ob primerno izbranih točkah $A$ in $B$ ter premicah $r$ in $s$ koeficient tangentnega pregiba rešitev kubične enačbe. Točki in premici izbere tako, da sta točki $X$ in $Y$ ravno presečišči z ordinatnima osema, iz česar lahko takoj razberemo koeficient tangente. V dokazu izpelje enačbi pripadajočih parabol in splošno enačbo njunih tangent ter iz tega dokaže rečeno. To je lahko odlična vaja za dijake, ki si želijo kakšnega izziva.

\subsubsection*{Primer reševanja kubične enačbe po Lillovi metodi}

Več primerov uporabe Belochinega postopka za reševanje kubičnih enačb je opisanih v~\cite[38--44]{zore2022}, tu pa si poglejmo, kako rešimo enačbo, ki ima tako pozitivne kot tudi negativne in ničelne koeficiente. Hull v~\cite[str.\ 90--92]{hull2013} obravnava enačbo
$$ x^3 - 7x - 6 = 0.$$
Po Lillovi metodi za točko $A$ vzamemo izhodišče $O$ in začrtamo želvino pot. Najprej gremo za $1$ v desno ($a=1$), se obrnemo za $90^\circ$ v pozitivno smer in obstojim na mestu ($b=0$), se zopet obrnemo  $90^\circ$ v pozitivno smer in se premaknemo za $7$ v nasprotno smer, torej zopet v desno ($c=-7$), na koncu pa se po ponovnem obratu za $90^\circ$ v pozitivno smer premaknemo za 6 navzgor ($d=-6$). Končamo v točki $T = (8, 6)$.

Označimo še premice $r: x = a = 1, r'$ (na sliki~\ref{fig:lill_primer1} označena z $L_1$) $: x = 2a = 2, s: y = b = 0$ in $s'$ (na sliki~\ref{fig:lill_primer1} označena z $L_2$) $: y = b + d = -6$.

\begin{figure}[h]
    \centering
    \includegraphics[width=0.4\textwidth]{images/kubična enačba/lill_primer_setup.png}
    \caption[Primer reševanja z Lillovo metodo (priprava)]{Priprava želvine poti in premic za reševanje enačbe z Lillovo metodo. Vzeto iz~\cite[str. 87]{hull2013}.}
    \label{fig:lill_primer1}
\end{figure}

Želvina pot je sedaj pripravljena, da točko $O$ prepognemo na premico $L_1$ in hkrati točko $T$ na premico $L_2$. Presečišči pregiba s premicama $r$ in $s$ nam data točki, kjer se laserski žarek za pravi kot odbije in na koncu zadane točko $T$. Kot, ki ga oklepa žarek z $x$-osjo ob koordinatnem izhodišču, nam podaja rešitev enačbe.

Na sliki~\ref{fig:lill_primer_pregibi} so konstrukcije vseh treh možnih pregibov. Začnimo z leve proti desni in za vsakega od njih pogledamo, kaj dobimo:
\begin{enumerate}
    \item V prvem primeru se nam točka $O$ preslika v točko $(2,2)$, točka $T$ pa v točko $(-4,-6)$, torej je kot $\theta = 45^\circ$, kar pomeni $x_{45^\circ} = -\tan 45^\circ = -1$. Preverimo, ali $x_{45^\circ}$ res reši našo enačbo. Zares, $(-1)^3 - 7\cdot(-1) - 6 = 0$.
    \item Točka $T$ se v drugem primeru preslika ravno v presečišče premic $L_1$ in $L_2$, torej v točko $(2,-6)$. Pregib torej premico $s$, ki je v našem primeru kar $x$-os, seka v točki $(5,0)$ (ker je presečišče središče daljice s krajišči v točki $T$ in njeni sliki). Rešitev $x_\theta$ lahko preberemo kar iz zadnjega pravokotnega trikotnika preko definicije konte funkcije tangens -- $x_\theta = -tan \theta = -6/3 = -2$. In res je $(-2)^3 - 7\cdot(-2) - 6 = 0$.
    \item V zadnjem primeru pa se v presečišče premic $L_1$ in $L_2$ preslika izhodišče $O$. Zato pregib premico $r$ seka v točki $(1,-3)$ in za kot $\theta$ gledamo prvi pravokotni trikotnik. Dobimo $x_\theta = -tan \theta = -(-3)/1 = 3$. Preverimo rešitev in dobimo $3^3 - 7\cdot3 - 6 = 0$.
\end{enumerate}

\begin{figure}[h]
    \centering
    \includegraphics[width=0.33\textwidth]{images/kubična enačba/lill_primer_pregib1.png}
    \includegraphics[width=0.65\textwidth]{images/kubična enačba/lill_primer_pregib2_3.png}
    \caption[Primer reševanja z Lillovo metodo (s pregibi)]{Pregibi, ki rešijo kubično enačbo $x^3 - 7x - 6 = 0$. Vzeto iz~\cite[str. 91--92]{hull2013}.}
    \label{fig:lill_primer_pregibi}
\end{figure}

Seveda se da enačbo $x^3 - 7x - 6 = 0$ hitreje rešiti z računanjem, vendar je ravno zaradi tega lep primer za uporabo Lillove metode z Belochinim pregibom, saj se da rešitve iz konstrukcije (ob natančnih prepogibih) takoj prebrati.

\subsubsection{Reševanje dveh starogrških problemov z Belochinim postopkom}

Problema \emph{podvojitve kocke} in \emph{trisekcije kota} smo rešili že v razdelku~\ref{podpogl:starogrskiproblemi}, zato si tukaj le pogledamo nastavek reševanja še z uporabo ravno spoznane metode. Enostavno povedano -- preblema prevedimo v reševanje kubičnih enačb. Bralec je ob sledečih enačbah povabljen, da sam konstruira želvino pot in opravi Belochin pregib.

Pri podvojitvi kocke je to zelo enostavno, saj rešujemo enačbo $x^3 - 2 = 0$.

Za trisekcijo kota se spomnimo, da velja $\cos 3\theta = 4 \cos^3 \theta - 3 \cos \theta$. Naj bo $k = \cos 3\theta$ dana konstanta (saj imamo dan poljuben kot, za lažje reševanje recimo $3 \theta = 60^\circ$, ki se ga z evklidskim orodjem ne da tretjiniti), iščemo pa $x = \cos \theta$. Torej rešujemo enačbo $4x^3-3x-k=0$.

\textcolor{red}{Lahko kkšna slikca. Pa če enačbe omeniš že v poglavju o starogrških problemih, pol tle to podpoglavje samo spremeni v opombo, npr.\ spomnimo se blablabla in da lahko bralec te enačbe reši z Lillovo metodo.}

\subsubsection{Reševanje kvadratne enačbe z Lillovo metodo}
\label{podpodl:kvadr_en_lill}

Lillovo lahko uporabimo tudi za reševanje kvadratne enačbe $a x^2 + b x + c = 0, a \neq 0$. Na enak način v koordinatni sistem zarišemo želvino pot, ki se začne v točki $A$ in konča v točki $B$. Za razliko od prej tu ne uporabimo Belochinega pregiba, temveč pregib iz operacije~\ref{op:O6}. Namesto dveh premic $r$ in $s$ imamo le eno -- naj bo $r$ nosilka daljice $b$. Kot prej -- na razdalji $a$ na drugi strani točke $A$ -- označimo še njeno vzporednico $r'$. Konstruiramo pregib, ki gre skozi točko $B$ in točko $A$ položi na premico $r'$. Njegovo presečišče s premico $r$ nam določi točko $X$, kjer se žarek iz točke $A$ pod pravim kotom odbije v točko $B$. Na sliki~\ref{fig:kv_en_lill} je primer kosntrukcije pri negativnem koeficientu $c$. S tem je kot $\theta$ določen. Premislili smo tudi že, da sta možna največ dva pregiba in da je število pregibov enako številu realnih rešitev enačbe.

\begin{figure}[h]
    \centering
    \includegraphics[width=0.5\textwidth]{images/kvadratna_enacba/kvadratna_enacba_lillova_metoda.png}
    \caption[Lillova metoda za kvadratno enačbo]{Reševanje kvadratne enačbe po Lillovi metodi z operacijo~\ref{op:O6} ($c < 0$).}
    \label{fig:kv_en_lill}
\end{figure}

\subsubsection{Hatorijeva konstrukcija}

Japonski matematik Koshiro Hatori navaja postopek, ki je zelo podoben Belochinem postopku, vendar ga je avtor iznašel neodvisno od Belochinega dela. Brez škode za splošnost predpostavi $a = 1$ in za reševanje enačbe $x^3 + bx^2 + cx + d = 0$ sledi naslednjim korakom:
\begin{itemize}
    \item Vkoordinatnem sistemu označimo točki $A = (b, 1)$ in $B = (d, c)$ ter premici $a: y = -1$ in $b: x = -d$.
    \item Opravimo pregib, ki točko $A$ položi na premico $a$ ter točko $B$ na premico $b$ (kar je ravno Belochin pregib).
\end{itemize}
Avtor zaključi, da je koeficient opravljenega pregiba rešitev naše enačbe.

Bralec lahko sam premisli, da je to v resnici ravno Belochin postopek, le da se želva na začetku svoje poti ne nahaja v koordinatnem izhodišču in je najprej usmerjena navzdol. Prav tako lahko izrazi koeficient pregiba s kotom ob začetni točki $A$ in res dobi $k = - \tan \theta$. \textcolor{red}{(to si tudi sama preverila in res drži)}. Za geometrijsko razlago preko parabol gl.~\cite{hatori2003}.

\opomba{Seveda bi lahko vzeli katerikoli $a \in \Q$ in vzeli premico $a: y = -a$.}

\subsection{Kubična in kvartična enačba v afini ravnini}

\textcolor{red}{A je naslov ok ali je preveč misteriozen in kontraverzen?}

Kot že omenjeno v uvodu tega poglavja, se Edwards in Shurman v~\cite{edwards2001} ukvarjata z reševanjem enačb tretje in četrte stopnje preko iskanja skupnih tangent na določene stožnice, pri tem pa uporabljata princip Belochinega pregiba. Pri kubični enačbi iz njenih koeficientov določita gorišči in premici vodnici dveh parabol, rešitve enačbe pa so koeficienti skupnih tangent. Postopek za reševanje kvartične enačbe je podoben, le da iz njenih koeficientov določita parabolo in krožnico, rešitve pa so začetne vrednosti skupnih tangent. Videli bomo, da slednji postopek deluje le za nekatere kvartične enačbe.

Preden se podamo na natančnejšo razčlenitev njunega dela, se najprej vprašajmo, kako lahko z origamijem določimo skupno tangento na parabolo in krožnico -- do sedaj to namreč znamo le v primeru dveh parabol (preko Belochinega pregiba). Postopek je v svojem bistvu zelo enostaven. Pri danem gorišču in premici vodnici parabole ter krožnici s središčem v točki $S$ in polmerom $r$ zarišemo krožnico z istim središčem ter dvakratnim polmerom, torej $2r$. Nato po zgledu Belochinega pregiba opravimo pregib, ki gorišče parabole položi na njeno premico vodnico, središče $S$ pa na rob krožnice s polmerom $2r$. Pregib je tako res skupna tangenta na parabolo in krožnico s polmerom $r$ (slika \textcolor{red}{dej kakšen slikovni primeeeeer -- lahko kar sliko 3 iz tega vira}).

\begin{opomba}
    V tem razdelku za potrebe reševanja izjemoma potrebujemo šestilo, s katerim iz koeficientov kvartične enačbe konstruiramo krožnico.
\end{opomba}

Postopek se trenutno lahko dozdeva enostaven, vendar nas do enačb iskanih stožnic čaka še dolga pot. (\textcolor{red}{Nekej od tega de se moramo spomnit splošne enačbe stožnic in afine pa projektivne geometrije blablabla})

\textcolor{red}{Definicija projektivne geometrije nad vektorskim prostorom V (mi bomo imeli V = $\R^3$.)}

Stožnica $\mathcal{S}$ ima v $\mathcal{P}(\R^3)$ \textcolor{red}{(A ta``P'' je normalen font al tak kot tle?)} homogenizirano \textcolor{red}{(?)} enačbo
\begin{equation}
    \label{en:stoznica_splosna}
    \mathcal{S}: a_{11}x^2 + 2a_{12}xy + a_{22}y^2 + 2a_{13}xz + 2a_{23}yz + a_{33}z^2 = 0,
\end{equation}
kar lahko zapišemo v obliki
\begin{equation*}
    \mathcal{S}: v^\intercal M v = 0,
    \text{ kjer sta } v =
    \begin{bmatrix}
        x\\
        y\\
        z
    \end{bmatrix}
    \text{ in } M =
    \begin{bmatrix}
        a_{11} & a_{12} & a_{13}\\
        a_{12} & a_{22} & a_{23}\\
        a_{13} & a_{23} & a_{33}
    \end{bmatrix}.
\end{equation*}

Pri tem je simetrična matrika $M$ definirana do neničelnega skalarnega večkratnika natančno. Stožnica $\mathcal{S}$ je \emph{neizrojena} (ni unija dveh premic ali ene, dvojno štete premice), če ima poln rang, t.\ j.\ \textcolor{red}{rang (-- dej v matematično okolje?)} $ M = 3$ oz. ekvivalentno, $\det M \neq 0$. V nadaljevanju bomo delali le z neizrojenimi stožnicami, torej vedno obstaja inverz matrike $M$, ki je prav tako simetričen.

Iz prvega minorja matrike $M$ lahko takoj preberemo, za katero vrsto neizrojene stožnice gre. Naj bo $A_M = a_{11}a_{22} - a_{12}^2$ \textcolor{red}{(Vir tega je Wikipedia: Matrix representation of conic sections. Kakšen bolj zanesljiv vir?)}:
\begin{itemize}
    \item $\mathcal{S}$ je hiperbola, če in samo če $A_M < 0$,
    \item $\mathcal{S}$ je parabola, če in samo če $A_M = 0$ in
    \item $\mathcal{S}$ je elipsa, če in samo če $A_M > 0$. Če poleg tega velja še $a_{11} = a_{22}$ in $a_{12} = 0$, je $\mathcal{S}$ krožnica.
\end{itemize}

\textcolor{red}{(Definicija dualnosti?)}

\textcolor{red}{\emph{Dualno stožnico} (a je prevod ok?)} stožnice $\mathcal{S}$ definiramo z inverzno matriko:
\begin{equation*}
    \mathcal{\hat{S}}: v^\intercal M^{-1} v = 0.
\end{equation*}

Naj bo $p \in \mathcal{S}$ točka na stožnici $\mathcal{S}$ in naj bo $q = Mp$. Potem je
 $$q^\intercal M^{-1} q = (Mp)^\intercal M^{-1} (Mp) = p^\intercal M^\intercal M^{-1} M p = p^\intercal M p = 0,$$
 kar pomeni, da je $q \in \mathcal{\hat{S}}$. Ker je $M$ obrnljiva, je preslikava $\mathcal{S} \longrightarrow \mathcal{\hat{S}}$ s predpisom $p \mapsto Mp$ bijekcija med stožnico $\mathcal{S}$ in njeno dualno stožnico $\mathcal{\hat{S}}$. Kaj pa so točke dualne stožnice $\mathcal{\hat{S}}$? (\textcolor{red}{Dokončaj; zakaj je $q$ ravno tangenta na $\mathcal{S}$?})

 Naj bosta $\mathcal{S}_1$ in $\mathcal{S}_2$ stožnici s pripadajočima simetričnima obrnljivima matrikama $M_1$ in $M_2$. Potem je njuna skupna tangenta skupna točka njunih dualnih stožnic $\mathcal{\hat{S}}_1$ in $\mathcal{\hat{S}}_2$. Torej iščemo $q \in \mathcal{\hat{S}}_1, \mathcal{\hat{S}}_2$, (\textcolor{red}{a je q res prou tangenta? to je pač točka od dualne stožnice}) ki reši sistem enačb
 \begin{equation}
    \label{eq:afin_sistem_tangenta_splosen}
    q^\intercal M^{-1}_1 q = 0 \; \text{ in } \; q^\intercal M^{-1}_2 q = 0.
 \end{equation}
 \textcolor{red}{kakšne oblike je q? oziroma tangenta?}

 \subsubsection*{Reševanje kubične enačbe}

 Zopet rešujemo kubično enačbo oblike $ x^3 + bx^2 + cx + d = 0, d \neq 0$. Za stožnici vzamemo naslednji paraboli:
 \begin{itemize}
    \item $\mathcal{P}_1: (y+c)^2 = -4d(x-b)$ z goriščem $F_1 (b - d, -c)$ in premico vodnico $L_1: x = b + d$,
    \item $\mathcal{P}_2: x^2 = -4y$ z goriščem $F_2 (0, -1)$ in premico vodnico $L_2: y = 1$.
 \end{itemize}

 Iz enačb parabol zapišemo njuni matriki
$$ M_1 =
    \begin{bmatrix}
        0 & 0 & 2d\\
        0 & 1 & c\\
        2d & c & c^2-4bd
    \end{bmatrix}
    \; \text{ in } \; M_2 =
    \begin{bmatrix}
        1 & 0 & 0\\
        0 & 0 & 2\\
        0 & 2 & 0
    \end{bmatrix}
$$
ter izračunamo njuna inverza
$$ M^{-1}_1 = \frac{1}{d}
    \begin{bmatrix}
        b & -c/2 & 1/2\\
        -c/2 & d & 0\\
        1/2 & 0 & 0
    \end{bmatrix}
\; \text{ in } \; M^{-1}_2 =
    \begin{bmatrix}
        1 & 0 & 0\\
        0 & 0 & 1/2\\
        0 & 1/2 & 0
    \end{bmatrix}.
$$

Naj bo $q = [A\;B\; C]^\intercal \in \mathcal{\hat{P}}_1, \mathcal{\hat{P}}_2$ skupna tangenta (\textcolor{red}{je to res ``tangenta'' al kako se temu reče?}) na paraboli $\mathcal{P}_1$ in $\mathcal{P}_2$. Iz sistema enačb~\ref{eq:afin_sistem_tangenta_splosen} dobimo nov sistem
\begin{equation}
    \label{eq:afin_sistem_tangenta_ABC_kub}
    bA^2 - cAB + AC + dB^2 = 0 \; \text{ in } \; -BC = A^2.
\end{equation}

Da dobimo \textcolor{red}{nevertikalno afino skupno} tangento na paraboli $\mathcal{P}_1$ in $\mathcal{P}_2$, normaliziramo $B = -1$ in $z = 1$, s čimer dehomogeniziramo (\textcolor{red}{? poglej kako je z izrazi, pa $z = 1$ je verjetno pač k je afina ravnina, zakaj pa je $B = -1$?}) tangento $Ax + By + Cz = 0$ v $y = Ax + C$. S tem se druga enačba v sistemu~\ref{eq:afin_sistem_tangenta_ABC_kub} preuredi v $C = A^2$, prva pa v
$$ A^3 + bA^2 + cA + d = 0,$$
kar pomeni, da nam $A$, koeficient skupne tangente, reši izvorno kubično enačbo.
\begin{opomba}
    S konstrukcijo pregibov pa še ne dobimo njegovega koeficienta. V praksi lahko to storimo tako, da poiščemo presečišče pregiba z $x$-osjo, npr.\ točko $(x_0, 0)$ in na razdalji $1$ v desno konstruiramo  pravokotnico na $x$-os skozi točko $(x_0 + 1, 0)$, ki bo tangento sekala ravno v točki $(x_0 + 1, A)$. Sedaj lahko tudi uradno potrdimo konstrukcijo realne rešitve kubične enačbe.
    \textcolor{red}{Lahko daš eno slikco za izi prikaz.} 
\end{opomba}

\textcolor{red}{Dodaj konkreten primer iz članka, origami pregibe si tudi fizično sprobala na enem listi.}

\subsubsection*{Reševanje kvartične enačbe}

Kot že povedano, sledeči postopek ne rešuje splošne enačbe četrte stopnje, temveč njeno zreducirano obliko
\begin{equation}
    \label{eq:reduc_kvart_ev}
    x^4 + bx^2 + 2cx + d = 0
\end{equation}
(\textcolor{red}{Omeni, da se da vsako kvartično enačbo zreducirat v to?})
Smiselno predpostavimo $d \neq 0$, saj bi se v nasprotnem primeru pri eliminaciji ničle $x = 0$ enačba prevedla na kubično. Poleg tega predpostavimo še $c \neq 0$, ki nam prepreči, da bi se enačba z uvedbo nove spremenljivke za $x^2$ prevedla celo na kvadratno enačbo.

\textcolor{red}{omeni Bezuatov izrek al kaj je že in da dualne stožnice imajo kvadratno enačbo, kar pomeni, da imata dve največ štiri skupne točke (tangente) v kompleksni projektivni ravnini; ker so inverzne matrike realne, nastopajo v konjugiranih parih, torej imata po nič, dve ali štiri skupne tangente (šteto z večkratnostjo). A je tu not tudi tangenta v neskončnosti všteta? Skratka, dve paraboli pa imata skupno tangento v neskončnosti, kar pomeni, da potem ne pustita dovolj afinih tangent, ki bi rešile kvartično enačbo (ker so afine tangente največ tri). Zato ne moremo vzeti dveh parabol.}

Zgornjo metodo bi lahko uporabili tudi tu. Avtorja pri predpostavki $bd - c^2 \neq 0$ vzameta naslednji stožnici:
\begin{itemize}
    \item $\mathcal{S}: dx^2 + 2cxy + by^2 + bd - c^2 = 0$ in
    \item parabolo od prej, t.\ j.\ $\mathcal{P}: x^2 = -4y$.
\end{itemize}
Bralec je povabljen, da po enakem postopku kot pri kubični enačbi izpelje, da koeficient skupne tangente $y = Ax + C$ reši enačbo~\ref{eq:reduc_kvart_ev}. \textcolor{red}{(To si izpeljala)}. Stožnicama $\mathcal{S}$ in $\mathcal{P}$ zaporedoma pripadata matriki
$$ M_\mathcal{S} =
    \begin{bmatrix}
        d & c & 0\\
        c & b & 0\\
        0 & 0 & bd - c^2
    \end{bmatrix}
    \; \text{ in } \; M_\mathcal{P} =
    \begin{bmatrix}
        1 & 0 & 0\\
        0 & 0 & 2\\
        0 & 2 & 0
    \end{bmatrix}.
$$
Ker za glavni minor matrike $M_\mathcal{S}$ velja $A_{M_\mathcal{S}} = bd - c^2 \neq 0$, stožnica $\mathcal{S}$ ni parabola in ker velja $c \neq 0$, ne more biti niti krožnica. Torej je lahko le elipsa ali hiperbola. Tu pa nastane težava, saj avtorja nista uspela najti splošne geometrijske metode za konstrukcijo skupne tangente na parabolo in elipso oz.\ hiperbolo. Ker pa znamo konstruirati skupno tangento na parabolo in krožnico, predlagata malo prilagojen postopek, ki je sicer algebraično manj eleganten, vendar geometrijsko toliko lažje izveden.

Pri danih koeficientih $b, c, d$ enačbe~\ref{eq:reduc_kvart_ev} vpeljimo novi oznaki
$$ e = \frac{\sqrt{bd - c^2}}{d} \; \text{ in } \; r = |e|\sqrt{-d}, $$
iz česar sledita relaciji, ki ju bomo kasneje uporabili pri izpeljavi:
\begin{equation}
    \label{eq:kvarticni_relaciji}
    b= de^2 + \frac{c^2}{d^2} \; \text{ in } \; -\frac{de^2}{r^2} = 1.
\end{equation}
\opomba{Smiselno za koeficiente $b, c, d$ predpostavimo $bd - c^2 > 0$ in $d < 0$. Zato s tem postopkom ne moremo reševati poljubnih enačb četrte stopnje.}
Avtorja nato predlagata naslednji matriki za dualni stožnici $\mathcal{\hat{C}}$ in $\mathcal{\hat{P}}$:
$$ M^{-1}_\mathcal{C} =
    \begin{bmatrix}
        1 & 0 & 0\\
        0 & 1 & 0\\
        0 & 0 & -1/r^2
    \end{bmatrix}
\; \text{ in } \; M^{-1}_\mathcal{P} =
    \begin{bmatrix}
        0 & 0 & de/2\\
        0 & -d & c/2\\
        de/2 & c/2 & 0
    \end{bmatrix}.
$$
Če izračunamo njuna inverza, dobimo matriki
$$ M_\mathcal{C} =
    \begin{bmatrix}
        1 & 0 & 0\\
        0 & 1 & 0\\
        0 & 0 & -r^2
    \end{bmatrix}
    \; \text{ in } \; M_\mathcal{P} = -\frac{1}{d^3e^2}
    \begin{bmatrix}
        c^2 & -cde & -2d^2e\\
        -cde & d^2e^2 & 0\\
        -2d^2e & 0 & 0
    \end{bmatrix}.
$$
Iz njiju zapišemo predpisa stožnic v afini ravnini ($z = 1$) in dobimo:
\begin{itemize}
    \item krožnico $\mathcal{C}: x^2 + y^2 = r^2$ s središčem v $S(0,0)$ in polmerom $r$ ter
    \item parabolo $\mathcal{P}: c^2x^2 - 2cdexy + d^2e^2y^2 - 4d^2ex = 0$ (ker je $A_{M_\mathcal{P}} = 0$, je $\mathcal{P}$ res parabola).
\end{itemize}
Z določitvijo gorišča in premice vodnice parabole $\mathcal{P}$ se bomo ukvarjali pozneje; najprej algebraično poiščimo skupno tangento na dani stožnici.

Naj bo $q = [A\;B\; C]^\intercal$ presečišče dualnih stožnic $\mathcal{\hat{C}}$ in $\mathcal{\hat{P}}$. Iz sistema enačb~\ref{eq:afin_sistem_tangenta_splosen} dobimo nov sistem
\begin{equation}
    \label{eq:afin_sistem_tangenta_ABC_kvart}
    A^2 + B^2 = \frac{C^2}{r^2} \; \text{ in } \; deAC - dB^2 + cBC = 0.
\end{equation}
Z določitvijo $B = -1$ in $z = 1$ zopet dobimo afino obliko skupne tangente kot $y = Ax + C$. Ko $B$ vstavimo v sistem~\ref{eq:afin_sistem_tangenta_ABC_kvart}, prvo enačbo množimo z $de^2C^2$, drugo kvadriramo, delimo z $d$ in vstavimo v prvo enačbo, dobimo
$$ - \frac{de^2}{r^2}C^4 - (de^2 + \frac{c^2}{d})C^2 + 2cC + d = 0.$$
V dobljeno enačbo vstavimo relaciji~\ref{eq:kvarticni_relaciji} in dobimo
$$C^4 + bC^2 + 2cC + d = 0,$$
kar pomeni, da nam $C$, začetna vrednost tangente oz.\ njeno presečišče z ordinatno osjo reši izvorno kvartično enačbo.

Sedaj poiščimo še gorišče in premico vodnico parabole $\mathcal{P}$. Potem bo postopek konstrukcije skupne tangente, kot smo ga opisali že zgoraj -- najprej narišemo krožnico s središčem $S$ in polmerom $2r$ ter gorišče in premico vodnico parabole. Nato opravimo pregib, ki hkrati položi gorišče na premico vodnico in središče krožnice na njen rob. Presečišče pregiba z ordinatno osjo je rešitev naše enačbe (seveda poiščemo vse možne pregibe, ki so največ štirje).

Iz enačbe parabole $\mathcal{P}: c^2x^2 - 2cdexy + d^2e^2y^2 - 4d^2ex = 0$ ne moremo enostavno prebrati gorišča in premice vodnice, zato $(x, y)$-sistem prevedimo v \textcolor{red}{(a je izraz primeren?)} $(z, w)$-sistem s sledečo preslikavo \textcolor{red}{(od kje pride ta $P$?)}:
$$ \begin{bmatrix} x\\y\\1\end{bmatrix} = P \cdot \begin{bmatrix} z\\w\\1\end{bmatrix}, \text{ kjer je } P =
\begin{bmatrix}
    c/bd & de & c^2e/b^2\\
    -e/b & c & -(bc + cde^2)/b^2\\
    0 & 0 & 1
\end{bmatrix}.
$$

Ko to vstavimo v matrično enačbo parabole $[x\;y\;1] P [x\;y\;1]^\intercal$, po računanju in poenostavljanju\footnote{Za izračun je potrebna programska oprema, kot je npr.\ Wolfram Mathematica. \textcolor{red}{imaš zračunano, lahko vstavim tle screenshot za dokaz, da sem zračunala?}} na koncu dobimo
$$ \mathcal{P}: z^2 = 4d^3e^2w. $$

V $(z, w)$ sistemu ima parabola gorišče v $F_{zw}(0, d^3e^2)$ in premico vodnico $L_{zw}: w = - d^3e^2$ oziroma $L_{zw}: (z, w) = (0, -d^3e^2) + t(1, 0), t \in \R $. Izhodišče v $(x, y)$-sistemu je tako v točki $O_{xy} (c^2e/b^2, -(bc + cde^2)/b^2)$. Iz tega lahko izračunamo še gorišče in premico vodnico v $(x, y)$-sistemu:
$$ F_{xy} = P \cdot \begin{bmatrix} 0\\d^3e^2\\1\end{bmatrix} \longrightarrow F_{xy} = O_{xy} + d^3e^2(de, c) = (\frac{c^2e}{b^2} + d^4e^3, -\frac{bc + cde^2}{b^2} + cd^3e^2),$$
$$ L_{xy} = P \cdot \left( \begin{bmatrix} 0\\-d^3e^2\\1\end{bmatrix} + t \cdot \begin{bmatrix} 1\\0\\1\end{bmatrix} \right) \longrightarrow L_{xy}: (x, y) = O_{xy} - d^3e^2(de, c) + t(c, -de), t \in R.$$

V eksplicitni obliki je enačba za premico vodnico
$$ L_{xy}: y = -\frac{de}{c} \cdot x - \frac{1}{bc} (c^2 + bc^2d^3e^2 + bd^5e^4). $$

Kot vidimo, gorišče in premica vodnica nimata lepih predpisov, vendar je po eni strani ta metoda precej bolj zanimiva kot pretvorba kvartične enačbe na enačbe nižje stopnje in reševanje z običajno Lillovo metodo.

\textcolor{red}{lahko daš še slikovni primer (slika 4 iz istega vira) poleg konkretne enačbe, samo za okus. Sta pa slika 3 in slika 4 povezani med sabo, gre za iste podatke.}

\subsection{Alperinova rešitev}

Za konec si poglejmo še eno zelo enostavno metodo, ki nam rešuje kubične enačbe. Lahko bi jo opisali že pred razdelkom~\ref{podpogl:kubicna_enacba}, vendar je zaradi rahle povezave s postopkom v prejšnjem razdelku zapišemo tu. Alperin v~\cite[str.\ 129]{alperin2000} sicer namesto splošne kubične enačbe vzame zreducirano obliko $x^3 + cx + d = 0$\footnote{V splošno kubično enačbo $x^3 + bx^2 + cx + d = 0$ vpeljemo novo spremenljivko $z = x - (1/3)b$. Bralec lahko za vajo izračuna, da s tem dobimo splošno kubično enačbo za $z$ brez kvadratnega člena.}, za paraboli pa enačbi kot v prejšnjem razdelku pri reševanju kubične enačbe, vendar s spremenljivkami, pomnoženimi z $-1/2$:
\begin{itemize}
    \item $\mathcal{P}_1: \left(y - \frac{c}{2}\right)^2 = 2dx$ z goriščem v točki $(\frac{d}{2}, \frac{c}{2})$ in premico vodnico $x = -\frac{d}{2}$ in
    \item $\mathcal{P}_2:  x^2 = 2x$ z goriščem v točki $(0, \frac{1}{2})$ in premico vodnico $y = -\frac{1}{2}$.
\end{itemize}
Avtor nato opravi Belochin pregib, ki gorišče vsake parabole položi na njeno premico vodnico in tako konstruira skupno tangento. Za njegovo enačbo obstaja le en tak pregib. \textcolor{red}{Zakaj, a je samo ena realna rešitev?}
Sedan pa se ne poslužimo uporabe matrik kot v zgornjem postopku, temveč analitično izrazimo koeficient skupne tangente. Naj bo $k$ iskani koeficient ter $(x_0, y_0)$ točka tangentnosti na prvi in $(x_1, y_1)$ točka tangentnosti na drugi paraboli. Iz enačbe prve parabole z implicitnim odvajanjem dobimo
$$ 2\left(y - \frac{c}{2}\right) \frac{dy}{dx} = 2d, \; \text{ torej } \; y_0 = \frac{d}{k} + \frac{c}{2} \; \text{ in } \; x_0 = \frac{\left(y_0 - \frac{c}{2}\right)^2}{2d} = \frac{d}{2k^2},$$
iz enačbe druge parabole pa
$$ \frac{dy}{dx} = x, \; \text{ torej } \; x_1 = k \; \text{ in } \; y_1 = \frac{k^2}{2}. $$
Ko koordinate vstavimo še v klasično enačbo za koeficient premice skozi dve točki, dobimo
$$ k = \frac{y_1 - y_0}{x_1 - x_0} = \frac{\frac{k^2}{2} - \frac{d}{k} - \frac{c}{2}}{k - \frac{d}{2k^2}}.
$$
Enačbo poenostavimo in res dobimo $k^3 + ck + d = 0$, torej je koeficient skupne tangente rešitev izvorne zreducirane kubićne enačbe.
\newpage
\section{Alhazenov problem}

Spoznajmo zanimiv problem, ki izhaja že iz stare Grčije. Čez poglavje si bomo pogledali več njegovih različic in poskusili najti origami konstrukcijo njegove rešitve.

\subsubsection*{Starogrški izvor problema}

V osnovi gre za problem s področja optike, ki naj bi ga zastavil grški matematik Ptolemaj (prb.\ 85--170 po Kr.): \emph{Pri danem sferičnem zrcalu in viru svetlobnega žarka poišči točko na zrcalu, od katere se bo svetlobni žarek odbil v oko opazovalca} (slika~\ref{fig:ptolemaj}).

\begin{figure}[h]
    \centering
    \includegraphics[width=0.25\textwidth]{images/alhazen/ptolemajev_problem.png}
    \caption[Ptolemajev problem]{Ptolemajev optični problem.}
    \label{fig:ptolemaj}
\end{figure}

Rešitev se da preformulirati v iskanje točke na krožnici, v kateri polmer krožnice razpolavlja kot, ki ga opravi svetlobni žarek. To velja zaradi \emph{odbojnega zakona}, vendar tega v takratni Grčiji še niso poznali. Pojdimo na začetek in opazujmo, kako se je oblika Ptolemajevega problema spreminjala skozi čas.

Grški matematik Heron iz Aleksandrije, med drugim znan po svoji Heronovi formuli za izračun ploščine trikotnika z danimi dolžinami stranic, je okoli leta 100 po Kr.\ zastavil in rešil naslednje vprašanje, ki je različica Ptolemajevega problema: \emph{Na isti strani premice ležita točki $A$ in $B$. Poišči točko $C$ na premici, da bo pot od točke $A$ do $B$ preko točke $C$ najkrajša}\footnote{Predpostavimo evklidsko metriko.}.

Sam točko $C$ konstruira zelo enostavno -- najprej točko $B$ zrcali čez premico v točko $B'$, nato pa s točko $C$ označi presečišče premice in daljice $AB'$ (slika~\ref{fig:heron}). Ker velja $|CB| = |CB'|$, je $|AC| + |CB| = |AC| + |CB'| = |AB'|$. Dolžina $|AB'|$ najkrajša možna razdalja med točkama $A$ in $B'$, zato je točka $C$ rešitev vprašanja.

\begin{figure}[h]
    \centering
    \includegraphics[width=0.4\textwidth]{images/alhazen/heron.png}
    \caption[Heronovo vprašanje]{Heronova konstrukcija najkrajše poti od točke $A$ do točke $B$.}
    \label{fig:heron}
\end{figure}

Zgornja konstrukcija pa je grškemu matematiku ponudila še nekaj -- opazil je, da pravokotnica na premico skozi točko $C$ namreč razpolavlja kot $\angle ACB$ (gl.\ rdeče oznake na sliki~\ref{fig:heron}). Tako lahko njegovo vprašanje preoblikujemo v: \emph{Na isti strani premice ležita točki $A$ in $B$. Poišči točko $C$ na premici, ki razpolavlja kot $\angle ACB$.}

\opomba{Bralec je prepuščen enostaven premislek, kako konstruirati točko $C$ z origamijem.}

\subsubsection*{Formulacija Alhazenovega problema}

S tem novim znanjem je Heron postavil temelje, na katerih je več stoletij kasneje matematik, astronom in fizik Ibn al-Haytham oz.\ po naše Alhazen (prb.\ 965--1040 na območju današnjega Iraka) formuliral odbojni zakon, ki pravi, da sta vpadni in odbojni kot žarka svetlobe od površja enaka. Alhazen se je tudi prvi bolj poglobil v Ptolemajev problem in prišel do pomembnih ugotovitev, zato problem ponekod imenujejo tudi Ptolemaj-Alhazenov problem\footnote{\emph{Ptolemy-Alhazen's problem}, op.\ prev.}, tu bomo uporabili imenovanje \emph{Alhazenov problem}\footnote{\emph{Alhazen's problem}, op.\ prev.}. 

Zopet formulirajmo problem. Namesto premice imamo torej krožnico $\mathcal{K}$ s središčem $O$. Naj točki $A$ in $B$ ležita znotraj krožnice (za točki zunaj krožnice je reševanje problema enako, je pa Alhazen originalno predpostavil ta položaj). Iščemo točko $C$ na krožnici $\mathcal{K}$, da se bo svetlobni žarek iz točke $A$ v točki $C$ odbil v točko $B$. Na enak način kot Hero lahko premislimo, da mora polmer $OC$ razpolavljati kot $ACB$ (slika~\ref{fig:alhazen1}).

\begin{figure}[h]
    \centering
    \includegraphics[width=0.6\textwidth]{images/alhazen/alhazen1.png}
    \caption[Alhazenov problem]{Ena od rešitev Alhazenovega problema v splošnem.}
    \label{fig:alhazen1}
\end{figure}

Za razliko od Heronove konstrukcije tu ne znamo konstruirati točke $B'$, saj je premica, čez katero bi morali zrcaliti točko $B$ (in tako s presečiščem daljice $AB'$ in premice dobiti točko $C$), ravno tangenta v točki $C$. Premice torej ne moremo dobiti brez točke $C$, točke $C$ pa ne moremo dobiti brez te premice. Tu se pokaže razsežnost problema.

Že tisoč let nazaj je Alhazen pokazal, da se da problem rešiti s stožnicami. V dolgem in za marsikaterega kasnejšega matematika zapletenem dokazu je pokazal, da so izkane točke odboja preseki te krožnice z določeno hiperbolo~\cite{wilk2015}. Algebraično rešitev je l.\ 1965 našel Jack M.\ Elkin, l.\ 1989 je problem rešil še Harald Riede, za njim l.\ 1997 pa še Peter M.\ Neumann. A pojdimo po vrsti.

\opomba{Opazimo lahko tudi, da ima zaradi enakega vpadnega in odbojnega kota žarka v točka $C$ lastnost točke, ki leži na \emph{elipsi} z goriščema $A$ in $B$ (na sliki~\ref{fig:alhazen1} označena s prekinjeno črto). Tangenta na krožnico v točki $C$ je hkrati tudi tangenta na elipso v isti točki, torej lahko problem preoblikujemo v iskanje vseh elips z goriščema v $A$ in $B$, ki so tangentne na krožnico $\mathcal{K}$. Vendar ne poznamo postopka, ki bi nam lahko poiskal enačbo te elipse, saj je ta odvisna od točke $C$, ta pa od tangente in obratno. Torej smo zopet na istem.}

\subsubsection*{Prevedba problema na reševanje kvartične enačbe}

Alhazen je med drugi ugotovil, da se da problem prevesti tudi v reševanje enačbe četrte stopnje. Naslednji nastavek za izpeljavo je vzeta iz~\cite[138--139]{geometricconstructions}. Brez škode za splošnost predpostavimo, da je krožnica $\mathcal{K}$ enotska, $O = (0,0), A = (0,a)$ in $B=(b,c)$ za take $0 \leq a, b, c \leq 1$, da te točke med sabo paroma ne sovpadajo (slika~\ref{fig:alhazen2}). Naj bo $C=(x,y)$ iskana točka na krožnici $\mathcal{K}$. Iz tega sledi pogoj
\begin{equation}
    \label{eq:pogoj_alh1}
    x^2 + y^2 = 1.
\end{equation}

\begin{figure}[h]
    \centering
    \includegraphics[width=0.6\textwidth]{images/alhazen/alhazen2.png}
    \caption[Alhazenov problem -- izpeljava]{Podlaga za izpeljavo kvartične enačbe za Alhazenov problem.}
    \label{fig:alhazen2}
\end{figure}

Naj bodo premice $AC$, $OC$ in $BC$ nosilke istoimenskih daljic in naj bo za vsako premico $i$ s $k_i$ označen njen koeficient. Dobimo
\begin{align*}
    &k_{AC} = \frac{y-a}{x}, \\
    &k_{OC} = \frac{y}{x}, \\
    &k_{BC} = \frac{y-c}{x-b}.
\end{align*}
Označimo $\alpha = \angle ACO = \angle OCB$. Ker sta kota enaka, velja
\begin{align*}
    \tan \alpha &= tan \alpha, \\
    \frac{k_{AC} - k_{OC}}{1 + k_{AC} k_{OC}} &= \frac{k_{OC} - k_{BC}}{1 + k_{OC} k_{BC}}, \\
    \frac{\frac{y-a}{x} - \frac{y}{x}}{1 + \frac{y-a}{x} \cdot \frac{y}{x}} &= \frac{\frac{y}{x} - \frac{y-c}{x-b}}{1 + \frac{y}{x} \cdot \frac{y-c}{x-b}}.
\end{align*}
Slednjo enačbo poenostavimo in z upoštevanjem pogoja~\ref{eq:pogoj_alh1} dobimo
$$ y(b + 2acx) = ab + (a+c)x - 2abx^2.$$
Enačbo kvadriramo, spet upoštevamo pogoj~\ref{eq:pogoj_alh1} in dobimo enačbo četrte stopnje (\textcolor{red}{si poračunala, tudi z wolframom, je na enem listu}):
$$ 4a^2(b^2 + c^2)x^4 - 4a^2bx^3 + (a^2 - 4a^2b^2 + b^2 + c^2 - 4a^2c^2 + 2ac)x^2 + 2ab(a-c)x + b^2(a^2-1) = 0.$$

Enačbe četrte stopnje v teoriji znamo rešiti tako računsko kot z origamijem, vendar nas to delo že na prvi pogled mogoče odbija.

\subsubsection*{Origami konstrukcija rešitve za poseben primer}

Za poseben primer izbire točk $A$ in $B$ pri dani krožnici na naše veselje vendarle obstaja zelo enostavna konstrukcija točke $C$. Scimemi v~\cite[str.\ 116-117]{scimemi2002} poda naslednji postopek (gl.\ sliko~\ref{fig:scimemi}):
\begin{enumerate}
    \item Naj bo $O$ središče krožnice $\mathcal{K}$ in $A$ točka na njej. Znotraj krožnice (lahko tudi na njej) si izberemo točko $B$, ki ne sovpada s prejšnjima točkama.
    \item Konstruiramo premico $a$ skozi točki $O$ in $A$ ter njeno pravokotnico $b$ v točki $A$.
    \item Točko $B$ zrcalimo čez središče $O$ v točko $D$.
    \item Opravimo Belochin pregib $p$, ki točko $D$ postavi na premico $a$, točko $B$ pa na premico $b$.
    \item Konstruiramo pravokotnico $c$ na pregib, ki poteka skozi točko $A$. Njeno drugo presečišče s krožnico $\mathcal{K}$ označimo s $C$.
\end{enumerate}

\begin{figure}[h]
    \centering
    \includegraphics[width=0.6\textwidth]{images/alhazen/scimemi.png}
    \caption[Scimemijeva rešitev]{Scimemijeva rešitev Alhazenovega problema, ko točka $A$ leži na krožnici.}
    \label{fig:scimemi}
\end{figure}

\begin{trditev}
    Točka $C$ je rešitev Alhazenovega problema za točke $O, A, B$.
\end{trditev}

\opomba{Avtor je v svoje delo vključil tudi dokaz, vendar je v njem več nejasnosti, priložena skica pa je zavajajoča. Zato je tu drug, lažji dokaz \textcolor{red}{(ki sem ga iznašla sama -- a lahko pišem tko v prvi osebi?)}.}

\begin{dokaz}
    Naj bo točka $A'$ zrcalna slika točke $A$ čez pregib $p$. Zarišemo še daljici $BA'$ in $DA'$. Potem zaradi simetrije čez pregib $p$ velja $\angle BA'D = \angle B'AD' = 90^\circ$ (slika~\ref{fig:scimemi_dokaz} levo).

    Naj bo $S$ presečišče premice $CO$ z daljico $A'B$. Ker točki $P$ in $A$ ležita na krožnici $\mathcal{K}$, je trikotnik $\triangle CAO$ z vrhom v središču $O$ enakokrak. Sledi $\angle OCA = \angle CAO = \angle CA'D$. Zadnja enakost sledi iz sovršnih kotov ob točki $A$ in simetrije čez pregib $p$. Torej sta kota z vrhom v točkah $P$ in $A'$ (na sliki~\ref{fig:scimemi_dokaz} desno) izmenična, iz česar sledi $CS \parallel DA'$. Zato velja $\angle OSB = \angle DA'B = 90^\circ$.

    \begin{figure}[h]
        \centering
        \includegraphics[width=0.47\textwidth]{images/alhazen/scimemi_dokaz1.png}
        \includegraphics[width=0.47\textwidth]{images/alhazen/scimemi_dokaz2.png}
        \caption[Dokaz Scimemijeve konstrukcije]{Geometrijski dokaz Scimemijeve konstkrukcije}
        \label{fig:scimemi_dokaz}
    \end{figure}

    Trikotnika $\triangle OSB$ in $\triangle DA'B$ sta zato podobna, ker pa je $O$ središče hipotenuze večjega trikotnika, je prvi dvakrat manjši, torej $|BS| = |SA'|$.

    Zaključimo, da sta zaradi skladnih katet pravokotna trikotnika $\triangle CSB$ in $\triangle CSA'$ skladna, torej res velja $\angle SCB = \angle SCA' = \angle SCA$ oziroma daljica $OC$ res razpolavlja kot $\angle ACB$.
\end{dokaz}

\opomba{Zelo je zanimiv stranski produkt te konstrukcije -- izkaže se namreč, da točke $A, B$ in $E$ (kjer je $E$ drugo presečišče poltraka $PB$ s krožnico $\mathcal{K}$, gl.\ sliko~\ref{fig:scimemi_opomba}) ter njihove slike vse ležijo na isti krožnici! Središče te krožnice pa je presečišče poltraka $CO$ s pregibom $p$. Prav tako velja $BA' \parallel EA$ in $|BE| = |A'A|$. Dokaz za to nalogo ni ključen in ga prepuščamo bralcu.}

\begin{figure}[h]
    \centering
    \includegraphics[width=0.6\textwidth]{images/alhazen/scimemi_stransko.png}
    \caption[Stranski produkt Scimemija]{Stranski produkti Scimemijeve konstrukcije točke $P$ (označeno z rdečo).}
    \label{fig:scimemi_opomba}
\end{figure}
\newpage
% \section{Origami konstrukcije z več hkratnimi prepogibi}

\textcolor{red}{Kvintična enačba}

\textcolor{red}{petinjenje kota}
% \newpage
\section{Zaključek}

Po zgledu evklidskih konstrukcij smo v nalogi spoznali, da lahko geometrijske in algebraične probleme rešujemo tudi s prepogibanjem papirja, pri čemer je origami celo močnejše orodje od neoznačenega ravnila in šestilo. Poleg kosntrukcij vseh razdalj, ki jih lahko opravijo z evklidskih orodjem, znamo tudi z origamijem zrcaliti in vrteti točke ter prenašati razdalje, korak naprej iz evklidskih konstrukcij pa je zmožnost izvajanja operacije kubičnega kubiranja in trisekcije poljubnega kota ter prepogibanja skupnih tangent na nekatere stožnice.

Zaradi tega nam prepogibanje papirja lahko reši tudi kubične in kvartične enačbe ter konstruira več pravilnih večkotnikov kot nam ga lahko evklidsko orodje. Zaslugo za nadvlado origamija nad njim ima Belochin pregib, ki edina od origami operacij ne more biti konstruirana z neoznačenim ravnilom in šestilom.

V nalogi smo se osredotočili le na origami z enkratnimi prepogibi. Če bi dovolili opravljanje več pregibov zapored in šele nato razgrnitev papirja, se nam s tem odpre še veliko več možnosti za raziskovanje, kaj lahko z origamijem v matematiki počnemo. Med drugim lahko poljuben kot razdelimo na pet skladnih delov in rešujemo enačbe pete stopnje. Znanost se v zadnjih petdesetih letih z origamijem kot pripomočkom za raziskovanje mateamtike veliko ukvarja in tudi aktivno aplicira v naš vsakdan. Primer je efektivno zlaganje objektov samih vase, med drugim satelitov, da je strošek pošiljanja v vesolje čimmanjši.

Origami je odličen primer, kako se lahko v običajnih stvareh okoli nas skriva čista matematika. Ker je papir s prepogibi in njihovimi presečišči odličen model evklidske ravnine, je kot didaktičen pripomoček za popestritev pouka origami zelo priporočljiv. Učenci lahko z njim preko različnih pristopov raziskujejo matematiko in razvijajo abstraktno in kritično mišljenje. Iz opazovanja konstrukcije lahko sami postavljajo hipoteze in jih dokažejo ali ovržejo; lahko sami raziskujejo, kako s prepogibanjem pokazati kakšne geometrijske lastnosti. Učitelj jih mora znati usmerjati, hkrati pa jim dati svobodo raziskovanja, saj se ravno s tem krepi samostojnost, odgovornost in veselje do izzivov.

\end{document}